\chapter{Introduction}
%---------%---------%---------%---------%---------%---------%---------%---------

\label{Introduction} 
\lhead{\emph{Introduction}} 
%---------%---------%---------%---------%---------%---------%---------%---------
    
    In this thesis we review a complete and self-contained proof of the 
    Pontryagin's Maximum Principle \citep{YongDG_ACIntro}. To proof this result
    we apply the so-called Ekeland's $\veps$-Variational Principle
    \citep{guler2010foundations}. We also reproduce the simulation of some 
    literature examples which follow the optimal control framework of Lenhart
    \citep{lenhart2007optimal}. That is, control a given dynamics with linear 
    terms and approximate the optimally policies with the 
    forward-backward sweep method. 
%---------%---------%---------%---------%---------%---------%---------%---------

%---------%---------%---------%---------%---------%---------%---------%---------
    In the literature of optimal control applied to biological models, contingent 
    policies such as vaccination, quarantine, isolation, treatment, among others,
    are naturally described by control terms \citep{SARS,lenhart2007optimal, PanettaFister, butler1997optimal, BearSalinasLenhart,articleLenhart}. This linear form 
    simplifies the characterization of optimal policies. In short, a policy is a 
    function that prescribes which actions apply according to information. If the 
    control policy only depends on time, then this policy is of open-loop. In the 
    other way, if it depends on the current state, then it is of closed-loop.  
    Thus, if a policy optimizes a given cost functional \textemdash a function 
    from $\mathbb{R}^n$ to $\mathbb{R}$ that describes the resource consumption 
    and the profit generation \textemdash, then this policy is optimal. For 
    example, in [reference] represents a vaccination campaign as a control policy 
    and the cost functional describes the balance between the necessary money to 
    run the campaign and the number of infected individuals to be minimized.
%---------%---------%---------%---------%---------%---------%---------%---------

%---------%---------%---------%---------%---------%---------%---------%--------- 
    First, we have to ensure the existence of an optimal policy. To this end, 
    we appeal to the theorems Arzela-Ascoli and Banach-Sacks, to the Filippov
    lemma and some other results. Newt we apply the Pontryagin's Maximum 
    Principle to characterize optimal control. That is, we get the necessary 
    conditions to approximate the optimal policy. However, some problems 
    lacks of unique optimal policies, which still is, an open problem.
%---------%---------%---------%---------%---------%---------%---------%---------

%---------%---------%---------%---------%---------%---------%---------%---------   
    The aim of this thesis is  to review the existence and characterizations of
    the underlying solution to optimal control problems with applications to
    biology and to approximate the regarding optimal policies. 
%---------%---------%---------%---------%---------%---------%---------%---------

%---------%---------%---------%---------%---------%---------%---------%---------
    After of this introduction, in Chapter 2 we introduce and prove the Ekeland's
    $\veps$-Variational Principle, which will give the existence of an approximate
    control. Chapter 3 presents the necessary theory to ensure the existence 
    of an optimal control. In Chapter 4 we enunciate and prove the 
    Pontryagin's maximum principle. Chapter 5 discusses the forward-backward sweep 
    method , which approximate the optimal policies. 
    Chapter 6 exposes the one dimensional control problems with one control 
    on their dynamics. 
    Chapter 7 describes the multidimensional control problems with one control and two
    controls on their dynamics 
    We closed this work with the conclusion and perspectives in Chapter 8. 
    
    
%---------%---------%---------%---------%---------%---------%---------%---------

%---------%---------%---------%---------%---------%---------%---------%---------
\newpage

