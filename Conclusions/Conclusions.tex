\chapter{Conclusions and perspectives}
%---------%---------%---------%---------%---------%---------%---------%---------
%---------%---------%---------%---------%---------%---------%---------%---------
    We have reviewed the proof related to the so called Pontryagin's Maximum 
    Principle. Appealing to the Ekeland's variational principle and other 
    auxiliary results we have clarified most of the details of the proof of this 
    seminal principle. According to the Pontryagin principle and its proof we have 
    understood the forward-backward sweep method. Consequently, we made a GitHub 
    repository \citep{python_Thesisrepo}, with the Python implementation code 
    that approximates the solution of optimally controlled biological models 
    reported in the literature. 
%---------%---------%---------%---------%---------%---------%---------%---------

%---------%---------%---------%---------%---------%---------%---------%---------
    Following the ideas of Suzanne Lenhart \citep{lenhart2007optimal}, we have 
    presented controlled models reported in literature. In each of them we 
    explained the formulation with linear control and, using the forward backward 
    sweep method, we optimized each functional. In the completion of this thesis 
    we have detected other ways to optimize the underlying functional cost. One 
    alternative that we have explored is the so-called differential evolution 
    optimization method, \citep{python_Thesisrepo}.
%---------%---------%---------%---------%---------%---------%---------%---------

%---------%---------%---------%---------%---------%---------%---------%---------
    All the above examples of the optimal control theory involves open-loop
    controls. These kind of models work under the following assumptions. 
    \begin{inparaenum} [i)]
        \item The model is perfect,
        \item there is no disturbance and,
        \item the parameters and inputs are known accurately.
    \end{inparaenum}
    However these assumptions are unrealistic. The authors in [referencia] report that 
    closed-loop controls would be more realistic. Although, it is difficult to 
    obtain optimal closed-loop controls for nonlinear systems, even so there 
    is a way to do it using the Bellman's Equation.
%---------%---------%---------%---------%---------%---------%---------%---------

%---------%---------%---------%---------%---------%---------%---------%---------
    With this in mind, we note two main ways to extend this work. The first one is 
    to study the closed-loop controls and its applications. The latter is to 
    change the kind of dynamics \textemdash discrete or continuous, deterministic 
    or stochastic. Each dynamic has its own theory and provides tools for a 
    great spectre of applications.
%---------%---------%---------%---------%---------%---------%---------%---------

%---------%---------%---------%---------%---------%---------%---------%---------