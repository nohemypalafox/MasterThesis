
\section{Ekeland's $\epsilon -$Variational Principle} 
%\label{Chap1-Sect5} 
%---------%---------%---------%---------%---------%---------%---------%---------
%---------%---------%---------%---------%---------%---------%---------%---------

    The purpose of this section is to establish the Ekeland's variational 
    principle {\cite{guler2010foundations}} for this we need the concept of 
    semicontinuous functions which will be defined over metric spaces $(E,d)$. 
    Also, we  define the set of extended real numbers as
    $$
        \extRealp := \{- \infty \} \cup \mathbb{R} \cup \{ + \infty \}.
    $$
    The set of extended real numbers is ordered and we can define the following 
    operations additionally to the usual operations over $\mathbb{R}$
    \begin{align*}
        x \in \mathbb{R} \cup \{+\infty\} &\implies x + (+ \infty) = +\infty, \\
        x \in \mathbb{R} \cup \{-\infty\} &\implies x + (- \infty) = -\infty, \\
        x > 0 &\implies x(+ \infty) = + \infty, \\
        x < 0 &\implies x(+ \infty) = - \infty, \\
        (- \infty)(+ \infty) &= -\infty, \\
        (- \infty)(- \infty) &= (+ \infty)(+ \infty) = + \infty, \\
        0(+ \infty) &= 0(-\infty) = 0.
    \end{align*}


    \begin{definition}
        Let $\{ x_n \}$ be a sequence of extended real numbers, that is 
        $x_n \in \extRealp$. The limit inferior of $\{ x_n \}$ is
        $$
            \liminf_{n \to \infty} x_n := \lim_{n \to \infty} \inf_{k \geq n}{x_{k}} %
            = \sup_{n} \inf_{k \geq n}{x_k},
        $$
        where the second equality follows since $\{ \inf_{k \geq n}{x_k} \}$ 
        is an increasing sequence in n. Similarly, the limit superior of 
        $\{ x_n \}$ is
        $$
            \limsup_{n \to \infty} x_n := \lim_{n \to \infty} \sup_{k \geq n}{x_{k}} %
            = \inf_{n} \sup_{k \geq n}{x_k}.
        $$
    \end{definition}
    
    \begin{definition}
        Let $f : E \to \extRealp $ be an extended real-valued function. The limit
        inferior of f as $x \in E$ converges to $x_0 \in E$  is defined by 
        $$
            \liminf_{x \to x_0} f(x) := \lim_{\delta \to 0} %
            \inf_{d(x, x_0) < \delta} f(x) = \sup_{\delta} \to 0 \inf_{d(x, x_0) < \delta} f(x),
        $$
        and its limit superior by 
        $$
            \limsup_{x \to x_0} f(x) := \lim_{\delta \to 0} %
            \sup_{d(x, x_0) < \delta} f(x) = \inf_{\delta} \to 0 \sup_{d(x, x_0) < \delta} f(x).
        $$
    \end{definition}
    
    \begin{lemma}
        Let $f : E \to \extRealp $. We have
        $$
            \liminf_{x \to x_0} f(x) = \inf_{\{x_n\}}{\liminf_{n \to \infty}f(x)},
        $$
        where the infimum on the right-hand side is taken over all sequences 
        $x_n \to x_0$. Similarly, 
        $$
            \limsup_{x \to x_0} f(x) = \sup_{\{x_n\}}{\limsup_{n \to \infty}f(x)}.
        $$
    \end{lemma}
    \begin{proof}
        Define 
        \begin{align*}
            M &:= \liminf_{x \to x_0} f(x),  \\
            L &:= \inf_{x_n}{\liminf_{n \to \infty} f(x_n)}, \\
            N_{\delta} &:= \{ x \in E : d(x, x_0) < \delta \}.
        \end{align*}
        We have to consider the cases $M = - \infty$, $M = \infty$ and 
        $ -\infty < M < \infty $. \\
        \underline{Case $M = -\infty$:}
        Note that it is enough to prove the 
        existence of a sequence $x_n \to x_0$ such that $f(x_n) \to - \infty$. 
        Because, we can write
        $$
            \liminf_{n \to \infty}{f(x_n)} = \lim_{n \to \infty} %
            \inf \{ f(x_n), f(x_{n+1}), \ldots \} = - \infty,
        $$
        and, any other sequence $y_n$ that converges to $x_0$ satisfies
        $$
            \liminf_{n \to \infty}f(y_n) \geq - \infty.
        $$
        By  the above 
        $$
            \inf_{x_n} \liminf_{n \to \infty} f(x_n) = -\infty.
        $$
        Since, $M = -\infty$ we have that
        $$
            M = \liminf_{x \to x_0} f(x) = \lim_{\delta \to 0} %
            \inf_{x \in N_{\delta}} f(x) = - \infty .
        $$
        Let $\delta = \frac{1}{n}$, for all $n \in \mathbb{N}$, then
        $$
            \inf_{x \in N_{1/n}} f(x) = -\infty, \forall n \in \mathbb{N}.
        $$
        Thus, there is $y_n \in N_{1/n}$ such that $f(y_n) < -n$. We take the 
        sequence defined by 
        $$
            x_n := y_n, \quad y_n \in  N_{1/n} \text{ for each } n \in \mathbb{N}.
        $$
        By the above, there is a sequence $x_n \to x_0$ such that 
        $f(x_n) \to - \infty$. Hence, $M = L$. \\
        \underline{Case $M = \infty$:} Since,
        $$
            \lim_{\delta} \inf_{x \in N_{\delta}} f(x) = \infty, 
        $$
        for a given $\epsilon > 0 $ there exists $\delta > 0 $ such that 
        $$
            \inf_{x \in N_{\delta}} f(x) > \epsilon.
        $$
        By the convergence of $x_n$, there is $N \in \mathbb{N}$ such that 
        $x_n \in N_{\delta}$ for all $n \geq N$. Then, 
        $$
            f(x_n) \geq \inf_{x_n \in N_{\delta}} f(x_n) > %
            \epsilon \quad \forall n \geq N
        $$ 
        That is, $f(x_n) \to \infty$ for any sequence $x_n \to x_0$. By the above
        we conclude $L = M$.
        \underline{Case $ -\infty < M < \infty$ :} \\
        First, we prove that $M \leq L$. By definition, given $\epsilon>0$ there
        is $\delta >0$ such that $\inf_{x \in N_{\delta}} f(x)$, this implies that
        $f(x) > M - \epsilon$ for all $x \in N_{\delta}$. Let $\{x_n\}$ be a sequence
        that converges to $x_0$. Thus, there is $N \in \mathbb{N}$ such that 
        $x_n \in N_{\delta}$ for all $n \in \mathbb{N}$ and
        $$
            f(x_n) > M - \epsilon \quad \forall x_n \in N_{\delta}, n \geq N.
        $$
        Then
        $$
            \liminf_{n \to \infty} f(x_n) = \lim_{n \to \infty} \inf_{x_n \in N_{delta}} %
            \geq M - \epsilon.
        $$
        The above holds for any sequence $x_n \to x_0$, then
        $$
            \inf_{\{x_n\}} \liminf_{n \to \infty} f(x_n) \geq M - \epsilon,
        $$
        for any $\epsilon > 0$. Hence, $M \leq L$.
        The reverse inequality 
    \end{proof}
%    \begin{definition}
%      lower semiconinuous functions
%   \end{definition}
%---------%---------%---------%---------%---------%---------%---------%---------
%---------%---------%---------%---------%---------%---------%---------%---------    
\begin{theorem}[{\cite[Thm.3.2*]{guler2010foundations}}]
    Let $(M, d)$ be a complete metric space, and let $f : M \to \extRealp$ be 
    a proper lower semicontinuous function that is bounded from below. Then, 
    for every $\epsilon > 0$, $\lambda > 0$, and $x \in M$ such that
    $$ 
        f(x) \leq \inf_{M} f + \epsilon,
    $$
    there exists an element $x_{\epsilon}\in M$ satisfying the following three 
    properties:
    \begin{align}
        & f(x_{\epsilon}) \leq f(x), \notag \\
        & d(x_{\epsilon}, x) \leq \lambda,  \\
        & f(x_{\epsilon}) < f(z) + \frac{\epsilon}{\lambda}d(z,x_{\epsilon})  %
         \text{ for all } z\in M, z \neq x_{\epsilon}\notag 
    \end{align}
\end{theorem}
