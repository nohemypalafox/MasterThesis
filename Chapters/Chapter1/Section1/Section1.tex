\section{Preliminaries}

%---------%---------%---------%---------%---------%---------%---------%---------
%---------%---------%---------%---------%---------%---------%---------%---------
\begin{theorem}[Interior Maximum Theorem, %
{\citep[Thm. 19.4, p.~209]{bartle1982elements}}]
    Let $c$ be an interior point of the domain of f, at which $f$ has a relative 
    maximum. If the derivative of $f$ at $c$ exists, then it must be equal to zero.
\end{theorem}
%---------%---------%---------%---------%---------%---------%---------%---------
%---------%---------%---------%---------%---------%---------%---------%---------


\begin{theorem}[Rolle's Theorem,%
{\citep[Thm. 19.5, p.~209]{bartle1982elements}}]
    Suppose that $f$ is continuous on a closed interval $J = [a,b]$, that the 
    derivative $f'$ exists in the open interval $(a,b)$, and that 
    $f(a) = f(b) = 0$. Then there exists a point $c \in (a,b)$ such that 
    $f'(c) = 0$.
\end{theorem}
\begin{proof}

\end{proof}
%---------%---------%---------%---------%---------%---------%---------%---------
%---------%---------%---------%---------%---------%---------%---------%---------
\begin{theorem}[Mean Value Theorem, %
{\cite[Thm. 19.6, p.~210]{bartle1982elements}}]
	Suppose that $f$ is continuous on a closed interval $J = [a,b]$ and 
	differentiable on the open interval $(a,b)$. Then there exists $c \in (a,b)$
	such that 
	$$
	    f(b) - f(a) = f'(c)(b - a).
	$$
\end{theorem}
\begin{proof}
    Supose that f is continuous on a closed interval $J=[a,b]$
\end{proof}


%---------%---------%---------%---------%---------%---------%---------%---------
\begin{lemma}[Fatou's Lemma, {\citep[Thm. 4.8, p.~33]{bartle2014elements}}]
    If $(f_n)$ belongs to $M^{+}(X,X)$, then
    $$
        \int \liminf f_n d\mu \leq \liminf \int f_n d\mu
    $$
\end{lemma}
%---------%---------%---------%---------%---------%---------%---------%---------
\begin{corollary}[{\citep[Thm. 5.4, p.~43]{bartle2014elements}}]
    If f is measurable, g is integrable and $|f| \leq |g|$, then f is integrable
    and
    $$
        \int |f| d\mu \leq  \int |g| d\mu
    $$
\end{corollary}
%---------%---------%---------%---------%---------%---------%---------%---------
\begin{theorem}[{\citep[Thm. 5.5, p.~43]{bartle2014elements}}]
    A constant multiply $\alpha f$ and a sum $f + g$ of functions in $L$ belongs
    to $L$ and
    \begin{align*}
         \int \alpha fd\mu &= \alpha \int f d\mu \\
         \int (f + g)d\mu  &= \int fd\mu + \int gd\mu
    \end{align*}
\end{theorem}
%---------%---------%---------%---------%---------%---------%---------%---------
\begin{theorem}[Lebesgue Dominated Convergence Theorem, %
{\citep[Thm. 5.6, p.~44]{bartle2014elements}}]
    Let $(f_n)$ be a sequence of integrable functions which converges almost
    everywhere to a real-valued measurable function $f$. If the exists an 
    integrable function $g$ such that $|f_n| < g$ for all n, then f is integrable
    and
    $$
    \int f d\mu = \lim \int f_n d\mu
    $$
\end{theorem}
\begin{proof}

\end{proof}
%---------%---------%---------%---------%---------%---------%---------%---------
\begin{corollary}[{\citep[Thm. 5.9, p.~46]{bartle2014elements}}]
    Suppose that for some $t_0 \in [a,b]$,  the function $x \rightarrow %
    f(x, t_0)$ is integrable on $X$, that $\partial f / \partial t$ exists on 
    $X \times [a,b]$, and that there exists an integrable function g on $X$ such 
    that
    $$
        \left|\dfrac{\partial f}{\partial t} (x,t) \right| \leq g(x).
    $$
    Then the function $F(t) = \displaystyle \int f(x,t)d\mu(x)$ is differentiable 
    on $[a,b]$ and
    $$
         \dfrac{dF}{dt} = \dfrac{d }{dt} 
         \int f(x,t)d\mu(x)  = \int \dfrac{\partial f}{\partial t} f(x,t)d\mu(x)
    $$
\end{corollary}
%---------%---------%---------%---------%---------%---------%---------%---------
%---------%---------%---------%---------%---------%---------%---------%---------
\begin{theorem}[Taylor's Theorem, {\citep[Thm.4, p.~391]{SpivakCalculus}}]
    Suppose that $f', \ldots f^{(n+1)}$, are defined on $[a,x]$ and 
    that $R_{n,a}(x)$ is defined by
    $$
        f(x) = f(a) + f'(a)(x-a) + \dots +
        \frac{f^{(n)}(a)}{n!}(x-a)^{n} + R_{n,a}(x). 
    $$
    Then
    \begin{enumerate}
        \item[i)]
            $R_{n,a}(x) = \dfrac{f^{(n+1)}(t)}{(n)!}(x-t)^{n}(x-a)$ for some 
            $t \in (a,x)$.
        \item[ii)]
            $R_{n,a}(x) = \dfrac{f^{(n+1)}(t)}{(n+1)!}(x-a)^{n+1}$ for some 
            $t \in (a,x)$.
        \item[iii)]
            Moreover, if $f^{(n+1)}$ is integrable on $[a,x]$, then
            $$
                R_{n,a}(x) = \int_{0}^{x}\dfrac{f^{(n+1)}(t)}{(n)!}(x-t)^{n} dt.
            $$
    \end{enumerate}
\end{theorem}
%---------%---------%---------%---------%---------%---------%---------%---------
The lagrange problem: \\
A general optimization problem with equality constraints if of the form

\begin{equation}
    \max \ (\min) \quad f(x_1, \ldots, x_n) \quad \text{subject to} \ %
    \left\lbrace 
        \begin{matrix}
            g_1(x_1, \ldots, x_n) = b_1 & \\
            \vdots & (m < n) \\
            g_m(x_1, \ldots, x_n) = b_m & \\
        \end{matrix}
    \right..
\end{equation}
We assume that $m<n$ because otherwise there are usually no degrees of freedom. 
In vector formulation, the problem is
$$
    \max \ (\min) \quad f(x) \quad \text{subject to} \ g_j(x) = b_j 
$$


%[theorem name, {\citep[Thm. #, page]{reference}}]
\begin{theorem}[Lagrange Theorem, {\citep[Thm. 3.3.1, p. 118]{LeonardLong}}]
    Suppose tha the function f and $g_1, \ldots, g_m$ are defined on a set S in 
    $\mathbf{R}^n$, and that $x^* = (x_{1}^{*}, \ldots, x_{n}^{*})$ that solves 
    problem
\end{theorem}
%---------%---------%---------%---------%---------%---------%---------%---------
%---------%---------%---------%---------%---------%---------%---------%---------
\begin{corollary}
    Suppose that for some $t_0 \in [a,b]$, $f(x,t_0) = \lim_{t \to t_0} f(x,t)$
    for each $x \in X$, and that there exists an integrable function g on X 
    such that $|f(t,x)| \leq g(x)$ for all $t\in[a,b]$. Then 
    $$
        \int{f(x,t_0)}d\mu = \lim_{t \to t_0} \int{f(x,t)}d\mu. 
    $$
\end{corollary}

\begin{corollary}
    If the function $t \to f(t,x)$ is continuous on $[a,b]$ for each fixed 
    $x \in X$ and exists $g \in \mathscr{L}$ such that 
    $$
        |f(x,t)| \leq g(x).
    $$
    Then the function
    $$
        F(t) = \displaystyle \int f(x,t)d\mu(x)
    $$
    is continuous on $[a,b]$.
\end{corollary}


%---------%---------%---------%---------%---------%---------%---------%---------
%---------%---------%---------%---------%---------%---------%---------%---------