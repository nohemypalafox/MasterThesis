\section{Preliminaries} \label{Chap1-Sect1} 



%---------%---------%---------%---------%---------%---------%---------%---------
%---------%---------%---------%---------%---------%---------%---------%---------
\begin{theorem}[Interior Maximum Theorem]
    Let $c$ be an interior point of the domain of f, at which $f$ has a relative 
    maximum. If the derivative of $f$ at $c$ exists, then it must be equal to zero.
    
\end{theorem}
%---------%---------%---------%---------%---------%---------%---------%---------
%---------%---------%---------%---------%---------%---------%---------%---------
\begin{theorem}[Rolle Theorem]
    Suppose that $f$ is continuous on a closed interval $J = [a,b]$, that the 
    derivative $f'$ exists in the open interval $(a,b)$, and that 
    $f(a) = f(b) = 0$. Then there exists a point $c \in (a,b)$ such that 
    $f'(c) = 0$.
\end{theorem}
\begin{proof}

\end{proof}
%---------%---------%---------%---------%---------%---------%---------%---------
%---------%---------%---------%---------%---------%---------%---------%---------
\begin{theorem}[(Mean Value Theorem]
	Suppose that $f$ is continuous on a closed interval $J = [a,b]$ and 
	differentiable on the open interval $(a,b)$. Then there exists $c \in (a,b)$ 
	such that 
	$$
	    f(b) - f(a) = f'(c)(b - a).
	$$
\end{theorem}
\begin{proof}

\end{proof}


Lebesgue Dominated convergence theorem
(corolario)
Taylor's Theorem
(corolario)
Fatou's Lemma
Theorem 5.5

Lagrange Theorem

(corolario 1 y 2)

%---------%---------%---------%---------%---------%---------%---------%---------
%---------%---------%---------%---------%---------%---------%---------%---------