\section{Optimal Control} %\label{Chap1-Sect2}
	
%---------%---------%---------%---------%---------%---------%---------%---------
%---------%---------%---------%---------%---------%---------%---------%---------

\begin{definition}
	Let $I \subset \mathbb{R}$ be an interval. We say a finite-valued function
    $u : I \leftarrow \mathbb{R}$ is piecewise continuous if it continuous if it
    is continuous at each $t \in I$, with possible exception of at most a finite 
    number of t, and if u is equal to either its left or right limit at every
    $t \in I$.
\end{definition}
%---------%---------%---------%---------%---------%---------%---------%---------
%---------%---------%---------%---------%---------%---------%---------%---------
\begin{definition}
	Let $x : I \rightarrow \mathbb{R}$ be continuous on I, differentiable at all
    but finitely points of I. Further, suppose that $x'$ is continuous
    wherever it is defined. Then, we say x is piecewise differentiable.
\end{definition}
%---------%---------%---------%---------%---------%---------%---------%---------
%---------%---------%---------%---------%---------%---------%---------%---------
\begin{definition}
	Let $k : I \rightarrow \mathbb{R}$. We say k is continuously differentiable
    if $k' $ exists and is continuous on I.
\end{definition}
%---------%---------%---------%---------%---------%---------%---------%---------
%---------%---------%---------%---------%---------%---------%---------%---------
\begin{definition}
	A function $k(t)$ is said to be concave on $[a,b]$ if 
    $$
    	\alpha k(t_1) + (1-\alpha)k(t_2) \leq k(\alpha t_1 + (1-\alpha)t_2)
    $$
    for all $0 \leq \alpha \leq 1$ and for any $a \leq t_1,t_2 \leq b$.
\end{definition}

A function $k$ is said to be convex on $[a,b]$ if it satisfies the reverse 
inequality , or equivalently, if $-k$ is concave. The second derivative of a
twice differentiable concave function is non-positive; in the case of a convex
function, is non-negative. If $k$ is concave and differentiable, then we have 
a tangent line property
	$$
    	k(t_2) - k(t_1) \geq (t_2 - t_1)k'(t_2)
	$$
for all $a \leq t_1,t_2 \leq b$. 
In the case where $k$ is a function in two variables, we have the analogue to
the tangent line property as follows
	$$
    	k(x_1,y_1) - k(x_2,y_2) \geq (x_1 - x_2)k_x(x_1,y_1) + 
        (y_1 - y_2)k_y(x_1,y_1)
	$$
    for all points $(x_1,y_1),(x_2,y_2)$ in the domain of $k$.

%---------%---------%---------%---------%---------%---------%---------%---------
%---------%---------%---------%---------%---------%---------%---------%---------

\begin{definition}
	A function k is called Lipchitz if there exists a constant c (particular
    to k) such that $|k(t_1) - k(t_2)| \leq c|t_1 - t_2|$ for all points $t_1,
    t_2$ in the domain of k. The constant c is called the Lipchitz constant of 
    k.
\end{definition}

Note that a Lipschitz function is uniformly continuous
%---------%---------%---------%---------%---------%---------%---------%---------
%---------%---------%---------%---------%---------%---------%---------%---------
\begin{theorem}
	If a function $k : I \rightarrow \mathbb{R}$ is piecewise differentiable on
    a bounded interval I, then K is Lipschitz
\end{theorem}
%---------%---------%---------%---------%---------%---------%---------%---------
%---------%---------%---------%---------%---------%---------%---------%---------

existence optimal control theorem

Pontryagins theorems


\begin{theorem}[{\cite[Thm.*]{lenhart2007optimal}}]
	Consider
    \begin{align*}
    	J(u) &= \int_{t_0}^{t_1} f(t,x(t),u(t))dt \\
        \text{subject to} \ x'(t) &= g(t,x(t),u(t)), \ x(t_0) = x_0 
    \end{align*}
    Suppose that $f(t,x(t),u(t))$ and $g(t,x(t),u(t))$ are both continuously 
    differentiable functions in their three arguments and concave in x and u. 
    Suppose $u^{*}$ is a control, with associated state $x^{*}$, and $\lambda$
    a piecewise differentiable function, such that $u^{*}$, $x^{*}$, and 
    $\lambda$together satisfy on $t_0 \leq t \leq t_1$:
    \begin{align*}
    	& f_{u} + \lambda g_{u} = 0, \\
        & \lambda ' = f_{u} + \lambda g_{u}, \\
        & \lambda (t_1) = 0, \\
        & \lambda (t) \geq 0.
    \end{align*}
    Then for all controls $u$, we have
    $$
    	J(u^{*}) \geq J(u)
    $$
\end{theorem}

\begin{theorem}
	Let the set of controls for problem (aqui va una referencia) be Lebesgue
    integrable functions (instead of just piecewise continuous functions) on
    $t_0 \leq t \leq t_1$ with values in $\mathbb{R}$ Suppose that 
    $f(t,x(t),u(t))$ is convex in $u$, and there exist constants $C_4$ and
    $C_1, C_2, C_3 > 0$ and $\beta > 1$ such that
    \begin{enumerate}
    	\item[i.]
        	$g(t,x,u) = \alpha (t,x) + \beta (t,x)u$
        \item[ii.]
        	$|g(t,x,u)| \leq C_1 |1 + |x| + |u||$
        \item[iii.]
        	$|g(t,x_1,u) - g(t,x,u)| \leq C_2 |x_1 - x|(1 + |u|)$
        \item[iv.]
        	$f(t,x,u) \geq C_3 |u|^{\beta} - C_4$
    \end{enumerate}
	for all t with $t_0 \leq t \leq t_1$, x, $x_1$, u in $\mathbb{R}$. Then 
    there exists an optimal control $u^{*}$ maximizing $J(u)$, with $J(u^{*})$
    finite.
\end{theorem}
\todo{Put here comments necessary to establish thm 3.1 [Lenhart's book] }
