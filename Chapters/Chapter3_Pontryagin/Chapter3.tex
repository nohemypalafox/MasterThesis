\chapter{Pontryagin's Maximum Principle}
\lhead{\emph{Variational Principles}}
%---------%---------%---------%---------%---------%---------%---------%---------
%---------%---------%---------%---------%---------%---------%---------%---------
    We use the definition of control system from \cref{Section:ContSys}.
    \begin{equation} \label{Eq;CtrlSys}
        \begin{aligned}
            &\Dot{X}(s) = f(s, X(s), u(s)), & s\in[t,T], \\
            &X(t) = x,
        \end{aligned}
    \end{equation}
    with terminal state constraint:
    $$
        X(T; t,x,u(s)) \in M,
    $$
    where $M \subseteq \mathbb{R}^{n}$ is fixed, and with cost functional
    $$
        J(t,x;u(\cdot)) = \int_{t}^{T} g(s, X(s), u(s))ds + h(X(T)).
    $$
    Recall,
    $$
        \widetilde{\mathcal{U}}_{x}^{M}[t,T] = \left\{ u(\cdot) \in \mathcal{U}[t,T] : %
        X(T; t,x,u(\cdot)) \in M \right\}
    $$
    and
    Problem $(OC)^\top$. For a given $(t,x) \in [0,T) \times \mathbb{R}^{n}, \ $ %
    with $ \ \widetilde{\mathcal{U}}_{x}^{M}[t,T] \neq \varnothing$, find a control % 
    $\Bar{u} \in \widetilde{\mathcal{U}}_{x}^{M}[t,T]$ such that 
    \begin{equation*}
        J(t,x; \Bar{u}(\cdot)) = \inf_{u(\cdot) \in %
        \widetilde{\mathcal{U}}_{x}^{M}[t,T] } %
        J(t,x; u(\cdot)) \equiv V(t,x)    
    \end{equation*}
    
    Also, we consider the assumptions C-1 (\ref{Cond:C1}) and C2(\ref{Cond:C2}).
    Additionally, we introduce a fourth condition
    \begin{asparaenum}
        \item[(\textbf{C}-4)] The map 
        \begin{equation*}
            x \mapsto \big(f(t,x,u), g(t,x,u), h(x) \big),
        \end{equation*}
        is differentiable, the map
        \begin{equation*}
            (x,u) \mapsto \big(f(t,x,u), f_x(t,x,u), g(t,x,u), g_x(t,x,u), h_x(x) \big),
        \end{equation*}
        is continuous and bounded, and
        \begin{align*}
            &\abs{f_{x}(t,x_1,u) - f_{x}(t,x_2,u)} \leq L \abs{x_1 - x_2},
            &\abs{g_{x}(t,x_1,u) - g_{x}(t,x_2,u)} \leq L \abs{x_1 - x_2},
        \end{align*}
        for some constant $L>0$.

    \end{asparaenum}
    In the first section we introduce previous results that would be necessary 
    for the proof of Pontryagin's principle. In the second one we present the proof
    of this principle

\section{Preliminaries}
%---------%---------%---------%---------%---------%---------%---------%---------
%---------%---------%---------%---------%---------%---------%---------%---------

%---------%---------%---------%---------%---------%---------%---------%---------
%               RHO IS A METRIC, U A METRIC SPACE
%---------%---------%---------%---------%---------%---------%---------%---------
\begin{lemma}[\citep{Ekeland_OTVP}, Lemma 7.2]\label{Lemma_rhometric}
    Consider the set of all measurable controls $\mathscr{U}[t,T]$. Define
    $$
        \rho(u(\cdot), v(\cdot)) = \lambda\left(\{s \in [t, T] :  %
            u(s) \neq v(s)\} \right), \hspace{1cm} u(\cdot), v(\cdot) \in \mathscr{U}[t,T].
    $$
    Then, $(\mathscr{U}[t,T], \rho )$ is a metric space.
\end{lemma}
\begin{proof}
    First, we will prove that $\rho$is a metric over $\mathscr{U}[t,T]$.
    \begin{asparaenum}[i)]
        \item Non-negativity. Note that 
            $$
                \rho(u(\cdot), v(\cdot))\geq 0 \, \forall u(\cdot), v(\cdot) \in \mathscr{U}[t,T],
            $$
            by definition of Lebesgue measure.
        \item Identity of Indiscernibles,
            $$
                \rho(u(\cdot), v(\cdot))=0 \iff  u(s) = v(s) \hspace{1cm} a.e. \, %
                    s \in [t,T].
            $$
            Suppose that $\rho(u(\cdot), v(\cdot))=0 $. By definition,
            $$
                \rho(u(\cdot), v(\cdot)) = \lambda\left(\{s \in [t, T] :  %
            u(s) \neq v(s)\}\right) = 0,
            $$
            that is, $u(s) = v(s)$ $a.e.$ $s \in [t,T]$.
            Now, assume $u(s) = v(s)$ $a.e.$ $s \in [t,T]$, by definition 
            $$
                \lambda\left(\{s \in [t, T] :  u(s) \neq v(s)\} \right)=0.
            $$
            That is, $ \rho(u(\cdot), v(\cdot))=0 $.
        \item Simmetry. By definition of $\rho$
            $$
                \rho(u(\cdot), v(\cdot)) = \rho(v(\cdot), u(\cdot)), \hspace{1cm}%
                    \forall u,v \in \mathscr{U}.
            $$
        \item Triangle inequality
            $$
                \rho(u(\cdot), v(\cdot)) \leq \rho(u(\cdot), w(\cdot)) +    %
                    \rho(w(\cdot), v(\cdot)), \hspace{1cm} \forall \,       %
                    u,v,w \in \mathscr{U}.
            $$
            We have
            $$
                \left\{s\in[t,T] : u(s) \neq v(s)\right\} \subseteq %
                \left\{s\in[t,T] : u(s) \neq w(s)\right\} \cup %
                \left\{s\in[t,T] : w(s) \neq v(s)\right\},
            $$
            since
            $$
                \left\{s\in[t,T] : u(s) = w(s)\right\} \cap %
                \left\{s\in[t,T] : w(s) = v(s)\right\} \subseteq %
                \left\{s\in[t,T] : u(s) = v(s)\right\}
            $$
            Then
            \begin{align*}
                \rho(u(\cdot), v(\cdot)) 
                    &\leq \lambda \left( 
                        \left\{s\in[t,T] : u(s) \neq w(s)\right\}%
                        \cup \left\{s\in[t,T] : w(s) \neq v(s)\right\} %
                    \right) \\
                    &\leq \lambda(\left\{s\in[t,T] : u(s) \neq w(s)\right\}) + %
                        \lambda(\left\{s\in[t,T] : w(s) \neq v(s)\right\}) \\
                    &\leq \rho(u(\cdot), w(\cdot)) + \rho(w(\cdot), v(\cdot)).
            \end{align*}
    \end{asparaenum}
    Hence $(\mathscr{U}[t,T], \rho)$ is a metric space.
\end{proof}
%---------%---------%---------%---------%---------%---------%---------%---------
%                        (U, RHO) COMPLETE METRIC SPACE!
%---------%---------%---------%---------%---------%---------%---------%---------
\begin{lemma}[\citep{Ekeland_OTVP}, Lemma 7.2]\label{Lemma_ucompleteMS}
    Consider $(\mathscr{U}[t,T], \rho) $ as in \Cref{Lemma_rhometric}. Then
    $(\mathscr{U}[t,T], \rho) $ is complete.
\end{lemma}
\begin{proof}
    We will use the usual method. That is, we take a Cauchy sequence  $\{u_n\}$
    in $\mathscr{U}[t,T]$, and prove that a subsequence $\{u_{n_k}\}$ converges in
    $\mathscr{U}[t,T]$.
    Let $\{u_n\}_{n=1}^{\infty} \subseteq \mathscr{U}[t,T]$ be a Cauchy sequence.
    Take a subsequence $\{u_{n_k}\}_{k=1}^{\infty}$ such that 
    $$
        \rho(u_{n_k}(\cdot), u_{n_{k+1}}(\cdot)) < \dfrac{1}{2^{k+1}}(T-t)
    $$
    Now, we prove that $\rho(u_{n_k})$ converges in $\mathscr{U}[t,T]$ . Define the
    set
    $$
        A_{k} := \bigcup_{p \geq k}\left\{ s\in[t,T] : u_{n_p}(s) \neq u_{n_{p+1}}(s)%
            \right\} \hspace{1cm} \forall k \in \mathbb{N}.
    $$
    Note that $A_{k+1} \subseteq A_k$, for each $k \in \mathbb{N}$. Further, for
    $k \in \mathbb{N}$
    \begin{align*}
        \lambda(A_k) 
            &= \lambda \left(\bigcup_{p \geq k} \{s\in[t,T] : u_{n_p}(s) %
                \neq u_{n_{p+1}}(s)\}\right) \\
            &\leq \sum_{p \geq k} \lambda(\{s\in[t,T] : u_{n_p}(s) \neq %
                u_{n_{p+1}}(s)\}) \\
            &= \sum_{p \geq k} \rho(u_{n_p}(\cdot), u_{n_{p+1}}(\cdot)) \\
            &\leq \sum_{p \geq k}\dfrac{T-t}{2^{p}} \\
            &= \dfrac{T-t}{2^{k}}. 
    \end{align*}
    Thus, $\stcomp{A_1}$
    $$
        \Bar{u}(s) := u_{n_{1}}(s) = u_{n_{l}}(s), \hspace{1cm} l \geq 1, \, s \in \stcomp{A_1}.
    $$
    Then, for each $s \in A_1$
    $$
        \lim_{l \to \infty} u_{n_l}(s) = u_{n_1}(s) = \Bar{u}(s).
    $$
    Similarly for $k$, $\Bar{u}(s) = u_{n_k} = u_{n_l}$, $l \geq k$, $s \in \stcomp{A_{k}}$.
    Hence,
    \begin{equation*}
        \Bar{u}(s) = \left\{
        \begin{matrix} 
            u_{n_1}(s), & s\in \stcomp{A_1} \\
            u_{n_2}(s), & s\in \stcomp{A_2} \\
            \vdots      & \vdots \\
            u_{n_k}(s), & s \in \stcomp{A_k} \\
            \vdots      & \vdots 
        \end{matrix}
        \right. 
    \end{equation*}
    In this way we obtain a function $\Bar{u}$ which is measurable since
    $$
        \Bar{u}(s) = \sum_{k=1}^{\infty}u_{n_k}(s) \ind_{\stcomp{A_k}}(s)
    $$
\end{proof}

%---------%---------%---------%---------%---------%---------%---------%---------
%                    CONTINUITY OF THE FUNCTION J 
%---------%---------%---------%---------%---------%---------%---------%---------

\begin{asparaenum}
    \item 
    Show the continuity of the mapping 
    $$
        u\mapsto \int_{t}^{\tau} g(s, X(s), u(s))ds, \ \ \ u\in U[t,\tau]
    $$
\end{asparaenum}

%---------%---------%---------%---------%---------%---------%---------%---------
\begin{lemma}\label{Lemma0}
    If $\{ u_{n}\}$ is Cauchy sequence in $(\mathscr{U},\rho)$, then $\{ X_{n}\}$ 
    is a Cauchy sequence in $(C[t, \tau ], ||\cdot ||)$, where $(u_{n}, X_{n})$
    satisfies $\cref{Eq;CtrlSys}$.
\end{lemma}
\begin{proof}
    From $\cref{Eq;CtrlSys}$ we know that
    $$
        X_{n}(s)= x+\int_{t}^{s} f(r, u_{n}(r), X_{n}(r)) dr,
    $$
    for $s\in [t, \tau]$ and $n\in\mathbb{N}$. For $n$ and $k$ in $\mathbb{N}$,
    $$
        X_{n}(s)- X_{k}(s)= \int_{t}^{s}[f(r, u_{n}(r), X_{n}(r))-f(r, u_{k}(r),%
        X_{k}(r))] dr,
    $$
    and
    \begin{align*}
        |X_{n}(s)- X_{k}(s)| &\leq \int_{A_{n,k}}|f(r, u_{n}(r), X_{n}(r))- %
            f(r, u_{k}(r), X_{k}(r))|dr \\
        & +\int_{A_{n,k}^{c}}|f(r, u_{n}(r), X_{n}(r))-f(r, u_{k}(r), X_{k}(r))|dr.
    \end{align*}
    where 
    $$
        A_{n,k}:= \{ r\in[t,s] \ : \ u_{n}(r)\neq u_{k}(r) \}.
    $$
    Further,
    \begin{align*}
        |X_{n}(s)- X_{k}(s)| &\leq \int_{A_{n,k}}|f(r, u_{n}(r), X_{n}(r))|dr +
        \int_{A_{n,k}}|f(r, u_{k}(r), X_{k}(r))|dr \\
        &\phantom{=} +\int_{A_{n,k}^{c}}|f(r, u_{n}(r), X_{n}(r))|dr +
        \int_{A_{n,k}^{c}}|f(r, u_{k}(r), X_{k}(r))|dr \\
        &\leq \int_{A_{n,k}} L(1+|X_{n}(r)|)dr + \int_{A_{n,k}}L(1+|X_{k}(r)|)dr \\
        &\phantom{=}+ \int_{A_{n,k}} L(X_{n}(r)-X_{k}(r))dr \\
        &\leq 2Le^{L(T+t)}(1+|x|)\lambda(A_{n,k})+\int_{t}^{s} %
        L\abs{X_{n}(r)-X_{k}(r)}dr.
    \end{align*}
    That is
    $$
        \abs{X_{n}(s)-X_{k}(s)}\leq 2Le^{L(T+t)}(1+|x|)\rho(u_{n}, %
            u_{k})+\int_{t}^{s}L\abs{X_{n}(r)-X_{k}(r)}dr,
    $$
    and by Gronwall's inequality \eqref{prop:GI},
    $$
        \abs{X_{n}(s)-X_{k}(s)}\leq 2Le^{L(T+t)}(1+|x|)\rho(u_{n}, u_{k})e^{L(s+t)}.
    $$
    Put $k=2Le^{2L(T+t)}(1+|x|)$. Then
    $$
        \abs{X_{n}(s)-X_{k}(s)}\leq k\rho(u_{n}, u_{k}), \ \ \ s\in [t,T].
    $$
    and so
    $$
        \norm{X_{n}(\cdot)-X_{k}(\cdot)}\leq k\rho(u_{n}, u_{k}),
    $$
    Therefore, $\{X_{n}\}$ is a Cauchy sequence, whenever $\{u_{n}\}$ is a Cauchy sequence.
\end{proof}
\begin{proposition}\label{Proposition1}
    The mapping $u \mapsto X_{n}$ from $(\mathscr{U}, \rho)$ into $(C[t,T], \norm{\cdot})$ is continuous.
\end{proposition}
\begin{proof}
    Let $\{u_{n}\}$ converge to $u$ in $(\mathscr{U},\rho)$. Denote by $X_{n}$ the unique solution \cref{prop2.1.1} toC
    \begin{equation}\label{dosestrellas}
        X_{n}(s)=x\int_{t}^{s} f(x, u_{n}(r), X_{n}(r))dr, \ \ \ s\in [t,T], n=0,1,2,\ldots
    \end{equation}
    By Lemma \ref{Lemma0}, $\{ X_{n}\}$ is a Cauchy sequence and $(C[t,T], \norm{\cdot})$ %
    is a Banach space, then exists $\Bar{X}\in C[t, T]$ such that 
    $$
        \lim_{n\to\infty}\norm{X_{n}(\cdot)-\Bar{X}}=0.
    $$
    We don't know whether $X_{0}$ equals $\Bar{X}$. To conclude the proof we will show 
    that indeed $X_{n}\to X_{0}$ in norm. Since $X_{n}\to\bar{X}$ in norm, then 
    $X_{n}(r)$ converges to $\bar{X}(r)$ for every $r\in[t, T]$. On the other hand, 
    $u_{n}\to u_{0}$ under $\rho$, then $u_{n}\to u_{0}$ a.e. Then, this facts on 
    the continuity of $f$ imply
    $$
        f(r, u_{n}(r), X_{n}(r))\to f(r, u_{0}(r), \Bar{X}(r)), \, \ \ a.e.
    $$
    Further,
    $$
        \abs{\int_{t}^{s} f(r, u_{n}(r), X_{n}(r))dr}\leq K,
    $$
    for some $K$ in $\mathbb{R}$. By Lesbesgue's dominated convergence theorem,
    $$
        \lim_{n\to\infty}\int_{t}^{s}f(r, u_{n}(r), X_{n}(r))dr%
        = \int_{t}^{s}f(r, u_{0}(r), \Bar{X}(r))dr.
    $$
    Thus, from \eqref{dosestrellas}
    $$
        \Bar{X}(s):= x+ \int_{t}^{s}f(r, u_{0}(r),\Bar{X}(r))dr
    $$
    and the uniqueness result (\cref{prop2.1.1}) implies 
    $$
        \Bar{X}(s)=X_{0}(s), \ \ \ s\in [t, T].
    $$
\end{proof}
\begin{theorem}\label{Thm:Jcontinuity}
    The mapping 
    $$
        u \mapsto \int_{t}^{T}g(r, u(r), X_0(r))dr
    $$
    is continuous.
\end{theorem}
\begin{proof}
    Let $\{u_{n}\}$ converge to $u_{0}$ in $(\mathscr{U},\rho)$. We use the 
    same notation $X_{n}$ given in \Cref{dosestrellas}. By \Cref{Proposition1} 
    and the continuity of $g(r, \cdot, \cdot)$ we have
    $$
        \lim_{n\to\infty}g(r, u_{n}(r), X_{n}(r)) = g(r, u_{0}(r), X_{0}(r)) \ \ \ a.e.
    $$
    Then by Lebesgue's theorem,
    $$
        \lim_{n\to\infty}\int_{t}^{T}g(r, u_{n}(r), X_{n}(r))dr=\int_{t}^{T}%
        g(r, u_{0}(r), X_{0}(r))dr.
    $$
\end{proof}

%---------%---------%---------%---------%---------%---------%---------%---------
%---------%---------%---------%---------%---------%---------%---------%---------

\begin{proposition}\label{prop6.8}
    Given $f \in L^{p}$, $1 < p \leq \infty$ and $\veps > 0$, there is a step
    function $\varphi$ and a continuous function $\psi$ such that 
    $\norm{f - \varphi}_{p} < \veps$ and $\norm{f - \psi}_{p} < \veps$.
\end{proposition}

%---------%---------%---------%---------%---------%---------%---------%---------
%                           Spike Variation Lemma
%---------%---------%---------%---------%---------%---------%---------%---------
\begin{lemma}[ Lemma 1.4.6, p. 30, \citep{YongDG_ACIntro}]\label{lem:spike}
    Suppose $f(\cdot) \in L^{1}([0,T] ; \mathbb{R}^{n})$ and for $\delta > 0$, let
    $$
        \mathbb{E}_{\delta} = \left \{ E \subseteq [0,T] \, : \, %
            \lambda(E) = \delta T  \right\},
    $$
    where $\lambda$ is the Lebesgue measure. Define $g:[0,T] \to \mathbb{R}^{n}$ as
    $$
        g(t) = \int_{0}^{t}\left(1 - \dfrac{1}{\delta} \ind_{E}\right)%
            f(s) ds.
    $$
    Then
    $$
        \inf_{E \in \mathbb{E}_{\delta} } \norm{g(t)}_{C([0,T] ; %
            \mathbb{R}^{n})} = 0.
    $$
\end{lemma}
\begin{proof}
    By Proposition \cref{prop6.8}, for any $f(\cdot) \in L^{1}([0,T]; \mathbb{R}^n)$
    and any $\veps >0$ there exist an $f_{\veps} \in C([0,T]; \mathbb{R}^n)$
    such that
    \begin{equation}
        \int_{0}^{T} \abs{f(r) - f_{\veps}(r)}dr < \veps.
    \end{equation}
    We can find a partition $0 = t_0<t_1 < \cdots < t_{k-1} < t_k = T$ of $[0,T]$
    such that
    \begin{equation}
        \norm{f_{\veps}(\cdot)}_{C([0,T]; \mathbb{R}^n)} \max_{1 \leq i \leq k}%
            (t_i -t_{i-1}) < \dfrac{\veps}{k}.
    \end{equation}
    Define the step function $\Bar{f}_{\veps}(\cdot)$ as
    \begin{equation}\label{SpikeVar;eqfeps}
        \Bar{f}_{\veps}(r) = \sum_{i=1}^{k}f_{\veps}(t_i)\ind_{(t_{i-1},t_i]}(r), \, r \in [0,T].
    \end{equation}
    First, we prove that 
    \begin{equation}
        \int_{0}^{T}\abs{f_{\veps}(r) - \Bar{f}_{\veps}(r)}dr < \veps.
    \end{equation}
    Using \cref{SpikeVar;eqfeps} we get
    \begin{align*}
        \int_{0}^{T}\abs{f_{\veps}(r) - \Bar{f}_{\veps}(r)}dr %
        &=
            \int_{0}^{T}\abs{f_{\veps}(r) - \sum_{i=1}^{k}f_{\veps}(t_i)\ind_{(t_{i-1},t_i]}(r) }dr 
        \\
        &=
            \int_{0}^{t_1}\abs{f_{\veps}(r) 
            - \sum_{i=1}^{k}f_{\veps}(t_i)%
            \ind_{(t_0,t_1]}(r) }dr + \cdots 
        \\
        & + \int_{t_{k-1}}^{T}\abs{f_{\veps}(r) - %
            \sum_{i=1}^{k}f_{\veps}(t_i)\ind_{(t_{i-1},t_i]}(r) }dr 
        \\
        &\leq   
            \int_{0}^{t_1}\abs{f_{\veps}(r) - \ind_{(t_0,t_1]}(r) f_{\veps}(t_1)} dr  + \cdots 
        \\
        & + \int_{t_{k-1}}^{T}\abs{f_{\veps}(r) - \ind_{(t_{k-1},T]}(r) f_{\veps}(T)}dr \\
        &=
            \int_{0}^{t_1}\abs{f_{\veps}(r) - f_{\veps}(t_1)} dr + \ldots + %
            \int_{t_{k-1}}^{T}\abs{f_{\veps}(r) - f_{\veps}(T)}dr \\
        &\leq
            \int_{0}^{t_1}\norm{f_{\veps}(\cdot)}dr + \ldots + %
            \int_{t_{k-1}}^{T}\norm{f_{\veps}(\cdot)}dr \\
        &\leq 
            k\int_{t_{i-1}}^{t_i}\norm{f_{\veps}(\cdot)} dr.
    \end{align*}
    Then,
    \begin{equation*}
        \int_{0}^{T}\abs{f_{\veps}(r) - \Bar{f}_{\veps}(r)}dr \leq %
        k \norm{f_{\veps}(\cdot)} \max_{1\leq i \leq k}(t_{i-1},t_i) < \veps.
    \end{equation*}
    Now, let
    \begin{equation}
        E_{\delta} = \bigcup_{i=1}^{k}[t_{i-1}, t_{i-1} + \delta(t_i - t_{i-1})].
    \end{equation}
    Note that $\lambda(E_{\delta}) = \sum_{i\geq k}\delta(t_i - t_{i-1}) = \delta T$.
    For any $s \in [0,T]$, there is an index $j$ such that $t_{j-1} < s \leq t_j$. 
    We prove that the integral 
    \begin{equation}\label{SpikVar;eqIi}
        I_{i} := \int_{t_{i-1}}^{t_i}\left(1 - \dfrac{1}{\delta} \ind_{E_{\delta}}(r)\right)%
        \Bar{f}_{\veps}(r)dr = 0,
    \end{equation}
    for all $(t_{i-1},t_i]$ which not enclose $s$. Substituting \cref{SpikeVar;eqfeps}
    in \cref{SpikVar;eqIi} yields
    \begin{align*}
        I_i &= \int_{t_{i-1}}^{t_i}\left(1 - \frac{1}{\delta} \ind_{E_{\delta}}(r)\right)%
                \sum_{i=1}^{k}f_{\veps}(t_i)\ind_{(t_{i-1},t_i]}(r) dr \\
            &= \int_{t_{i-1}}^{t_i}\left(1 - \frac{1}{\delta} \ind_{E_{\delta}}(r)\right)%
                f_{\veps}(t_i) dr \\
            &= f_{\veps}(t_i) \int_{t_{i-1}}^{t_i}\left(1 - \frac{1}{\delta} %
                \ind_{[t_{i-1}, t_{i-1} + \delta (t_i - t_{i-1})]}(r)\right) \\
            &= f_{\veps}(t_i)\left[(t_i - t_{i-1}) - \frac{1}{\delta}%
                \delta (t_i - t_{i-1})\right] \\
            &= 0.
    \end{align*}
    Now we consider the interval $(t_{j-1}, s]$ and estimate
    \begin{equation*}
        I_{j} := \abs{\int_{t_{i-1}}^{s} \left(1 - \frac{1}{\delta}%
                 \ind_{E_{\delta}}(r)\right) \Bar{f}_{\veps}(r) dr}.
    \end{equation*}
    By definition of $\Bar{f}_{\veps}$ we have
    \begin{align*}
        I_{j} &= \abs{\int_{t_{i-1}}^{s} \left(1 - \frac{1}{\delta}%
                \ind_{E_{\delta}}(r)\right) f_{\veps}(r) dr}\\
            &= \abs{f_{\veps}(t_j)} \abs{\int_{t_{i-1}}^{s} \left(1 - %
                \frac{1}{\delta}\ind_{[t_{j-1}, t_{j-1}+ \delta(t_j - t_{j-1})]}\right)dr} \\
            &= \abs{f_{\veps}(t_j)} \abs{s-t_{j-1} - \dfrac{1}{\delta} \left\{%
                (s - t_{j-1}) \wedge [\delta(t_j - t_{j-1})]\right\} } \\
            &\leq  \abs{f_{\veps}(t_j)}(t_j - t_{j-1}) \\
            &< \norm{f_{\veps}(\cdot)} \max_{1\leq i \leq k}(t_i - t_{i-1}) \\
            &< \dfrac{\veps}{k} < \veps.
    \end{align*}
    Now,
    \begin{align*}
        \abs{\int_{0}^{s} \left(1 - \frac{1}{\delta} \ind_{E_{\delta}}(r) \right)f(r)dr} &= %
            \abs{\int_{0}^{s} \left(1 - \frac{1}{\delta} \ind_{E_{\delta}}(r) \right)(f(r) - f_{\veps}(r))dr} \\ %
            &+ \abs{\int_{0}^{s} \left(1 - \frac{1}{\delta} \ind_{E_{\delta}}(r) \right)(f_{\veps}(r) - \Bar{f}_{\veps}(r))dr} \\%
            &+ \abs{\int_{0}^{s} \left(1 - \frac{1}{\delta} \ind_{E_{\delta}}(r) \right) \Bar{f}_{\veps}(r)dr} %
    \end{align*}
    Note that,
    \begin{align*}
        \abs{\int_{0}^{s} \left(1 - \frac{1}{\delta} \ind_{E_{\delta}}(r) \right)%
        (f(r) - f_{\veps}(r))dr} %
        &= \int_{0}^{s} \abs{1 - \frac{1}{\delta} \ind_{E_{\delta}}(r)}\abs{f(r) - f_{\veps}(r)}dr \\
        &\leq \dfrac{1 +\delta}{\delta} \int_{0}^{s}\abs{f(r) - f_{\veps}(r)}dr \\
        &< \dfrac{(1+\delta)\veps}{\delta}.
    \end{align*}
    Similar ideas applies to show that 
    \begin{equation*}
        \abs{\int_{0}^{s} \left(1 - \frac{1}{\delta} \ind_{E_{\delta}}(r) \right)%
            (f_{\veps}(r) - \Bar{f}_{\veps}(r))dr}  < \dfrac{(1+\delta)\veps}{\delta}.
    \end{equation*}
    Thus
    \begin{equation*}
        \abs{\int_{0}^{s} \left(1 - \frac{1}{\delta} \ind_{E_{\delta}}(r) %
        \right)f(r)dr} < \dfrac{2(1+\delta)\veps}{\delta} + \veps.
    \end{equation*}
    Since $\veps$ is arbitrary we obtain the conclusion.
\end{proof}

%---------%---------%---------%---------%---------%---------%---------%---------
%---------%---------%---------%---------%---------%---------%---------%---------
    Since $0<\delta <1$ is fixed in spike variation \cref{lem:spike}, by the
    definition of infimum there is $E_\delta \in \mathbb{E}_{\delta}$ such that
    \begin{equation*}
        \sup_{\tau \in [t,T]} \abs{\int_{t}^{\tau} f(s)ds - %
        \frac{1}{\delta}\int_{t}^{\tau} \ind_{E_\delta}(s)f(s)ds} < \delta,
    \end{equation*}
    that is 
    $$
        \abs{\delta \int_{t}^{\tau}f(s)ds - \frac{1}{\delta}\int_{t}^{\tau} %
        \ind_{E_\delta}(s)f(s)ds} < \delta, \, \, \forall \tau \in [t,T].
    $$
    Then
    $$
        \abs{\delta \int_{t}^{\tau} f(s)ds - \int_{t}^{\tau} \ind_{E_\delta}(s)%
        f(s)ds} < \delta^2, \, \, \forall \tau \in [t,T].
    $$
    Define 
    $$
        r_{\delta}(\tau) := \delta \int_{t}^{\tau}f(s)ds - %
        \int_{t}^{\tau}\ind_{E_\delta}(s)f(s)ds, \, \, \tau \in [t,T].
    $$
    So we obtain the next corollary.
%---------%---------%---------%---------%---------%---------%---------%---------
%---------%---------%---------%---------%---------%---------%---------%---------
    \begin{corollary}\label{cor:spike}
        Suppose $f(\cdot) \in L^{1}([t,T] ; \mathbb{R}^{n})$ and for
        $0 < \delta < 1$, let
        $$
            \mathbb{E}_{\delta} = \left \{ E \subseteq [0,T] \, : \, %
                \lambda(E) = \delta (T-t)  \right\},
        $$
        where $\lambda$ is the Lebesgue measure. Then, there exists 
        $E_\delta \in \mathbb{E}_{\delta}$ and a function $
        r_\delta \in L^{1}([t,T] ; \mathbb{R}^{n}) $ such that 
        \begin{equation*}
            \delta\int_{t}^{\tau}f(s)ds = \int_{t}^{\tau} \ind_{E_\delta}(s)%
            f(s)ds + r_\delta(\tau), \, \, \tau \in [t,T],
        \end{equation*}
        and $\abs{r_{\delta}(\tau)} < {\delta}^2$ for all $\tau \in [t,T]$.
    \end{corollary}
%---------%---------%---------%---------%---------%---------%---------%---------
%---------%---------%---------%---------%---------%---------%---------%---------   
    \begin{theorem}[Danskin's theorem, p.20, \citep{guler2010foundations}]\label{thm:danskin}
        Let $X \subseteq \mathbb{R}^n$ open and $Y$ a compact set. Suppose that 
        $f : X \times Y \to \mathbb{R}$ is continuous and $\nabla_{x} f(x,y)$ 
        exists and is continuous. Define
        $$
            \varphi(x) := \min_{y \in Y} \{f(x,y)\}.
        $$
        Then $\varphi$ is continuous and the directional derivative of $\phi$ 
        exists and is given by
        $$
            D_{v}^{+}\varphi(x) = \min_{y\in Y(x)}\{\langle \nabla_{x} f(x,y), v \rangle\},
        $$
        where $Y(x) = \{y\in Y \, : \, \varphi(x) = f(x,y)\}$ is the set of 
        minimizers.
        If the set of minimizers has only one element, that is, $Y(x) = \{y_0\}$
        then
        $$ 
            D_{v}^{+}\varphi(x) = \{\langle \nabla_{x} f(x,y_0), v \rangle\},
        $$
    \end{theorem}
    
%---------%---------%---------%---------%---------%---------%---------%---------
%---------%---------%---------%---------%---------%---------%---------%---------    
\begin{proposition}[\citep{YongDG_ACIntro}, Proposition 1.4.8, p. 32]
    Let $M \subseteq \mathbb{R}^{n}$ be a non-empty closed convex set. Then there 
    exists a map $P_{M} : \mathbb{R}^{n} \to M$ such that
    \begin{asparaenum}[i)]
        \item
            $ \abs{x - P_{M}(x)} = \inf_{y \in M}\abs{x-y} \equiv d(x,M), $
        \item 
            $\abs{P_{M}(x_1) - P_M(x_2)} \leq \abs{x_1 - x_2},$ for all 
            $x_1, x_2$ in $ \mathbb{R^{n}}$.
    \end{asparaenum}
    Moreover, for $z \in M$, 
    \begin{asparaenum}[iii)]
        \item $z = P_{M}(x)$ if and only if $\langle x-z, y-z \rangle \leq 0$.
    \end{asparaenum}
\end{proposition}
%---------%---------%---------%---------%---------%---------%---------%---------
%---------%---------%---------%---------%---------%---------%---------%---------
\begin{proof}
    Let $\{z_k\} \subseteq M$ such that $\lim_{k \to \infty}\abs{x - z_k} = d(x,M)$
    for any $x \in \mathbb{R}^{n}$. Since $z_k \in M$ for all $k \in \mathbb{N}$,
    $\{z_k\}$ is bounded. We may assume that $z_k \to \bar{z} \in M$, if not we 
    can extract a subsequence. Then
    $$
        \lim_{k\to\infty}\abs{x - z_k} = \abs{x - \bar{z}} = d(x,M).
    $$
    Now, let $\bar{y} \in M$ such that $\abs{x - \bar{y}} = d(x,M)$. By the 
    convexity of M, $\dfrac{\bar{y} + \bar{z}}{2} \in M$, which implies 
    \begin{align*}
        (d(x,M))^2 &\leq \abs{x - \dfrac{\bar{y} + \bar{z}}{2}}^2 \\
            &= \frac{1}{4}\abs{2x - \bar{y} - \bar{z}}^2 \\
            &= \frac{1}{4}\abs{x - \bar{y} + x - \bar{z}}^2 \\
            &\leq \frac{1}{4}(\abs{(x - \bar{y}) + (x - \bar{z})}^2 + %
            \abs{(x - \bar{y}) - (x - \bar{z})}^2 - \abs{\bar{y} - \bar{z}}^2) \\
            &\leq \frac{1}{4}(2\abs{x - \bar{y}}^2 + 2\abs{x - \bar{z}}^2 - %
                \abs{\bar{y} - \bar{z}}^2) \\
            &= (d(x,M))^2 - \dfrac{1}{4}\abs{\bar{y} - \bar{z}}^2),
    \end{align*}
    thus, $\bar{y} = \bar{z}$. Consequently, $P_M$ is a well-defined map.
%---------%---------%---------%---------%---------%---------%---------%---------

%---------%---------%---------%---------%---------%---------%---------%---------    
    Suppose that $P_M(x) \in M$. Then for any $y \in M$ and $\alpha \in (0,1)$,
    we have
    $$
        P_M(x) + \alpha(y - P_M(x)) = (1-\alpha)P_M(x) + \alpha y \, \in M.
    $$
    Then,
    $$
        \abs{P_M(x) - x}^2 \leq \abs{P_M(x) + \alpha (y-P_M(x)) - x}^2,
    $$
    which implies
    \begin{align*}
        0 &\leq \abs{P_M(x) - x} + \abs{\alpha(y - P_M(x))}^2 - \abs{P_M(x) - x}^2 \\
        &= 2\alpha \langle P_M(x) - x, y - P_M(x)\rangle + \alpha^2\abs{y-P_M(x)}^2 \\
        &= -2\alpha \langle P_M(x) - x, y - P_M(x)\rangle + \alpha^2\abs{y-P_M(x)}^2.
    \end{align*}
    Dividing by $\alpha$ and multiplying by $-1$ we have
    $$
        \langle P_M(x) - x, y - P_M(x)\rangle - \alpha \abs{y-P_M(x)}^2 \leq 0.
    $$
    Letting $\alpha \to 0$ we get
    \begin{equation}\label{prdless0}
        \langle x- P_M(x), y-P_M(x)\rangle \leq 0, \, \, \forall y \in M.
    \end{equation}
    Now, suppose that, for $z \in M$ for $z\in M$
    $$
        \langle x-z, y-z \rangle \leq 0, \, \, \forall y \in M.
    $$
    Note
    $
        \abs{y-x}^2 = \abs{z-x}^2 + \abs{y-z}^2 + 2\langle y-z, z-x \rangle,
    $
    then
    $$
         \abs{y-x}^2 - \abs{z-x}^2 = \abs{y-z}^2 + 2\langle y-z, z-x \rangle,
    $$
    for all $y \in M$. Thus $\abs{y-x} \geq \abs{z-x}$ for all $y \in M$. By
    definition of infimum $z = P_M(x)$.
    
    From \cref{prdless0}, for any $x_1, x_2 \in  \mathbf{R}^n$, we have 
    $$
        \langle P_M(x_1) - P_M(x_2), x_2 - P_M(x_2)\rangle \leq 0,
    $$
    and
    $$
        \langle P_M(x_2) - P_M(x_1), x_1 - P_M(x_1)\rangle = %
        \langle P_M(x_1) - P_M(x_2), P_M(x_2) - x_1\rangle \leq 0.
    $$
    Adding the both inequalities we obtain
    $$
        \langle P_M(x_1) - P_M(x_2), x_2 - P_M(x_2) -x_1 + P_M(x_1)\rangle \leq 0
    $$
    $$
        \langle P_M(x_1) - P_M(x_2), P_M(x_1) - P_M(x_2) - (x_1 -x_2)\rangle \leq 0
    $$
\end{proof}


%---------%---------%---------%---------%---------%---------%---------%---------
%---------%---------%---------%---------%---------%---------%---------%---------

%---------%---------%---------%---------%---------%---------%---------%---------
%---------%---------%---------%---------%---------%---------%---------%---------


Let us assume the following,
\begin{theorem}[Taylor's Theorem, p. 359 \citep{marsden1993elementary}]\label{thm:TT}
    Let $f : A \to \mathbb{R}$ be of class $C^{r}$ for $A \subseteq \mathbb{R}^{n}$
    an open set. Let $x,y \in A$, ans suppose that the segment joining $x$ and $y$
    lies in $A$. Then there is a point $c$ on that segment such that
    \begin{equation*}
        f(y) - f(x) = \sum_{k=1}^{r-1}\dfrac{1}{k!}\mathbf{D}^{k}f(x)(y-x,\ldots, y-x)%
        + \dfrac{1}{r!}\mathbf{D}^{r}f(c)(y-x,\ldots, y-x),
    \end{equation*}
    where $\mathbf{D}^{k}f(x)(y-x,\ldots, y-x)$ denotes $\mathbf{D}^{k}f(x)$ as a
    $k$-linear map applied to the $k$-tuple $(y-x,\ldots, y-x)$. In coordinates,
    \begin{equation*}
        \mathbf{D}^{k}f(x)(y-x,\ldots, y-x) = \sum_{i_1, \ldots, i_k  =1}%
        \left(\dfrac{\partial^{k}f}{\partial x_{i_1} \cdots \partial x_{i_k}}\right)%
        \left(y_{i_1} - x_{i_1}\right) \cdots (y_{i_k} - x_{i_k}).
    \end{equation*}
    Setting $y = x + h$, we can write the Taylor formula as
    \begin{equation*}
        f(x+h) = f(x) + \mathbf{D}f(x)\cdot h + \cdots  + \dfrac{1}{(r-1)!}%
        \mathbf{D}^{r-1}f(x)\cdot (h, \ldots, h) + R_{r-1}(x,h),
    \end{equation*}
    where $R_{r-1}(x,y)$ is the remainder. Furthermore,
    \begin{equation*}
        \dfrac{R_{r-1}(x,h)}{\norm{h}^{r-1}} \to 0 \text{ as } h \to 0. 
    \end{equation*}
\end{theorem}

%---------%---------%---------%---------%---------%---------%---------%---------
%---------%---------%---------%---------%---------%---------%---------%---------

\section{Proof of Pontraying's Maximum Principle} 

According to the (OC)$^{T}$ problem (*), we consider the adjoint equation 
\begin{equation}\label{psi_der}
    \Dot{\psi}(s) = - f_{x}(s, \Bar{X}(s), \Bar{u}(s))^{\top}\psi(s) + %
        \psi^{0}g_{x}(s, \Bar{X}(s), \Bar{u}(s))^{\top}, s \in [t,T],
\end{equation}
with
\begin{align*}
    &\psi^{0} \leq 0, \\
    &\abs{\psi^{0}}^{2} + \abs{\psi(T) - \psi^{0}h_{x}(\Bar{X}(T))^{\top}}^{2}=1.
\end{align*}
We also define the Hamiltonian function regarding to the Problem (OC)$^{T}$ as
\begin{equation}
    H(s, x, u, \psi^{0}, \psi) := \psi^{0}g(s,x,u) + \langle \psi^{0}, f(s,x,u) \rangle,
\end{equation}
with $(s, x, u, \psi^{0}, \psi) \in [t,T]\times \mathbb{R}^{n}\times U \times %
        \mathbb{R}\times \mathbb{R}^{n}$.

\begin{theorem}[Pontryagin maximum principle]
    Let (C4) hold. Let $M$ be a non-empety closed convex set. Suppose 
    $(\Bar{X}(\cdot), \Bar{u}(\cdot))$ is an optimal pair of problem $(OC)^\top$ 
    for the initial pair $(t,x)$ and $\psi$ is the solution of \cref{psi_der}. 
    Then the following conditions holds:
    \begin{asparaenum}[\bf{(}\bf{P}-1\bf{)}]
        \item Maximum condition:
        \begin{equation*}
            H(s, \Bar{X}(s), \Bar{u}, \psi^{0}, \psi(s)) = %
            \max_{u \in U} H(s, \Bar{X}(s), \psi^{0}, \psi(s)), \hspace{1cm}%
                a.e. s \in [t,T],
        \end{equation*}
        \item Transversality condition:
        \begin{equation*}
            \langle \psi(T) - \psi^{0}h_x (\Bar{X}(T))^\top, y -\Bar{X}(T)\rangle %
            \geq 0, \hspace{1cm} \forall y \in M.
        \end{equation*}
    \end{asparaenum}
\end{theorem}
\begin{remark}
    Note that in the case $M = \mathbb{R}^{n}$ the transversality condition becomes
    $$
        \psi(T) = \psi^{0}h_x (\Bar{X}(T))^\top.
    $$
\end{remark}
%---------%---------%---------%---------%---------%---------%---------%---------
%---------%---------%---------%---------%---------%---------%---------%---------
\begin{proof}
    We prove this theorem, following the next steps.
    \begin{inparaenum} [\textbf{Step}-1]
        \item
            Introduce an auxiliary cost functional $J_{\veps}^{T}$.
        \item
            Apply the Corollary \eqref{EVP_Corollary} of the E.V.P to 
            the functional $J_{\veps}^{T}$, in order to obtain an 
            $\veps$-optimal pair $(x, u^{\veps})$.
        \item
            From the \cref{cor:spike} of the spike variation lemma we obtain the necessary 
            conditions for the $\veps$-optimal pair.
        \item
            Take the limit $\veps \to 0$ to obtain the necessary 
            conditions for the original problem.
    \end{inparaenum}
    
    \begin{asparaenum}[\textbf{Step}-1]
        \item
        First, without loss of generality we may assume that 
        $$
            J(\bar{u}) = J(t,x; \Bar{u}(\cdot)) = \int_{t}^{T} g(s, \Bar{X}(s), \Bar{u}(s)) ds %
                + h(\Bar{X}(t)) = 0,
        $$
        otherwise, we can consider the functional 
        $
             J(u) -  J(\Bar{u}).
        $
        Let $\veps >0$ and define the functional
        \begin{equation}
            J_{\veps}(u) = %
                \left[
                    \left( J(u) + \veps \right)^{2}
                    + 
                    d_{M}^2\left(X_{u}(T)\right) 
                \right]^{1/2} \geq 0,
        \end{equation}
        where+
        \begin{equation*}
            d_{M}(x) = \min_{y\in M}(\abs{x-y}) = \abs{x - P_{M}(x)},
        \end{equation*}
        and
        \begin{equation*}
            d_{M}^2(x) = \min_{y\in M}(\abs{x-y}^2) = \abs{x - P_{M}(x)}^2,
        \end{equation*}
        for all $x \in \mathbb{R}^{n}$. Note that
        \begin{equation*}
             d_{M}^2(x) = \min\{\abs{x-y}^2 \,:\, y\in Y\},
        \end{equation*}
        with $Y = M \cap \{z \, :\, \abs{z-P_M(x)} \leq 1 \}$ and the set of
        minimizers is $Y(x) = \{P_{M}(x)\}$. So, by Danskin \Cref{thm:danskin}
        \begin{align*}
            \nabla_x d_M^2(x) &= \nabla_x (\abs{x-y}^2) \vert_{y = P_M(x)} \\
                &= 2(x-y)\vert_{y = P_M(x)} \\
                &= 2(x-P_M(x)).
        \end{align*}
%---------%---------%---------%---------%---------%---------%---------%---------
%---------%---------%---------%---------%---------%---------%---------%---------
        \item
        First, we have that $\mathcal{U}$ is a complete metric space by
        \Cref{Lemma_ucompleteMS}. Also, note that $J_{\veps}$ is continuous by
        \Cref{{Thm:Jcontinuity}}, bounded 
        below and
        \begin{align*}
            J_{\veps}(\Bar{u}) &= \veps       \\ 
                &\leq \inf_{u(\cdot) \in \mathcal{\widetilde{U}}_x^{M}}J_{\veps}(u(\cdot)) %
                + \veps.
        \end{align*}
        By \Cref{EVP_Corollary} there is $u^{\veps} \in \mathcal{U}[t,T]$ 
        such that
        \begin{asparaenum}
            \item
                $J_{\veps}(u^{\veps}) \leq J_{\veps}(\Bar{u})$,
            \item
                $\rho(u^{\veps}, \Bar{u}) \leq \sqrt{\veps}$,
            \item
                $J_{\veps}(u^{\veps}) \leq J_{\veps}(u)%
                + \sqrt{\veps}\rho(u, u^{\veps})$.
        \end{asparaenum}
        Thus, $u^{\veps}$ is a minimum of the map 
        \begin{equation*}
            u \longmapsto J_{\veps}(u) + %
            \sqrt{\veps}\rho(u, u^{\veps}).
        \end{equation*}
%---------%---------%---------%---------%---------%---------%---------%---------
%---------%---------%---------%---------%---------%---------%---------%---------
        \item We obtain the necessary conditions for the $\veps$-pair 
        $(x^{\veps}(\cdot), u^{\veps}(\cdot))$. Let $\veps>0$ be fixed and for
        each $0<\delta<1$ and $u\in \mathcal{U}$ we apply the \Cref{cor:spike}
        to the map
        \begin{equation*}
            \tau \longmapsto %
            \begin{pmatrix}
                g(\tau, X^{\veps}(\tau), u(\tau)) %
                    - g(\tau, X^{\veps}(\tau), u^{\veps}(\tau)) \\
                f(\tau, X^{\veps}(\tau), u^{\veps}(\tau)) %
                    - f(\tau, X^{\veps}(\tau), u^{\veps}(\tau)) 
            \end{pmatrix},
        \end{equation*}
        then, there is $E_\delta^{\veps} \in \mathbb{E}_{\delta}$ and
        a function $ r_\delta \in L^{1}([t,T] ; \mathbb{R}^{n}) $ such that 
        \begin{align*}
            \delta  &\int_{t}^{s} %
            \begin{pmatrix}
                g(s, X^{\veps}(\tau), u(\tau)) %
                    - g(\tau, X^{\veps}(\tau), u^{\veps}(\tau)) \\
                f(\tau, X^{\veps}(\tau), u^{\veps}(\tau)) %
                    - f(\tau, X^{\veps}(\tau), u^{\veps}(\tau)) 
            \end{pmatrix} d\tau \\
            &= \int_{t}^{\tau} \ind_{E_\delta^{\veps}}(s)%
            \begin{pmatrix}
                g(s, X^{\veps}(\tau), u(\tau)) %
                    - g(\tau, X^{\veps}(\tau), u^{\veps}(\tau)) \\
                f(\tau, X^{\veps}(\tau), u^{\veps}(\tau)) %
                    - f(\tau, X^{\veps}(\tau), u^{\veps}(\tau)) 
            \end{pmatrix} d\tau + %
            \begin{pmatrix}
                r_\delta^{0,\veps}(s) \\
                r_\delta^{\veps}(s)
            \end{pmatrix}
        \end{align*}
        where 
        $\abs{r_\delta^{0,\veps}(s)}  + \abs{r_\delta^{\veps}(s)} \leq {\delta}^2$ for all $ s \in [t,T]$. Thus, given $u$ and $\delta$, the
        set $E_{\delta}^{\veps}$, given by \Cref{cor:spike} is used to define
        the spike variation $u^{\veps}_{\delta}$ of the optimal control
        $u^{\veps}$ as 
        \begin{equation*}
            u^{\veps}_{\delta}(s) = \left\{ %
                \begin{matrix}
                    u^{\veps}(s) & \text{ if } & s \in [t,T]\setminus E^{\veps}_{\delta} \\
                    u(s)  & \text{ if } &  E^{\veps}_{\delta}
                \end{matrix} \right.
        \end{equation*}
        where $E^{\veps}_{\delta} \subseteq [t,T]$ and 
        $\lambda(E^{\veps}_{\delta}) = \delta(T-t)$. Let $X^{\veps}(\cdot) %
        = X(\cdot, t, x, u^{\veps}(\cdot))$. First, consider
        \begin{equation*}
            I_{g,f} := \int_{t}^{s} %
            \begin{pmatrix}
                g(\tau, X^{\veps}(\tau), u_{\delta}^{\veps}(\tau)) - g(\tau, X^{\veps}(\tau), u^{\veps}(\tau)) \\
                f(\tau, X^{\veps}(\tau), u_{\delta}^{\veps}(\tau)) - f(\tau, X^{\veps}(\tau), u^{\veps}(\tau)) 
            \end{pmatrix} d\tau 
        \end{equation*}.
        
        Splitting  $I_{g,f}$ into the sets $[t,s]\cap E_{\delta}^{\veps}$, 
        $[t,s]\setminus E_{\delta}^{\veps}$ and adding a conveniently zero we have
        \begin{align*}
            \delta I_{g,f}
            &=  \delta \int\limits_{[t,s]\cap E_{\delta}^{\veps}} %
                \begin{pmatrix}
                    g(\tau, X^{\veps}(\tau), u(\tau)) - g(\tau, X^{\veps}(\tau), u^{\veps}(\tau)) \\
                    f(\tau, X^{\veps}(\tau), u(\tau)) - f(\tau, X^{\veps}(\tau), u^{\veps}(\tau)) 
                \end{pmatrix} d\tau \\ 
            &+  \delta \int\limits_{[t,s]\setminus E_{\delta}^{\veps}} %
                \begin{pmatrix}
                    g(\tau, X^{\veps}(\tau), u^{\veps}(\tau)) - g(\tau, X^{\veps}(\tau), u^{\veps}(\tau)) \\
                    f(\tau, X^{\veps}(\tau), u^{\veps}(\tau)) - f(\tau, X^{\veps}(\tau), u^{\veps}(\tau)) 
                \end{pmatrix} d\tau \\ 
            &+  \int\limits_{[t,s]\cap E_{\delta}^{\veps}} %
                \begin{pmatrix}
                    g(\tau, X^{\veps}(\tau), u(\tau)) - g(\tau, X^{\veps}(\tau), u^{\veps}(\tau)) \\
                    f(\tau, X^{\veps}(\tau), u(\tau)) - f(\tau, X^{\veps}(\tau), u^{\veps}(\tau)) 
                \end{pmatrix} d\tau \\
            &- \int\limits_{[t,s]\cap E_{\delta}^{\veps}} %
                \begin{pmatrix}
                    g(\tau, X^{\veps}(\tau), u(\tau)) - g(\tau, X^{\veps}(\tau), u^{\veps}(\tau)) \\
                    f(\tau, X^{\veps}(\tau), u(\tau)) - f(\tau, X^{\veps}(\tau), u^{\veps}(\tau)) 
                \end{pmatrix} d\tau. 
        \end{align*}
        Then 
        \begin{align*}
            \delta I_{g,f} &= \delta \int\limits_{[t,s]\cap E_{\delta}^{\veps}} %
                \begin{pmatrix}
                    g(\tau, X^{\veps}(\tau), u(\tau)) - g(\tau, X^{\veps}(\tau), u^{\veps}(\tau)) \\
                    f(\tau, X^{\veps}(\tau), u(\tau)) - f(\tau, X^{\veps}(\tau), u^{\veps}(\tau)) 
                \end{pmatrix} d\tau \\
            &+  \int\limits_{[t,s]\cap E_{\delta}^{\veps}} %
                \begin{pmatrix}
                    g(\tau, X^{\veps}(\tau), u(\tau)) - g(\tau, X^{\veps}(\tau), u^{\veps}(\tau)) \\
                    f(\tau, X^{\veps}(\tau), u(\tau)) - f(\tau, X^{\veps}(\tau), u^{\veps}(\tau)) 
                \end{pmatrix} d\tau \\
            &- \int\limits_{[t,s]\cap E_{\delta}^{\veps}} %
                \begin{pmatrix}
                    g(\tau, X^{\veps}(\tau), u(\tau)) - g(\tau, X^{\veps}(\tau), u^{\veps}(\tau)) \\
                    f(\tau, X^{\veps}(\tau), u(\tau)) - f(\tau, X^{\veps}(\tau), u^{\veps}(\tau)) 
                \end{pmatrix} d\tau 
        \end{align*}
        From here, using $(D3)$ we have
        \begin{align*}
            &\Bigg\vert \delta I_{g,f}  %
            - \int\limits_{[t,s]\cap E_{\delta}^{\veps}} %
                \begin{pmatrix}
                    g(\tau, X^{\veps}(\tau), u(\tau)) - g(\tau, X^{\veps}(\tau), u^{\veps}(\tau)) \\
                    f(\tau, X^{\veps}(\tau), u(\tau)) - f(\tau, X^{\veps}(\tau), u^{\veps}(\tau)) 
                \end{pmatrix} d\tau \Bigg\vert \\
            &= \Bigg\vert  (\delta -1) \int\limits_{[t,s]\cap E_{\delta}^{\veps}} %
                \begin{pmatrix}
                    g(\tau, X^{\veps}(\tau), u(\tau)) - g(\tau, X^{\veps}(\tau), u^{\veps}(\tau)) \\
                    f(\tau, X^{\veps}(\tau), u(\tau)) - f(\tau, X^{\veps}(\tau), u^{\veps}(\tau)) 
                \end{pmatrix} d\tau \Bigg\vert \\
            &\leq (\delta + 1) \int_{t}^{T}%
                \abs{\begin{pmatrix}
                    \Bar{\omega}(u, u^{\veps}) \\
                    \Bar{\omega}(u, u^{\veps})
                \end{pmatrix}} \ind_{E_{\delta}^{\veps}} d\tau \\
            &= (\delta + 1)\delta (T-t)K \\
            &= (\delta ^{2} + \delta)K
        \end{align*}
        The above implies that
        \begin{align*}
            &\delta \int_{t}^{s} %
            \begin{pmatrix}
                g(\tau, X^{\veps}(\tau), u_{\delta}^{\veps}(\tau)) - g(\tau, X^{\veps}(\tau), u^{\veps}(\tau)) \\
                f(\tau, X^{\veps}(\tau), u_{\delta}^{\veps}(\tau)) - f(\tau, X^{\veps}(\tau), u^{\veps}(\tau)) 
            \end{pmatrix} d\tau \\
            &= \int\limits_{[t,s]\cap E_{\delta}^{\veps}} %
                \begin{pmatrix}
                    g(\tau, X^{\veps}(\tau), u(\tau)) - g(\tau, X^{\veps}(\tau), u^{\veps}(\tau)) \\
                    f(\tau, X^{\veps}(\tau), u(\tau)) - f(\tau, X^{\veps}(\tau), u^{\veps}(\tau)) 
                \end{pmatrix} d\tau + %
                \begin{pmatrix}
                    r_{\delta}^{\veps, 0}(s) \\
                    r_{\delta}^{\veps}(s)
                \end{pmatrix}
        \end{align*}
        with 
        \begin{equation*}
            \abs{r_{\delta}^{\veps, 0}(s)} + \abs{r_{\delta}^{\veps}(s)} \leq \delta^2, \, s\in[t,T].
        \end{equation*}
        
        Now, define $X_{\delta}^{\veps} := X(\cdot; t, x, u_{\delta}^{\veps})$ and
        \begin{equation*}
            Y_{\delta}^{\veps} = \dfrac{X_{\delta}^{\veps}(s) - X^{\veps}(s)}{\delta}, \, s \in [t,T].
        \end{equation*}
        Then, hay algo raro aquí
%---------%---------%---------%---------%---------%---------%---------%---------
%---------%---------%---------%---------%---------%---------%---------%---------
        \begin{align*}
            Y_{\delta}^{\veps} %
            &=  \dfrac{1}{\delta} \int_{t}^{s}\left[f(\tau, X_{\delta}^{\veps}%
                (\tau),u_{\delta}^{\veps}(\tau)) - f(\tau,X^{\veps}(\tau), %
                u^{\veps}(\tau) )\right]d\tau \\
            &=  \dfrac{1}{\delta} \int_{t}^{s}\left[f(\tau, X_{\delta}^{\veps}%
                (\tau), u_{\delta}^{\veps}(\tau)) - f(\tau,X^{\veps}(\tau),%
                u^{\veps}_{\delta}(\tau) )\right]d\tau \\
             &+ \frac{1}{\delta} \int_{t}^{s} \left[f(\tau,X^{\veps}(\tau), %
                u^{\veps}_{\delta}(\tau)) - f(\tau,X^{\veps}(\tau), %
                u^{\veps}(\tau) )\right]d\tau \\
            &=  \dfrac{1}{\delta} \int_{t}^{s}\left[f(\tau, X_{\delta}^{\veps} %
                (\tau), u_{\delta}^{\veps}(\tau)) - f(\tau,X^{\veps}(\tau),%
                u^{\veps}_{\delta}(\tau) )\right]d\tau \\
             &+ \frac{1}{\delta} \int\limits_{[t,s] \cap E_{\delta}^{\veps}} %
                \left[f(\tau,X^{\veps}(\tau), u(\tau)) - f(\tau,X^{\veps}(\tau),%
                u^{\veps}(\tau) )\right]d\tau \\
             &+ \frac{1}{\delta}\int\limits_{[t,s] \setminus E_{\delta}^{\veps}} %
                \left[f(\tau,X^{\veps}(\tau), u^{\veps}(\tau)) - %
                f(\tau,X^{\veps}(\tau),u^{\veps}(\tau) )\right]d\tau \\
            &=  \dfrac{1}{\delta} \int_{t}^{s}\left[f(\tau, X_{\delta}^{\veps} %
                (\tau), u_{\delta}^{\veps}(\tau)) - f(\tau,X^{\veps}(\tau),%
                u^{\veps}_{\delta}(\tau) )\right]d\tau \\
             &+ \frac{1}{\delta} \int\limits_{[t,s] \cap E_{\delta}^{\veps}} %
                \left[f(\tau,X^{\veps}(\tau), u(\tau)) - f(\tau,X^{\veps}(\tau),%
                u^{\veps}(\tau) )\right]d\tau \\
            &=  \int_{t}^{s}\int_{0}^{1}f_{x}\big(\tau, X^{\veps}(\tau) + %
                \theta[X_{\delta}^{\veps}(\tau) - X^{\veps}(\tau)], %
                u_{\delta}^{\veps}(\tau)\big)Y_{\delta}^{\veps}(\tau)d\theta %
                d\tau \\
            &+ \int_{t}^{s}\big[f(\tau,X^{\veps}(\tau),u^{\veps}_{\delta}(\tau))%
                - f(\tau,X^{\veps}(\tau),u^{\veps}(\tau) )\big]d\tau %
                - \dfrac{r_{\delta}^{\veps}(\tau)}{\delta}d\tau .
        \end{align*}
        Now, we let $Y^{\veps}(\cdot)$ be the solution to the following:
        \begin{equation*}
            \begin{aligned}
                \Dot{Y}^{\veps}(s) &= f_{x}(s, X^{\veps}(s), u^{\veps}(s)) Y^{\veps}(s) \\
                    &+ f(s, X^{\veps}(s), u(s)) - f(s, X^{\veps}(s), u^{\veps}(s)), %
                        \hspace{0.5cm} s\in[t,T],\\
                Y^{\veps}(t) &= 0.  
            \end{aligned}
        \end{equation*}
        It is not hard to show that
        \begin{equation*}
            \lim_{\delta \to 0} \norm{Y_{\delta}^{\veps}(\cdot) - %
                Y^{\veps}(\cdot)}_{C([t,T];\mathbb{R}^{n})} = 0.
        \end{equation*}
        On the other hand, by the optimality of $u^{\veps}(\cdot)$, we have
        \begin{align*}
            -\sqrt{\veps} (T-t) &= -\sqrt{\veps}\dfrac{\lambda (E_{\delta}^{\veps}) }{\delta} \\
            &\leq -\sqrt{\veps}\dfrac{\rho(u_{\delta}^{\veps}(\cdot), u^{\veps}(\cdot))}{\delta} \\
            &\leq \dfrac{1}{\delta}[J_{\veps}^{T}(t, x; u_{\delta}^{\veps}(\cdot)) %
                - J_{\veps}^{T}(t, x; u^{\veps}(\cdot))] \\
            &=  \dfrac{\left\{ [J^{T}(t,x; u_{\delta}^{\veps}(\cdot)) %
                + \veps]^{+}\right\}^{2} - \left\{[J^{T}(t,x; u^{\veps}(\cdot)) + %
                \veps]^{+}\right\}^{2}}{\delta[J_{\veps}^{T}(t,x; u_{\delta}^{\veps}(\cdot))%
                + J_{\veps}^{T}(t,x; u_{\delta}^{\veps}(\cdot))]} \\
            &+ \dfrac{d_{M}(X_{\delta}^{\veps}(T))^2 + d_{M}(X^{\veps}(T))^2}%
                {\delta[J_{\veps}^{T}(t,x; u_{\delta}^{\veps}(\cdot))%
                + J_{\veps}^{T}(t,x; u_{\delta}^{\veps}(\cdot))]} \\
            &=  \Bar{\psi}_{\delta}^{\veps, 0} \Big\{\dfrac{1}{\delta}%
                \int_{t}^{T}[g(s, X_{\delta}^{\veps}(s) u_{\delta}^{\veps}) %
                    - g(s, X^{\veps}(s), u^{\veps})]ds  \\
                    &+ \dfrac{h(X_{\delta}^{\veps}(T)) - h(X^{\veps}(T))}{\delta}\Big\} %
                    +\Bar{\psi}_{\delta}^{\veps}Y_{\delta}^{\veps}(T),
        \end{align*}
        where
        \begin{equation*}
            \begin{aligned}
                \Bar{\psi}_{\delta}^{\veps, 0} &= %
                    \dfrac{[J^{T}(t,x; u_{\delta}^{\veps}(\cdot)) + \veps]^{2} + %
                    [J^{T}(t,x; u^{\veps}(\cdot)) + \veps]^{2}}%
                    { J_{\veps}^{T}(t,x; u_{\delta}^{\veps}(\cdot)) %
                    + J_{\veps}^{T}(t,x; u^{\veps}(\cdot))}, \\
                \Bar{\psi}_{\delta}^{\veps} &= %
                    \dfrac{\int_{0}^{1} \nabla(d_{M}^{2})(X^{\veps}(T) + %
                    \theta[X_{\delta}^{\veps}(T) - X^{\veps}(T)])d\theta}%
                    { J_{\veps}^{T}(t,x; u_{\delta}^{\veps}(\cdot)) %
                    + J_{\veps}^{T}(t,x; u^{\veps}(\cdot))}.
            \end{aligned}
        \end{equation*}
        Then
        \begin{equation*}
            \begin{aligned}
                \lim_{\delta \to 0}\Bar{\psi}_{\delta}^{\veps, 0} %
                    &= \Bar{\psi}^{\veps, 0} %
                    \equiv \dfrac{[J^{T}(t,x; u^{\veps}(\cdot)) + \veps]^{+}}%
                    {J_{\veps}^{T}((t,x; u^{\veps}(\cdot))}, \\
                \lim_{\delta \to 0}\Bar{\psi}_{\delta}^{\veps} %
                    &= \Bar{\psi}^{\veps}\dfrac{d_{M}(X^{\veps}(T))\partial %
                    d_M(X^{\veps}(T))}{J_{\veps}^{T}((t,x; u^{\veps}(\cdot))},
            \end{aligned}
        \end{equation*}
        where 
        \begin{equation*}
            \partial d_M(X^{\veps}(T)) = %
            \left\{ %
                \begin{matrix}
                    \dfrac{X^{\veps}(T) - X^{\veps}(T)}{\norm{X^{\veps}(T) - P_M(X^{\veps}(T))}}, & X^{\veps}(T) \notin M, \\
                    \nu(X^{\veps}(T)), & X^{\veps}(T) \notin \partial M, \\
                    0, & X^{\veps}(T) \notin M \setminus \partial M.
                \end{matrix}
            \right.
        \end{equation*}
        In the above, recall that $P_M(X^{\veps}(T))$ is the projection of 
        $X^{\veps}(T)$ onto the convex set $M$, $\nu(X^{\veps}(T))$ is a unit 
        outward normal of $M$ at $X^{\veps}(T) \in \partial M$, defined to be
        a unit vector satisfying
        \begin{equation*}
            \langle \nu(X^{\veps}(T)), y - X^{\veps}(T)\rangle \leq 0, \forall \in M.
        \end{equation*}
        It's clear that
        \begin{equation*}
            \begin{aligned}
                &\abs{ \Bar{\psi}^{\veps,0}}^2 + \abs{\Bar{\psi}^{\veps}}^{2} = 1, \forall \epsilon > 0, \\
                &\langle  \Bar{\psi}^{\veps} , y - X^{\veps}(T)\rangle \leq 0, \forall y\in M.
            \end{aligned}    
        \end{equation*}
  
        Further,
        \begin{align*}
            &\dfrac{1}{\delta} \int\limits_{t}^{T}\Big[%
                g(\tau, X_{\delta}^{\veps}(\tau), u_{\delta}^{\veps}(\tau))%
                - g(\tau, X^{\veps}(\tau),u^{\veps}(\tau))\Big] d\tau \\
            &= \dfrac{1}{\delta} \int\limits_{t}^{T}\Big[%
                g(\tau, X_{\delta}^{\veps}(\tau), u_{\delta}^{\veps}(\tau))%
                - g(\tau, X^{\veps}(\tau),u^{\veps}(\tau))\Big] d\tau \\
            &+ \dfrac{1}{\delta} \int\limits_{[t,s]\cap E_{\delta}^{\veps}}\Big[%
                g(\tau, X^{\veps}(\tau), u(\tau))%
                - g(\tau, X^{\veps}(\tau),u^{\veps}(\tau))\Big] d\tau \\
            &= \int\limits_{t}^{T} \int\limits_{0}^{1} %
                g_x(\tau, X^{\veps} + \theta[X_{\delta}^{\veps}(\tau)-%
                X^{\veps}(\tau)], u_{\delta}^{\veps}(\tau))Y_{\delta}^{\veps}(\tau)%
                d\theta d\tau \\
            &+ \int\limits_{t}^{T}\Big[g(\tau, X^{\veps}(\tau), u(\tau))%
                - g(\tau, X^{\veps}(\tau),u^{\veps}(\tau))\Big]d\tau %
                + \int\limits_{t}^{s}\dfrac{r_{\delta}^{\veps,0}(\tau)}{\delta} d\tau \\
            &\to \int\limits_{t}^{T}\Big[g_{x}(\tau, X^{\veps}(\tau), u^{\veps}(\tau))%
                Y^{\veps}(\tau) + g(\tau, X^{\veps}, u(\tau)) - g(\tau, X^{\veps}(\tau), %
                u^{\veps}(\tau))\Big] d\tau.
        \end{align*}
        \item
            Consequently, letting $\delta \to 0$ in 2.12 , one has
            \begin{align*}
                -\sqrt{\veps}(T-t) 
                &
                \leq \Bar{\phi} ^ {\veps,0}
                \int_{t} ^ {T}
                \Big[%
                    g_{x}(\tau, X^{\veps}(\tau), u^{\veps}(\tau))
                    Y^{\veps}(\tau) 
                    \\
                    &+ %
                    g(\tau, X ^ {\veps} (\tau), u(\tau))
                    - 
                    g(\tau, X^{\veps}(\tau), u^{\veps}(\tau))
                \Big]
                d \tau 
                + \Big[
                    \Bar{\psi} ^ {\veps, 0} h_x (X^{\veps}(T)) %
                    + \Bar{\psi}^{\veps}
                \Big]
                Y ^ {\veps}(T).
            \end{align*}
            By (2.13), we may assume that $(\psi^{\veps, 0}, \psi^{\veps})$ is 
            convergent, as $\veps \to 0$. Denote
            \begin{equation*}
                \lim_{\veps\to 0}(\psi^{\veps, 0}, \psi^{\veps}) = -(\psi^{0}, \psi),
            \end{equation*}
            with
            \begin{equation*}
                \begin{aligned}
                    &\abs{\psi^{0}}^2 + \abs{\psi}^2 = 1, \, \psi^{0} \leq 0, \\
                    &\langle \psi, y - \bar{X}(T) \rangle \geq 0.
                \end{aligned}
            \end{equation*}
            Also,
            \begin{equation*}
                \lim_{\veps \to 0} \norm{Y^{\veps}(\cdot) - Y(\cdot)}_{C([t,T]; %
                \mathbb{R}^{n})} = 0,
            \end{equation*}
            where
            \begin{align*}
                \Dot{Y}(s) &= f_x(s, \bar{X}(s), \Bar{u}(s))Y(s) + f(s, \bar{X}(s), u(s))
                    - f(s, \bar{X}(s), \Bar{u}(s)), \\
                Y(t) &= 0,
            \end{align*}
            with $s \in [t,T]$.
            Let $\psi(\cdot)$ be the solution of the adjoint equation (2.9) with
            \begin{equation*}
                \psi(T) = \psi + \psi^{0}h_x(\bar{X}(T)).
            \end{equation*}
            Then (2.10) - (2.11) hold. Now, let $\veps \to 0$ in (2.14), we obtain
            \begin{align*}
                0 &\geq \int_{t}^{T}\psi^{0}\Big[g_x(s, \bar{X}(s), u(s))Y(s) + %
                    g(s, \bar{X}(s), u(s)) - g(s, \bar{X}(s), \Bar{u}(s))\Big]ds \\
                &+ \Big[ \psi^{0}h_x(X(T)) + \psi(s) \Big]Y(T) \\
                &=  \int_{t}^{T}\Big\{\psi^{0}\Big[g_x(s, \bar{X}(s), u(s)) Y(s) %
                    + g(s, \bar{X}(s), u(s)) - g(s, \bar{X}(s), \bar{u}(s))\Big] \\
                &+ \big\langle \Dot{\psi}(s), Y(s)\big\rangle + %
                    \big\langle \psi(s), f_x(s, \bar{X}(s), \bar{u}(s))Y(s)\big\rangle \\
                &+ \big\langle \psi(s), f(s, \bar{X}(s), u(s)) - f(s, \bar{X}(s), %
                    \bar{u}(s)) \big\rangle \Big\} ds \\
                &=  \int_{t}^{T}\Big\{ \psi^{0}\Big[g(s, \bar{X}(s), u(s)) - %
                    g(s, \bar{X}(s), \bar{u}(s))\Big] \\
                &+ \big\langle \psi(s), f(s,\bar{X}(s),u(s)) - f(s,\bar{X}(s),%
                    \bar{u}(s)) \big\rangle\Big\}ds \\
                &= \int_{t}^{T}\Big[H\big(s,\bar{X}(s),u(s), \psi^{0}, \psi(s) \big) %
                    - H \big(s,\bar{X}(s),\bar{u}(s),\psi^0,\psi(s) \big)\Big]ds.
            \end{align*}
            This implies (2.8).
    \end{asparaenum}
%---------%---------%---------%---------%---------%---------%---------%---------
%---------%---------%---------%---------%---------%---------%---------%---------
\end{proof}

    Now, we introduce a particular case of this Theorem. This is presented in
    \citep{lenhart2007optimal} and we are going to use this one on the followings
    chapters.
\newpage