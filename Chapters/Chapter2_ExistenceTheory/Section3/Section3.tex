\section{Existence theory for optimal policies}
%---------%---------%---------%---------%---------%---------%---------%---------
    In Section 3 we prove the existence of an optimal policy for the $(OC)^T$
    problem.
    
    Let $T\in (0,\infty)$ be fixed. Consider the control system 
    $$\left\{ \begin{array}{l}
    	\dot{X}(s)=f(s,u(s),X(s)\,\,s\in [t,T], \\
    	X(t)=x,\\
    \end{array}
    \right.
    $$
    with terminal state constraint
    $$
        X(T;t,x,u(\cdot))\in M,
    $$
    where $M\subseteq \mathbb{R}^n$ is fixed, and with cost functional
    \begin{equation*}
        J^T(t,x;u(\cdot))=\int_{t}^{T}g(s,u(s),X(s)) ds + h(X(T)).
    \end{equation*}
    We recall 
    $$
        \tilde{\mathcal{U}}^M_x[t,T] = %
            \{u(\cdot)\in\mathcal{U}[t,T] \, : \, X(T;t,x,u(\cdot))\in M\},
    $$
    and recall the following optimal control problem:
%---------%---------%---------%---------%---------%---------%---------%---------

%---------%---------%---------%---------%---------%---------%---------%---------
    \textbf{Problem $(OC)^T$} : For given $(t,x)\in \mathbb{R}_{+}\times \mathbb{R}^n$ 
    with $\tilde{\mathcal{U}}^M_x[t,T]\neq\emptyset$, find a $\bar{u}(\cdot) \in %
    \tilde{\mathcal{U}}^M_x[t,T]$ such that
    \begin{equation}\label{eq2.4} 
        J^T(t,x;\bar{u}(\cdot))=\inf_{u(\cdot)\in \tilde{\mathcal{U}}^M_x[t,T]} %
        J^T(t,x;u(\cdot))\equiv V(t,x).
    \end{equation}
%---------%---------%---------%---------%---------%---------%---------%---------
    Any $\bar{u}(\cdot)\in \tilde{\mathcal{U}}^M_x[t,T]$ satisfying (\cref{eq2.4}) 
    is called an optimal control, the corresponding $\bar{X}(\cdot)\equiv %
    X(\cdot;t,x,\bar{u}(\cdot))$ is called an optimal state trajectory, and %
    $(\bar{u}(\cdot),\bar{X}(\cdot))$ is called optimal pair, we call $V(\cdot,\cdot)$ %
    the value functional of problem $(OC)^T$.
%---------%---------%---------%---------%---------%---------%---------%---------

%---------%---------%---------%---------%---------%---------%---------%---------
We introduce the following assumption for the cost functional:
\begin{asparaenum}\label{Cond:C2}
    \item[(\textbf{C}-2)]
        The maps $g:\mathbb{R}_{+}\times U\times \mathbb{R}^n\to \mathbb{R}$
        and $h.\mathbb{R}^n\to \mathbb{R}$ are measurable and there exists 
        a continuous function $\omega:\mathbb{R}_{+}\times\mathbb{R}_{+}\to %
        \mathbb{R}_{+}$, called a local modulus of continuity, which is increasing 
        in each argument, and $\omega(r,0)=0$ for every $r\geq 0$, such that 
        $$
            \abs{g(s,u,x_1)-g(s,u,x_2)} +\abs{h(x_1)-h(x_2)}\leq \omega(\abs{x_1}%
            \vee \abs{x_2},\abs{x_1-x_2}),
        $$
        for every $ (s,u)\in \mathbb{R}_{+}\times U,x_1,x_2\in \mathbb{R}^n$, where 
        $|x_1|\vee |x_2|=\max\{|x_1|,|x_2|\}$, and
        $$
            \sup_{(s,u)\in \mathbb{R}_{+}\times U}|g(s,u,0)|\equiv g_0<\infty.
        $$
\end{asparaenum}
%---------%---------%---------%---------%---------%---------%---------%---------

%---------%---------%---------%---------%---------%---------%---------%---------
%In what follows, $\omega(\cdot.\cdot)$ will stand for a generic local modulus of continuity which can be different from line to line.

    For any $(t,x)\in [0,T]\times\mathbb{R}^n$, let us introduce the following set
    $$
        \mathbb{E}(t,x)=\{(z^0,z)\in \mathbb{R}\times \mathbb{R}_{+}|z^0    %
        \geq g(t,u,x),z=f(t,u,x),\, u\in U\}.
    $$
    
    The following assumption gives some compatibility between the control system 
    and the cost functional.
%---------%---------%---------%---------%---------%---------%---------%---------

%---------%---------%---------%---------%---------%---------%---------%---------
    \begin{asparaenum}\label{Cond:C3}
        \item[(\textbf{C}-3)]
            For almost all $t\in [0,T]$, the following Cesari property holds at any
            $x\in \mathbb{R}^n$,
            $$
                \bigcap_{\delta>0}\bar{co}\mathbb{E}(t,B_{\delta}(x))=\mathbb{E}(t,x),
            $$
            where, we recall that $B_{\delta}(x)$ is the open ball centered at $x$ 
            with radius $\delta>0$, and $\bar{co}(E)$ stands for the closed convex 
            hull of the set $E$ (the smallest closed convex set containing E).
    \end{asparaenum}
    Observe that if $\mathbb{E}(t,x)$ has Cesari property at $x$, the $\mathbb{E}(t,x)$ 
    is convex and closed.
%---------%---------%---------%---------%---------%---------%---------%---------

%---------%---------%---------%---------%---------%---------%---------%---------
    \begin{theorem}[{\cite[Thm.2.2.1, p. 40]{YongDG_ACIntro}}]\label{thm:ExistsTheo}
    	Let (C1)-(C3) hold. Let $M\subseteq \mathbb{R}^n$ be a non-empty closed set. 
    	Let $(t,x)\in [0,T]\times\mathbb{R}^n$ be given and 
    	$\tilde{\mathcal{U}}^M_x[t,T]\neq \emptyset$. Then, the $(OC)^T$ problem
    	(\ref{eq2.4}) admits at least one optimal pair.
    \end{theorem}
%---------%---------%---------%---------%---------%---------%---------%---------
%---------%---------%---------%---------%---------%---------%---------%---------
    \begin{proof}
        Let $u_k(\cdot)\in\tilde{\mathcal{U}}^M_x[t,T]$ be a minimizing sequence. 
        By proposition \Cref{prop2.1.1}
        \begin{equation}\label{eq2.5}
            |X_k(s)|\leq e^{L(s-t)}(1+|x|)-1,\, s\in [t,T] \,\,k\geq 1.
        \end{equation}
        and for any $t\leq \tau<s\leq T$,
        \begin{eqnarray*}
            |X_k(s)-X_k(\tau)|&=& |X(s;t,x,u_k(\cdot))-X(\tau;t,x,u(\cdot))|\\
            &\leq& [e^{L(s-\tau)}-1][1+X(\tau;t,x,u_k(\cdot))]\\
            &\leq& [e^{L(s-\tau)}-1]e^{L(\tau-t)}(1+|x|).
        \end{eqnarray*}
        
        Hence, the sequence $\{X_k(\cdot)\}$ is uniformly bounded and equi-continuous. 
        Therefore, By Arzela-Ascoli \Cref{AAT}, we may assume that the sequence is 
        convergent to some $\bar{X}(\cdot)$ in $C([t,T];\mathbb{R}^n)$. On the other 
        hand,
        $$
            |f(d,u_k(s),X_k(s)|\leq L(1+|X_k(s))\leq Le^{L(s-t)}(1+|x|).
        $$
        Also, by \cref{eq2.5} and (C2) (\ref{Cond:C2}), we have
        \begin{eqnarray*}
            |g(s,u_k(s),X_k(s))|&\leq& |g(s,u_k(s),0)|+|g(s,u_k(s),X_k(s))-g(s,u_k(s),0)|\\
            &\leq& g_0+ \omega(|X_k(s)|,|X_k(s)|)\leq g_0+\omega(e^{LT}(1+|x|),e^{LT}(1+|x|))\\
            &\leq& K,\,s\in [t,T],\,k\geq 1.
        \end{eqnarray*}
        Hence, by extracting a subsequence if necessary, we may assume that
        $$
            \left\{ 
            \begin{array}{l}
                g(\cdot,u_k(\cdot),X_k(\cdot))\to \bar{g}(\cdot),\,\mbox{ weakly in }\, %
                    L^2([t,T];\mathbb{R}^n), \\
                f(\cdot,u_k(\cdot),X_k(\cdot))\to \bar{f}(\cdot),\,\mbox{ weakly in }\, %
                    L^2([t,T];\mathbb{R}^n),\\
            \end{array}
            \right.
        $$
        for some $\bar{g}(\cdot)$ and $\bar{f}(\cdot)$. Then by Banach-Saks Theorem (\ref{BST}), 
        we have
        \begin{equation}\label{eq2.6}
            \left\{ \begin{array}{l}
                \tilde{g}_k(\cdot):=\frac{1}{k}\sum_{i=1}^{k}g(\cdot,u_i(\cdot),X_i(\cdot))\to \bar{g}(\cdot),\,\mbox{strongly in}\, L^2([t,T];\mathbb{R}^n), \\
                \tilde{f}_k(\cdot):=\frac{1}{k}\sum_{i=1}^{k}f(\cdot,u_i(\cdot),X_i(\cdot))\to \bar{f}(\cdot),\,\mbox{strongly in}\, L^2([t,T];\mathbb{R}^n).\\
                \end{array}
            \right.
        \end{equation}
        On the other hand, by (C1) and the convergence of $X_k(\cdot)\to \bar{X}(\cdot)$ 
        in $C([t,T];\mathbb{R}^n)$, we have
        \begin{eqnarray*}
            |\tilde{f}_k(s)-\frac{1}{k}\sum_{i=1}^{k}f(s,u_i(s),\bar{X}(s)| %
            &\leq& \frac{1}{k}\sum_{i=1}^{k}|f(s,u_i(s),X_i(s))-f(s,u_i(s),\bar{X}(s))| \\
            &\leq&\frac{1}{k}\sum_{i=1}^{k}|X_i(s)-\bar{X}(s)|\to 0\,\,k\to \infty,
        \end{eqnarray*}
        uniformly in $s\in [t,T]$. Similarly, by (C2),
        \begin{eqnarray*}
        	|\tilde{g}_k(s)-\frac{1}{k}\sum_{i=1}^{k}g(s,u_i(s),\bar{X}(s)|&\leq& \frac{1}{k}\sum_{i=1}^{k}|g(s,u_i(s),X_i(s))-g(s,u_i(s),\bar{X}(s))|\\
        	&\leq&\frac{1}{k}\sum_{i=1}^{k}\omega(|X_i(s)|\vee |\bar{X}(s)|,%
        	|X_i(s)-\bar{X}(s)|)\to 0\,\,k\to \infty,
        \end{eqnarray*}
        
        uniformly in $s\in [t,T]$. Next, by the definition of $\mathbb{E}(t,x)$, we have
        
        $$\begin{pmatrix}
        	g(s,u_i(s),X_i(s)) \\ f(s,u_i(s),X_i(s))
        \end{pmatrix} \in \mathbb{E}(s,X_i(s)),\,i\geq 1,\, s\in [t,T].$$
        
        Hence, for any $\delta>0$, there exits a $K_{\delta}>0$ such that 
        
        \begin{equation}\label{eq2.7}
        	\begin{pmatrix}
        		\tilde{g}_k(s) \\ \tilde{f}_k(s)
        	\end{pmatrix}\in \bar{co}\mathbb{E}(s,B_{\delta}(\bar{X}(s))),\, K\geq K_{\delta},\, s\in[t,T].
        \end{equation}
        
        Combinig (\ref{eq2.6}) and (\ref{eq2.7}), using (C3), we obtain
        
        \begin{equation*}
        \begin{pmatrix}
        \bar{g}(s) \\ \bar{f}(s)
        \end{pmatrix}=\lim_{k\to \infty}\begin{pmatrix}
        \tilde{g}_k(s) \\ \tilde{f}_k(s)
        \end{pmatrix}\in \bigcap_{\delta>0}\bar{co}\mathbb{E}(s,B_{\delta}(\bar{X}(s)))=\mathbb{E}(s,\bar{X}(s)).
        \end{equation*}
        
        Then by Filippov's Lemma (\ref{FL}), there exits a $\bar{u}(\cdot)\in \mathcal{U}[t,T]$ such that
        
        $$\bar{g}(s)\geq g(s,\bar{u}(s),\bar{X}(s)),\,\bar{f}(s)=f(s,\bar{u}(s),\bar{X}(s)),\, s\in [t,T].$$
        
        This means $\bar{X}(\cdot)=X(\cdot;t,x,\bar{u})$. On the other hand, since
        
        $$\bar{X}_k(T)\equiv X(T;t,x,\bar{u}_k(\cdot))\in M,\, k\geq 1$$
        
        one has 
        $$\bar{X}(T)\equiv X(T;t,x,\bar{u}(\cdot))\in M,$$
        
        which means that $\bar{u}(\cdot)\in \tilde{\mathcal{U}}^M_x[t,T]$. finally, by Fatou's Lemma
        
        \begin{eqnarray*}
            J^T(t,x;\bar{u}(\cdot))%
            &\leq \int_{t}^{T}\bar{g}(s)ds +h(\bar{X}(T))\\
            &\leq \liminf_{k\to \infty}[\int_{t}^{T}\tilde{g}_k(s)ds + h(X_k(T))]\\
            &=  \lim_{k\to \infty} \frac{1}{k}  J^T(t,x;u_i(\cdot))\\
            &=  \lim_{k\to \infty} J^T(t,x;u_k(\cdot))\\
            &=  \inf_{u(\cdot)\in\tilde{\mathcal{U}}^M_x[t,T]}J^T(t,x;u(\cdot)).
        \end{eqnarray*}
        
        This means that $(\bar{u}(\cdot),\bar{X}(\cdot))$ is an optimal pair.
    \end{proof}

%---------%---------%---------%---------%---------%---------%---------%---------
%---------%---------%---------%---------%---------%---------%---------%---------