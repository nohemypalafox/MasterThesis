\section{Mold and Fungicide} %\label{Chap2-Sect1} 
%---------%---------%---------%---------%---------%---------%---------%---------
%---------%---------%---------%---------%---------%---------%---------%---------
    For the first example, we consider a concentration of mold at a time $t$,
    denoted by $x(t)$, that we want to reduce over a fixed period of time. We
    assume $x$ grows at a rate $r$ and has a carrying capacity $M$. To reduce
    the mold population we have to apply some chemicals, like fungicide. We
    denote by $u(t)$ the amount of fungicide added to kill the mold at time $t$.
    The effects of the mold and the fungicide are negative for the individuals
    around them \textemdash plants, animals, etc. \textemdash, so the aim is to
    minimize both of them. The minimization problem we consider is from
    \citep{lenhart2007optimal} and it is as follows
    
    \begin{align*}
    	\min_{u} &\int_{0}^{T} (Ax^2(t) + u^2(t))dt \\
        \text{subject to } x'(t) &= r(M - x(t)) - u(t)x(t), \ x(0) = x_0 > 0,
    \end{align*}
    
    \noindent where $u$ and $x$ are quadratic terms since we need to penalize larger amounts 
    of mold and fungicide. Also, $A$ is a weight parameter that 
    balances the relative importance of the two terms in the objective functional. 
    Note that the Hamiltonian, the optimality condition, the adjoint equation and the transversality 
    conditions are
    \begin{align*}
        &H = Ax^2 + u^2 + \lambda [r(M-x) - ux], \\
        &2u - \lambda x = 0, \\
        &\lambda ' = -2Ax + \lambda(r + u), \\
        &\lambda(T) = 0.
    \end{align*}
    respectively. These conditions are used in the forward-backward sweep method.
        
    \noindent Note that in Figure \ref{Figure_MF_1} that, initially, the 
    control and the state increases. Then, both become constant, here we 
    say that the control and the state are in equilibrium because both stay 
    at constant value. Then, the control eventually begins decreasing and
    vanishes to zero, but the state never decreases. We want to decrease or
    eliminate the mold concentration. To do this we have to use a higher 
    weight parameter $A$.
    
    \begin{figure}[h] 
    	\begin{center}
        	\includegraphics[width=.65\textwidth,keepaspectratio]%
            {Chapters/Chapter5_Applications1/Figures/MF_state_control_adjoint_1}
    		\caption{%
            	The horizontal axis represents time $t$. The vertical axis
                represents, in each case, the state $x$, the control $u$ and the
                solution of the adjoint equation $\lambda$. The parameters are
          		$r = \num{0.3}$, 
          		$M = \num{10}$, 
          		$A = \num{1}$, 
          		$x_0 = \num{1}$, 
          		$T =\num{5}$.
        	}\label{Figure_MF_1}
    	\end{center}
    \end{figure}
%---------%---------%---------%---------%---------%---------%---------%---------
%---------%---------%---------%---------%---------%---------%---------%---------
    \noindent Now, let $A = 10$ and see \ref{Figure_MF_2}.  First, with this
    change, the level of fungicide is much higher. Also, we can notice that the
    state and the control still have a long period of equilibrium. The control
    begins at its greatest point, decreasing slightly before becoming constant,
    then decreases again. As desired, the state decreases from its initial
    amount and then becomes constant. However, when the level of fungicide
    starts to decrease again, at the end of the interval, the level of mold
    rapidly increases. Now, we compare the optimal state with a state where no
    fungicide is used.
    
     \begin{figure}[h] 
    	\begin{center}
        	\includegraphics[width=.65\textwidth,keepaspectratio]%
        	{Chapters/Chapter5_Applications1/Figures/MF_state_control_adjoint_2}
    		\caption{%
            	The horizontal axis represents time $t$. The vertical axis represents 
            	the mold concentration $x$. The optimal mold population is 
            	represented by a solid line, and the mold population without control 
            	is the dashed line. The optimal increases at the end of the interval, 
            	but is held much lower overall than if no fungicide was used. The
            	parameters are:
          		$r = \num{0.3}$, 
          		$M = \num{10}$, 
          		$A = \num{10}$, 
          		$x_0 = \num{1}$, 
          		$T =\num{5}$.
        	}\label{Figure_MF_2}
    	\end{center}
    \end{figure}    
    
    In Figure \ref{Figure_MF_4} we present a comparison between the uncontrolled 
     problem and the controlled one with the same parameters used below.
     
    \begin{figure}[h] 
    	\begin{center}
        	\includegraphics[width=.65\textwidth,keepaspectratio]%
        	{Chapters/Chapter5_Applications1/Figures/MF_control_uncontrol}
    		\caption{%
            	The horizontal axis represents time $t$. The vertical axis represents 
            	the mold concentration $x$. The optimal mold population is 
            	represented by a solid line, and the mold population without control 
            	is the dashed line. The optimal increases at the end of the interval, 
            	but is held much lower overall than if no fungicide was used. The
            	parameters are:
          		$r = \num{0.3}$, 
          		$M = \num{10}$, 
          		$A = \num{10}$, 
          		$x_0 = \num{1}$, 
          		$T =\num{5}$.
        	}\label{Figure_MF_3}
    	\end{center}
    \end{figure}    
%---------%---------%---------%---------%---------%---------%---------%---------
%---------%---------%---------%---------%---------%---------%---------%--------- 
    
    Now, if we vary the carrying capacity $M$, the mold would increase quickly.
    We need to balance this effect with the mold concentration and the fungicide. 
    Consider $r = \num{0.6}$, $M = \num{5}$, $A = \num{10}$, $x_0 = \num{5}$, 
    $T =\num{5}$. See \cref{Figure_MF_4}. Here, the control begins with an extreme
    concentration of fungicide and then, rapidly,the mold and the fungicide achieve a small value and become constant again. So, an initial value like this for the 
    fungicide is devastating for the mold population.
    
    \begin{figure}[h] 
    	\begin{center}
        	\includegraphics[width=.65\textwidth,keepaspectratio]%
        	{Chapters/Chapter5_Applications1/Figures/MF_state_control_adjoint_3}
    		\caption{%
            	The horizontal axis represents time $t$. The vertical axis
                represents, in each case, the mold concentration $x$, the 
                fungicide $u$ and the solution of the adjoint equation $\lambda$.
                The parameters are
          		$r = \num{0.6}$, 
                $M = \num{5}$, 
                $A = \num{10}$, 
                $x_0 = \num{5}$, 
                $T =\num{5}$.
        	}\label{Figure_MF_4}
    	\end{center}
    \end{figure}
    
    
    
\newpage
