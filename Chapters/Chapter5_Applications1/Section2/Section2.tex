\section{Bacteria}	%\label{Chap2-Sect2}
	
%---------%---------%---------%---------%---------%---------%---------%---------
%---------%---------%---------%---------%---------%---------%---------%---------
    Suppose that we have a bacteria growing in a laboratory. If we let the bacteria alone, it will grow exponentially with a growth rate $r$. We can add a certain chemical nutrient that accelerates the reproduction process of the bacteria and also creates a side effect, a byproduct chemical that retards the growth.  It is believed that the size of the bacteria population and the level of impediment are related. The larger the bacteria population is, the smaller the effect of this byproduct will have. So, we consider the following problem from \citep{lenhart2007optimal}.
    
    \begin{align*}
    	&\max_{u} \ Cx(1) - \int_{0}^{1} u^2(t) dt \\
        \text{subject to } \ x'(t) &= rx(t) + Au(t)x(t) - Bu^2(t)\exp{-x(t)},% 
        \ x(0) = x_0,
    \end{align*}
    
    \noindent with $A, B, C \geq 0$. In this problem, $x(t)$ represents the bacteria
    concentration at time $t$, and $u(t)$ is the amount of the chemical being
    added at time $t$. The parameter $A$ is the relative strength of the 
    chemical nutrient increasing growth, $B$ is the strength of the byproduct 
    and $r$ is the growth rate. For this maximization problem, the Hamiltonian
    is $H = u^2 + rx + \lambda[Aux - Bu^2 e^{-x}]$, then the optimality condition
    and the adjoint equation are
    
    \begin{align*}
        &\lambda Ax + 2u(1 - \lambda Be^{-x}) = 0, \\
        & \lambda' = -r - \lambda(Au + Bu^2 e^{-x}).
    \end{align*}
    
    \noindent Note that in this example we have a payoff term. So, the 
    transversality condition is   $\lambda (1) = C$.
    
    \begin{figure}[h] 
    	\begin{center}
        	\includegraphics[width=.65\textwidth,keepaspectratio]%
        	{Chapters/Chapter5_Applications1/Figures/Bacteria_control_state}
    		\caption{%
            	The horizontal axis represents time $t$. The vertical axis
                represents, in each case, the state $x$ and the control $u$.
                The parameters are
          		$r = \num{1}$, 
          		$A = \num{1}$,
                $B = \num{12}$,
                $C = \num{1}$,
          		$x_0 = \num{1}$, 
        	}\label{Figure_BG_1}
    	\end{center}
    \end{figure}
    
    \noindent In Figure \ref{Figure_BG_1}, we can observe that if $x$ becomes larger, then the $e^{-x}$ term decreases, and the second chemical has less of a hindering effect. Consequently, the level of chemical ($u$) added starts fairly low and steadily increases, with noticeably higher rates of increase around $t = 0.6$ and $t = 0.8$. As such, the bacteria growth is approximately exponential early in the time interval, but begins to increase more and more rapidly. Figure \ref{Figure_BG_2} compares the controlled bacteria population and the uncontrolled using the parameters in Figure \ref{Figure_BG_1}. 
    
    \begin{figure}[h] 
    	\begin{center}
        	\includegraphics[width=.65\textwidth,keepaspectratio]%
            {Chapters/Chapter5_Applications1/Figures/Bacteria_control_uncontrol}
    		\caption{%
                The horizontal axis represents time $t$. The vertical axis
                represents the bacteria concentration ($x(t)$). The parameters
                are
                $r = \num{1}$,
                $A = \num{1}$,
                $B = \num{12}$,
                $C = \num{1}$,
          		$x_0 = \num{1}$,
        	}\label{Figure_BG_2}
    	\end{center}
    \end{figure}

    \noindent Now, consider a small initial population without varying any other parameters, see Figure \ref{Figure_BG_3}. We can observe that the amount of chemical used is smaller.  Due to this small initial count, the bacteria population never gets large enough for the byproduct to be as insignificant as before.
    
    \begin{figure}[h] 
    	\begin{center}
        	\includegraphics[width=.65\textwidth,keepaspectratio]%
            {Chapters/Chapter5_Applications1/Figures/B_state_control_adjoint}
    		\caption{%
            	The horizontal axis represents time $t$. The vertical axis
                represents, in each case, the state $x$, the control $u$ and the
                solution of the adjoint equation $\lambda$. The
                parameters are
          		$r = \num{1}$, 
          		$A = \num{1}$,
                $B = \num{12}$,
                $C = \num{1}$,
          		$x_0 = \num{0.1}$, 
        	}\label{Figure_BG_3}
    	\end{center}
    \end{figure}

\newpage 



