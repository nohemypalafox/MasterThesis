\section{Mold and Fungcide} %\label{Chap2-Sect1} 

%---------%---------%---------%---------%---------%---------%---------%---------
%---------%---------%---------%---------%---------%---------%---------%---------

Consider the minimization problem
\begin{align*}
	\min_{u} &\int_{0}^{T} (Ax^2(t) + u^2(t))dt \\
    \text{subject to } x'(t) &= r(M - x(t)) - u(t)x(t), \ x(0) = x_0,
\end{align*}
where $x$ denote the concentration of a mold with $x_0 > 0$ and $u$, the amount 
of fungcide known to kill it. The coefficient $r$ is the growth rate, $M$ is the
carrying capacity and the coefficient $A$ is the weight parameter, balancing the
relative importance of the two terms in the objective functional. 

In the next figure, note that, initially, the control and the state increases then, both levels off to become constant, here we say the control and the 
state are in equilibrium because both stay at constant values. Then, the 
control eventually begins decreasing and vanishes to zero, but the state 
never decreases, with heavy growth at the beginning and end of the interval
and constant in the middle. We want to decrease or eliminate the mold concentration that is the state $x$. To do this we have to use a higher weight
parameter.

\begin{figure}[htb] 
	\begin{center}
    	\includegraphics[width=.65\textwidth,keepaspectratio]%
        {Chapters/Chapter2/Figures/MF_state_control_adjoint_1}
		\caption{%
        	The horizontal axis represents time $t$. The vertical axis
            represents, in each case, the state $x$, the control $u$ and the
            solution of the adjoint equation $\lambda$. The parameters are
      		$r = \num{0.3}$, 
      		$M = \num{10}$, 
      		$A = \num{1}$, 
      		$x_0 = \num{1}$, 
      		$T =\num{5}$.
    	}\label{Figure_MF_1}
	\end{center}
\end{figure}

Now, consider the weight parameter $A = 10$ and see \ref{Figure_MF_2}. Firstable, with this change, the level of fungcide used is much higher. Also,
we can notice that the state and the control still have a long period of 
equilibrium. The control begins at its greatest point, decreasing slightly 
before becoming constant, then decrease again. As desire, the state decreases
from its initial amount and then becomes constant. However, when the level of
fungcide start to decrease again, at the end of the interval, the level of
mold rapidly increases. Now, we compare the optimal state with a state where
no fungcide is used.

\begin{figure}[htb] 
	\begin{center}
    	\includegraphics[width=.65\textwidth,keepaspectratio]%
    	{Chapters/Chapter2/Figures/MF_state_control_adjoint_2}
		\caption{%
        	The horizontal axis represents time $t$. The vertical axis
            represents, in each case, the state $x$, the control $u$ and the
            solution of the adjoint equation $\lambda$. The parameters are
      		$r = \num{0.3}$, 
      		$M = \num{10}$, 
      		$A = \num{10}$, 
      		$x_0 = \num{1}$, 
      		$T =\num{5}$.
    	}\label{Figure_MF_2}
	\end{center}
\end{figure}

\begin{figure}[htb] 
	\begin{center}
    	\includegraphics[width=.65\textwidth,keepaspectratio]%
    	{Chapters/Chapter2/Figures/MF_control_uncontrol}
		\caption{%
        	The horizontal axis represents time $t$. The vertical axis
            represents the mold concentration $x$. The optimal mold population
            is represented by a solid line, and the mold population without
            control is the dashed line. The optimal increases at the of the
            interval, but is held much lower overall than if no fungcide was
            used. The parameters are
      		$r = \num{0.3}$, 
      		$M = \num{10}$, 
      		$A = \num{10}$, 
      		$x_0 = \num{1}$, 
      		$T =\num{5}$.
    	}\label{Figure_MF_3}
	\end{center}
\end{figure}

Now, if we vary the carrying capacity M, the mold would increase more rapidly. 
We need to balance this effect with the control and state.
Lets try the next values $r = \num{0.6}$, $M = \num{5}$, $A = \num{10}$, 
$x_0 = \num{5}$, $T =\num{5}$.

\begin{figure}[htb] 
	\begin{center}
    	\includegraphics[width=.65\textwidth,keepaspectratio]%
    	{Chapters/Chapter2/Figures/MF_state_control_adjoint_3}
		\caption{%
        	The horizontal axis represents time $t$. The vertical axis
            represents, in each case, the state $x$, the control $u$ and the
            solution of the adjoint equation $\lambda$. The parameters are
      		$r = \num{0.6}$, 
            $M = \num{5}$, 
            $A = \num{10}$, 
            $x_0 = \num{5}$, 
            $T =\num{5}$.
    	}\label{Figure_MF_4}
	\end{center}
\end{figure}

\newpage
