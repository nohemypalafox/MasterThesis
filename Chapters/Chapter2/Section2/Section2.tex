\section{Bacteria}	%\label{Chap2-Sect2}
	
%---------%---------%---------%---------%---------%---------%---------%---------
%---------%---------%---------%---------%---------%---------%---------%---------

Consider the maximization problem
\begin{align*}
	&\max_{u} \ Cx(1) - \int_{0}^{1} u^2(t) dt \\
    \text{subject to } \ x'(t) &= rx(t) + Au(t)x(t) - Bu^2(t)\exp{-x(t)},% 
    \ x(0) = x_0,
\end{align*}
with $A, B, C \geq 0$. In this problem, $x$ represents the bacteria
concentration at time $t$, $u$ is the amount of the chemical being added at time
$t$. The parameter $A$ is the relative strength of the chemical nutrient
increasing growth, $B$ is the strength of the byproduct and $r$ is the growth 
rate. The first thing to note, is that in this example we have a payoff term. 
Here, $\phi (x) = Cx$ and $\phi ' = C$. This means, the adjoint is not zero at
the end of the interval, instead $\lambda (1) = C$. 
\todo{why $\lambda > 0$}

CONSULTAR CON SAUL ESTA PARTE: 
Now, see the Figure.  As $x$ becomes larger the $e^{-x}$ term decreases, and
the byproduct has less of a hindering effect. Consequently, the level of chemical ($u$) added starts fairly low and steadily increases, with noticeably higher rates of increase around $t = 0.6$ and $t = 0.8$. As such, the bacteria growth is approximately exponential early in the time interval, but begins to increase more and more
rapidly.

\begin{figure}[htb] 
	\begin{center}
    	\includegraphics[width=.65\textwidth,keepaspectratio]%
        {Chapters/Chapter2/Figures/Bacteria_control_uncontrol}
		\caption{%
        	The horizontal axis represents time $t$. The vertical axis
            represents, in each case, the state $x$, the control $u$ and the
            solution of the adjoint equation $\lambda$. The parameters are
      		$r = \num{1}$, 
      		$A = \num{1}$,
            $B = \num{12}$,
            $C = \num{1}$,
      		$x_0 = \num{1}$, 
    	}\label{Figure_BG_1}
	\end{center}
\end{figure}

\begin{figure}[htb] 
	\begin{center}
    	\includegraphics[width=.65\textwidth,keepaspectratio]%
        {Chapters/Chapter2/Figures/B_state_control_adjoint}
		\caption{%
        	The horizontal axis represents time $t$. The vertical axis
            represents the bacteria concentration ($x(t)$). The parameters are
      		$r = \num{1}$, 
      		$A = \num{1}$,
            $B = \num{12}$,
            $C = \num{1}$,
      		$x_0 = \num{1}$, 
    	}\label{Figure_BG_2}
	\end{center}
\end{figure}
Now, consider a small initial population. We can see on Figure 
\ref{Figure_BG_3} that the chemical used is very small.
\begin{figure}[htb] 
	\begin{center}
    	\includegraphics[width=.65\textwidth,keepaspectratio]%
        {Chapters/Chapter2/Figures/Bacteria_control_state}
		\caption{%
        	The horizontal axis represents time $t$. The vertical axis
            represents, in each case, the state $x$ and the control $u$. The
            parameters are
      		$r = \num{1}$, 
      		$A = \num{1}$,
            $B = \num{12}$,
            $C = \num{1}$,
      		$x_0 = \num{0.1}$, 
    	}\label{Figure_BG_3}
	\end{center}
\end{figure}

\newpage 



