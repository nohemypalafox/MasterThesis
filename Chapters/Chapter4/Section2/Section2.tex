%---------%---------%---------%---------%---------%---------%---------%---------
%---------                       Chapter 4, Section 2                ----------%
%---------%---------%---------%---------%---------%---------%---------%---------
\section{MDR-TB with control}  \label{Chap4-Sect2} 

The model for the two-strain tuberculosis with control $u_1$ and $u_2$
from {\citep{articleLenhart}}

\begin{align*}
	\dot{S}	&= 
    	\Lambda -\beta_{1} S \frac{I_1}{N} - \beta_{3}S\frac{I_2}{N} - 
        \mu S 
        \\
    \dot{L}_1 &= 
    	\beta_{1}S \frac{I_1}{N} - (\mu + k_1 + u_1(t)r_1)L_{1} + 
        (1 - u_2(t))pr_2 I_1 + \beta_{2} T \frac{I_1}{N} -
        \beta_{3} L_{1}\frac{I_2}{N}  
        \\
    \dot{I}_1 &=  
    	k_1 L_{1} - (\mu + d_1 + r_2)I_1  
        \\
    \dot{L}_2 &=  
    	(1 - u_2(t))q r_2 I_1 - (\mu + k_2)L_2 + \beta_{3}(S + L_1 + 
        T)\frac{I_2}{N}  
        \\
    \dot{I}_2 &=  
    	k_2 L_2 - (\mu + d_2)I_2  
        \\
    \dot{T} &=  
    	u_1(t)r_1 L_{1} + (1-(1-u_2(t))(p+q))r_2 I_1 - \beta_{2} T \frac{I_1}{N} - 
        \beta_{3}T\frac{I_2}{N}  
\end{align*}


The objective functional to be minimized is 
$$
    J(u_1(t),u_2(t)) = \int_{0}^{t_f} %
        \left[% 
            L_2(t) + I_2(t) + \dfrac{B_1}{2} u_{1}^{2}(t) + \dfrac{B_2}{2} u_{2}^{2}(t)%
        \right] dt
$$
Our goal is to find an optimal control pair $u_{1}^{*}$ and $u_{2}^{*}$ such that
$$
    J(u_{1}^{*},u_{2}^{*}) = min_{\Omega}J(u_1,u_2)
$$
where $\Omega = \{(u_1,u_2) \in L^{1}(0, t_{f}) | a_i \leq u_i \leq b_i, \}$ and
$a_i , b_i$ are fixed positive constants. 

The Hamiltonian is
$$
    H = L_2 + I_2 + \dfrac{B_1}{2} u_{1}^{2} + \dfrac{B_2}{2} u_{2}^{2} +
        \sum_{i=1}^{6} \lambda_i g_i
$$
where $g_i$ is the right hand side of the differential equation of the $i$th
state variable.

\begin{theorem}
    There exists an optimal control pair $u_{1}^{*}$ and $u_{2}^{*}$ and 
    corresponding solution, $S^{*}, L_{1}^{*}, I_{1}^{*}, L_{2}^{*}, I_{2}^{*}$
    and $T^{*}$, that minimizes $J(u_1,u_2)$ over $\Omega$. Moreover, there exists
    adjoint functions, $\lambda_{1}(t), \ldots, \lambda_{6}(t)$ such that
    \begin{align*}
	    \dot{\lambda}_1	&= 
	        \lambda_{1} \left(\beta_{1}\frac{I_1}{N} + \beta_{3}\frac{I_2}{N} 
	            + \mu \right)
	       -\lambda_{2} \beta_{1}\frac{I_1}{N}
	       -\lambda_{4} \beta_{3}\frac{I_2}{N}
        \\
        \dot{\lambda}_2	&= 
	        \lambda_{2} \left( \mu + k_1 + u_1 r_1 + \beta_{3}\frac{I_2}{N}         \right)
	       -\lambda_{3} k_1
	       -\lambda_{4} \beta_{3}\frac{I_2}{N}
	       -\lambda_{6} \left( u_1 r_1 \right)
        \\
        \dot{\lambda}_3	&= 
	        \lambda_{1} \beta_{1}\frac{S}{N}
	       -\lambda_{2} \left( \beta_{1}\frac{S}{N} + (1 - u_2)pr_2 +
	            \beta_{2}\frac{T}{N} \right)
	       +\lambda_{3} \left( \mu + d_1 + r_2 \right)
	       -\lambda_{4} \left(1 - u_2\right)qr_2  \\
	       &\quad -\lambda_{6} \left((1 - ( 1- u_2)(p+q))r_2 - \beta_{2}\frac{T}{N}
	            \right)
	   \\
	   \dot{\lambda}_4	&= -1
	       +\lambda_{4} \left( \mu + k_2 \right)
	       -\lambda_{5} k_2
        \\
        \dot{\lambda}_5	&= -1
	       +\lambda_{1} \beta_{3}\frac{S}{N} 
	       +\lambda_{2} \beta_{3}\frac{L_1}{N}
	       -\lambda_{4} \beta_{3}\frac{S + L_1 + T}{N}
	       +\lambda_{5} \left( \mu + d_2 \right)
	       +\lambda_{6} \beta_{3}\frac{T}{N}
        \\
        \dot{\lambda}_6	&= 
	       -\lambda_{2} \beta_{2}\frac{I_1}{N}
	       -\lambda_{4} \beta_{3}\frac{I_2}{N}
	       -\lambda_{6} \left( \beta_{2}\frac{I_1}{N} + \beta_{3}\frac{I_2}{N}
	            + \mu \right).
    \end{align*}
\end{theorem}

%---------%---------%---------%---------%---------%---------%---------%---------
\begin{table}
	\centering
	\begin{tabular}{ll}
		\toprule
			\textbf{Parameters} & \textbf{Values}
            \\
        \midrule
            $\beta_1$ & 13
            \\
        	$\beta_2$ & 13
			\\
            $\beta_3$ & 0.0131, 0.0217, 0.029, 0.0436
       		\\
     	    $\mu$ & 0.0143
			\\
     	    $d_1$ & 0
     	    \\
     	    $d_2$ & 0
     	    \\
     	    $k_1$ & 0.5
 			\\
     	    $k_2$ & 1
			\\
     	    $r_1$ & 2
     	    \\
     	    $r_2$ & 1
     	    \\
     	    $p$ & 0.4
     	    \\
     	    $q$ & 0.1
			\\
			$N$ & 6000, 12000, 30000
            \\
			$\Lambda$ & $\mu N$
            \\
            $t_f$ & $5$ years
            \\
     	    $B_1$ & $50$
			\\
     	    $B_2$ & $500$
     	    \\
     	    Lower bound for controls & $0.05$
			\\
            Upper bound for controls & $0.95$
       		\\
		\bottomrule
    \end{tabular}
	\caption{Values of the parameters}
\end{table}


\begin{table}
	\centering
	\begin{tabular}{ll}
		\toprule
			\textbf{States} & \textbf{Values}
            \\
        \midrule
            $S(0)$ & $(76/120)N$
            \\
        	$L_1(0)$ & $(36/120)N$
			\\
            $I_1(0)$ & $(4/120)N$
       		\\
     	    $L_2(0)$ & $(2/120)N$
			\\
     	    $I_2(0)$ & $(1/120)N$
     	    \\
     	    $T(0)$ & $(1/120)N$
     	    \\
		\bottomrule
    \end{tabular}
	\caption{Initial conditions}
\end{table}
%---------%---------%---------%---------%---------%---------%---------%---------
\begin{figure}[htb] 
	\begin{center}
    	\includegraphics[width=\textwidth,keepaspectratio]%
    	{Chapters/Chapter4/Figures/figure_1_two_strain_tbm}
		\caption{%
        	In the left side, the green line represents the uncotrolled state of
        	MDR-TB infected population and the orange dashed line represents the 
        	controlled state. In the right side, the two controls are plotted
            The parameters that were used are:
    	}\label{Figure_TBM_1}
	\end{center}
\end{figure}
%---------%---------%---------%---------%---------%---------%---------%---------
\begin{figure}[htb] 
	\begin{center}
    	\includegraphics[width=\textwidth,keepaspectratio]%
    	{Chapters/Chapter4/Figures/figure_2_two_strain_tbm}
		\caption{%
        	In the left side, the green line represents the uncotrolled state of
        	MDR-TB infected population and the orange dashed line represents the 
        	controlled state. In the right side, the two controls are plotted
            The parameters that were used are:
    	}\label{Figure_TBM_2}
	\end{center}
\end{figure}
%---------%---------%---------%---------%---------%---------%---------%---------
\begin{figure}[htb] 
	\begin{center}
    	\includegraphics[width=\textwidth,keepaspectratio]%
    	{Chapters/Chapter4/Figures/figure_3_two_strain_tbm}
		\caption{%
        	The horizontal axis represents time $t$. The vertical axis
            represents, in each case, the state $x$, the control $u$ and the
            solution of the adjoint equation $\lambda$. The parameters are
      		$r = \num{0.3}$, 
      		$M = \num{10}$, 
      		$A = \num{10}$, 
      		$x_0 = \num{1}$, 
      		$T =\num{5}$.
    	}\label{Figure_TBM_3}
	\end{center}
\end{figure}
%---------%---------%---------%---------%---------%---------%---------%---------

\newpage