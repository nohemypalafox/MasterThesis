%---------%---------%---------%---------%---------%---------%---------%---------
%---------                       Chapter 4, Section 1                ----------%
%---------%---------%---------%---------%---------%---------%---------%---------
\section{MDR-TB without control}  \label{Chap4-Sect1} 

A two-strain TB Model based on \cite{Castillo-Chavez1997}:

The population is divided into six compartments
\begin{align*}
	\dot{S}	&= 
    	\Lambda -\beta_{1} S \frac{I_1}{N} - \beta_{3}S\frac{I_2}{N} - 
        \mu S 
        \\
    \dot{L}_1 &= 
    	\beta_{1}S \frac{I_1}{N} - (\mu + k_1 + r_1)L_{1} + 
        pr_2 I_1 + \beta_{2} T \frac{I_1}{N} -
        \beta_{3} L_{1}\frac{I_2}{N}  
        \\
    \dot{I}_1 &=  
    	k_1 L_{1} - (\mu + d_1 + r_2)I_1  
        \\
    \dot{L}_2 &=  
    	q r_2 I_1 - (\mu + k_2)L_2 + \beta_{3}(S + L_1 + 
        T)\frac{I_2}{N}  
        \\
    \dot{I}_2 &=  
    	k_2 L_2 - (\mu + d_2)I_2  
        \\
    \dot{T} &=  
    	r_1 L_{1} + (1-(p+q))r_2 I_1 - \beta_{2} T \frac{I_1}{N} - 
        \beta_{3}T\frac{I_2}{N}  
\end{align*}

%\todo{corregir: describr las clases por ejemplo que significa $J$}
\begin{table}\label{tbl-ModRes}
	\centering
	\begin{tabular}{ll}
		\toprule
			\textbf{Parameters} & \textbf{Description}
        \\
        \midrule
            $\Lambda$ & Recruitment rate
            \\
            $\beta_1$ & Probability that a susceptible individual become infected by one
            \\
                      & infectious individual per contact per unit of time.
            \\
        	$\beta_2$ & Probability that a recovered individual become infected by one
            \\
                      & infectious individual per contact per unit of time.
            \\
            $\beta_3$ & Probability that uninfected individuals become infected by one
            \\
                      & resistant-TB infectious individual per contact per unit of time.
            \\
     	    $k_1$ & Rate at which an individual leaves the latent class of TB
 			\\
			    & by becoming infectious. 
            \\
     	    $k_2$ & Rate at which an individual leaves the latent class of MDR-TB
 			\\
			    & by becoming infectious.
            \\
			$\mu$ & Per-capita natural death rate.
			\\
     	    $d_1$ & Per-capita disease induced death rate for TB.	 
     	    \\
     	    $d_2$ & Per-capita disease induced death rate for MDR-TB.
     	    \\
     	    $r_1$ & Treatment rate of individuals with latent TB.
     	    \\
     	    $r_2$ & Treatment rate of individuals with infectious TB.
     	    \\
     	    $p+q$ & Proportion of treated infectious individuals that did 
     	    \\
     	        & not complete their treatment.	
			\\
		\bottomrule
	\end{tabular}
	\caption{Description of parameters for the MDR-TB model}
	\label{tbl-ModRes}
\end{table}

We can find the disease-free equilibrium 
$x_0 = (S^*, L_{1}^{*}, I_{1}^{*}, L_{2}^{*}, I_{2}^{*}, T^{*})$.
The disease-free equilibrium is 
$$
    x_0 = \left(\frac{\lambda}{\mu}, 0 , 0 ,0, 0, 0\right)
$$
In the next part we calculate the basic reproduction number $\mathscr{R}_{0}$. 
First we have to reorder the equations in such a way that the first equations
are the infectious ones.
\begin{align*}
    \dot{L}_1 &= 
    	\beta_{1}S \frac{I_1}{N} - (\mu + k_1 + r_1)L_{1} + 
        pr_2 I_1 + \beta_{2} T \frac{I_1}{N} -
        \beta_{3} L_{1}\frac{I_2}{N}  
        \\
    \dot{I}_1 &=  
    	k_1 L_{1} - (\mu + d_1 + r_2)I_1  
        \\
    \dot{L}_2 &=  
    	q r_2 I_1 - (\mu + k_2)L_2 + \beta_{3}(S + L_1 + 
        T)\frac{I_2}{N}  
        \\
    \dot{I}_2 &=  
    	k_2 L_2 - (\mu + d_2)I_2  
        \\
    \dot{S}	&= 
    	\Lambda -\beta_{1} S \frac{I_1}{N} - \beta_{3}S\frac{I_2}{N} - 
        \mu S 
        \\
    \dot{T} &=  
    	r_1 L_{1} + (1-(p+q))r_2 I_1 - \beta_{2} T \frac{I_1}{N} - 
        \beta_{3}T\frac{I_2}{N}  
\end{align*}

For this model, the progression from $L_1$ to $I_1$, and $L_2$ to $I_2$ are 
not considered to be new infections, the only terms that represent new infections
are $ \beta_{1} S \frac{I_1}{N}$, $\beta_{2} T \frac{I_1}{N}$,  $(p + q )r_2 I_1$, and
$\beta_{3} (S + L_{1} + T)\frac{I_2}{N} $. Hence, the functions $\mathscr{F}$ and 
$\mathscr{V}$ are
\begin{equation*}
    \mathscr{F} = 
        \begin{pmatrix}
            \beta_{1} S \frac{I_1}{N} + \beta_{2} T \frac{I_1}{N} + p r_2 I_1 \\
            \beta_{3} (S + L_{1} + T)\frac{I_2}{N} + q r_2 I_1 \\
            0 \\
            0 \\
            0 \\
            0
        \end{pmatrix},
\end{equation*}

\begin{equation*}
    \mathscr{V} = 
        \begin{pmatrix}
            (\mu + k_1 + r_1)L_{1} +  \beta_{3} L_{1}\frac{I_2}{N} \\
            (\mu + k_2)L_2 \\
            - k_1 L_{1} + (\mu + d_1 + r_2)I_1 \\
            - k_2 L_2 + (\mu + d_2)I_2 \\
            - \Lambda + \beta_{1} S \frac{I_1}{N} + \beta_{3}S\frac{I_2}{N} + \mu S \\
            - r_1 L_{1} + (p+q - 1)r_2 I_1 + \beta_{2} T \frac{I_1}{N} + \beta_{3}T\frac{I_2}{N}  
        \end{pmatrix}.
\end{equation*}

\begin{equation*}
    F(x) = 
        \begin{pmatrix}
            0 & 0 & \beta_{1} S \frac{1}{N} + \beta_{2} T \frac{1}{N} + p r_2 & 0 \\
            \beta_{3}\frac{I_2}{N} & 0 &   q r_2  &   \beta_{3} (S + L_{1} + T)\frac{1}{N} \\
    		0 & 0 & 0 & 0 \\
		    0 & 0 & 0 & 0
        \end{pmatrix},
\end{equation*}
\begin{equation*}
    V(x) =
        \begin{pmatrix}
            \mu + k_1 + r_1 +  \beta_{3} \frac{I_2}{N} & 0 & 0 & \beta_{3} \frac{L_1}{N} \\
            0 & \mu + k_2 & 0 & 0 \\
            -k_1 & 0 & \mu + d_1 + r_2 & 0 \\
            0 & 0 & -k_2 & \mu + d_2 
        \end{pmatrix}
\end{equation*}

Evaluate the disease-free equilibrium $x_0 = \left(0, 0, 0, 0, N, 0 \right)$ 
\begin{equation*}
    F(x) = 
        \begin{pmatrix}
            0 & 0 & \beta_{1} + p r_2 & 0 \\
            0 & 0 &   q r_2  &   \beta_{3} \\
    		0 & 0 & 0 & 0 \\
		    0 & 0 & 0 & 0
        \end{pmatrix},
\end{equation*}
\begin{equation*}
    V(x) =
        \begin{pmatrix}
            \mu + k_1 + r_1 & 0 & 0 & 0 \\
            0 & \mu + k_2 & 0 & 0 \\
            -k_1 & 0 & \mu + d_1 + r_2 & 0 \\
            0 & 0 & -k_2 & \mu + d_2
        \end{pmatrix}
\end{equation*}
Calculate the next generation matrix on $x_0$
\begin{align*}	
	FV^{-1}(x_0) 
	&=  
    \begin{pmatrix}
        0 & 0 & \beta_{1} + p r_2 & 0 \\
        0 & 0 &   q r_1  &   \beta_{3} \\
		0 & 0 & 0 & 0 \\
	    0 & 0 & 0 & 0
    \end{pmatrix}
	\begin{pmatrix}
		\frac{1}{\mu + k_1 + r_1} & 0 & 0 & 0 \\
		0 & \frac{1}{\mu + k_2} & 0 & 0 \\
		\frac{k_1}{(\mu + k_1 + r_1)(\mu + d_1 +r_2)} & 0 & \frac{1}{\mu + d_1 +r_2} & 0 \\
		0 & \frac{k_2}{(\mu + k_2)(\mu + d_2)} & 0 & \frac{1}{\mu + d_2} \\
	\end{pmatrix}  \\
	&= 
	\begin{pmatrix}
		\dfrac{k_1(\beta_1 + pr_2)}{(\mu + k_1 + r_1)(\mu + d_1 +r_2)} 
        & 0 & \dfrac{\beta_1 + pr_2}{\mu + d_1 +r_2} & 0 
    \\
		\dfrac{k_1 q r_2}{(\mu + k_1 + r_1)(\mu + d_1 +r_2)} 
			& \dfrac{k_2 \beta_3}{(\mu + k_2)(\mu + d_2)} 
      & \dfrac{qr_2}{\mu + d_1 +r_2} & \dfrac{\beta_3}{\mu + d_2} 
    \\
		0 & 0 & 0 & 0 
		\\
		0 & 0 & 0 & 0 
		\\
	\end{pmatrix}.
\end{align*}
Define $K = FV^{-1}(x_0)$, then
\begin{align*}	
	\det(K-\lambda I_d)
	&= 
	\begin{vmatrix}
		\dfrac{k_1(\beta_1 + pr_2)}{(\mu + k_1 + r_1)(\mu + d_1 +r_2)} 
	    -\lambda 
	    & 0 
	    & \dfrac{\beta_1 + pr_2}{\mu + d_1 +r_2} 
	    & 0 
	  \\
		\dfrac{k_1 qr_2}{(\mu + k_1 + r_1)(\mu + d_1 +r_2)} 
		& \dfrac{k_2 \beta_3}{(\mu + k_2)(\mu + d_2)} -\lambda 
		& \dfrac{qr_2}{\mu + d_1 +r_2} 
		& \dfrac{\beta_3}{\mu + d_2} 
		\\
		0 & 0 & -\lambda & 0 
		\\
		0 & 0 & 0 & -\lambda 
		\\
	\end{vmatrix} \notag \\
	&= -\lambda
	\begin{vmatrix}
		\dfrac{k_1(\beta_1 + pr_2)}{(\mu + k_1 + r_1)(\mu + d_1 +r_2)} 
			-\lambda & 0 & 0 
		\\
		\dfrac{k_1 q r_2}{(\mu + k_1 + r_1)(\mu + d_1 +r_2)} 
			& \dfrac{k_2 \beta_3}{(\mu + k_2)(\mu + d_2)}
        -\lambda &\dfrac{\beta_3}{\mu + d_2} 
    \\
		0 & 0 & -\lambda 
	\end{vmatrix}
	\\
	&= \lambda^2
	\begin{vmatrix}
		\dfrac{k_1(\beta_1 + pr_2)}{(\mu + k_1 + r_1)(\mu + d_1 +r_2)} -\lambda 
			& 0 
		\\
		\dfrac{k_1 q r_2}{(\mu + k_1 + r_1)(\mu + d_1 +r_2)} 
			& \dfrac{k_2 \beta_3}{(\mu + k_2)(\mu + d_2)} 
       -\lambda 
    \\
	\end{vmatrix} 
	\\
		&= 
		\lambda^2 
	  \left( 
		  \dfrac{k_1(\beta_1 + pr_2)}{(\mu + k_1 + r_1)(\mu + d_1 +r_2)} 
		  -\lambda 
		\right) 
    \left(
	    \dfrac{k_2 \beta_3}{(\mu + k_2)(\mu + d_2)}-\lambda 
	  \right).
\end{align*} 
%---------%---------%---------%---------%---------%---------%---------%---------
%---------%---------%---------%---------%---------%---------%---------%---------
By the above,
$$
	\mathscr{R}_{1} = 
		\dfrac{k_1(\beta_{1} + pr_2)}{(\mu + k_1 + r_1)(\mu + d_1 +r_2)} ,
		\qquad
	\mathscr{R}_{2} = 
		\dfrac{k_2 	\beta_{3}}{(\mu + k_2)(\mu + d_2)}.
$$
The basic reproduction number is 
$
	\mathscr{R}_{0} =  \max\lbrace \mathscr{R}_{1},\mathscr{R}_{2} 
\rbrace
$.
 \newpage
% un equilibrio libre de enfermedad, tres endemicos, donde uno de ellos es limite%
 We have two cases, when $q=0$ and $q>0$
 \begin{theorem}
    (Estability theorem for $q=0$)
 \end{theorem}
  \begin{theorem}
    (Estability theorem for $q>0$)
 \end{theorem}