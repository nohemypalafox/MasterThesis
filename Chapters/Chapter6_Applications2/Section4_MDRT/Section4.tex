%CHAPTER 4
\section{Multidrug-resistant Tuberculosis} % Main chapter title \label{Chap1} 
\lhead{\emph{Multidrug-resistant Tuberculosis (MDR-TB)}} 
%---------%---------%---------%---------%---------%---------%---------%---------

%---------%---------%---------%---------%---------%---------%---------%---------
%---------                           MDR-TB                          ----------%
%---------%---------%---------%---------%---------%---------%---------%---------
    The Tuberculosis (TB) is a disease caused by a bacteria called \textit{ Mycobacterium tuberculosis} affecting, principally, the lungs. This disease is one of the top 10 causes of death worldwide and is a leading killer of HIV-positive people. Fortunately, it is a treatable and curable disease. But, failures in treatment, inappropriate use of medicines can cause a drug resistance to TB. So, the Multidrug-resistant tuberculosis (MDR-TB) is a form of TB caused by bacteria that do not respond to isoniazid and rifampicin, the 2 most powerful anti-TB drugs. The MDR-TB is still curable, but is expensive and requires chemotherapy.
    The following is MDR-TB model based on \cite{Castillo-Chavez1997} without 
    control
    \begin{align*}
    	\dot{S}	&= 
        	\Lambda -\beta_{1} S \frac{I_1}{N} - \beta_{3}S\frac{I_2}{N} - 
            \mu S 
            \\
        \dot{L}_1 &= 
        	\beta_{1}S \frac{I_1}{N} - (\mu + k_1 + r_1)L_{1} + 
            pr_2 I_1 + \beta_{2} T \frac{I_1}{N} -
            \beta_{3} L_{1}\frac{I_2}{N}  
            \\
        \dot{I}_1 &=  
        	k_1 L_{1} - (\mu + d_1 + r_2)I_1  
            \\
        \dot{L}_2 &=  
        	q r_2 I_1 - (\mu + k_2)L_2 + \beta_{3}(S + L_1 + 
            T)\frac{I_2}{N}  
            \\
        \dot{I}_2 &=  
        	k_2 L_2 - (\mu + d_2)I_2  
            \\
        \dot{T} &=  
        	r_1 L_{1} + (1-(p+q))r_2 I_1 - \beta_{2} T \frac{I_1}{N} - 
            \beta_{3}T\frac{I_2}{N}  
    \end{align*}
    
    %\todo{corregir: describr las clases por ejemplo que significa $J$}
\begin{table}\label{tbl-ModRes}
	\centering
	\begin{tabular}{ll}
		\toprule
			\textbf{Parameters} & \textbf{Description}
        \\
        \midrule
            $\Lambda$ & Recruitment rate
            \\
            $\beta_1$ & Probability that a susceptible individual become infected by one
            \\
                      & infectious individual per contact per unit of time.
            \\
        	$\beta_2$ & Probability that a recovered individual become infected by one
            \\
                      & infectious individual per contact per unit of time.
            \\
            $\beta_3$ & Probability that uninfected individuals become infected by one
            \\
                      & resistant-TB infectious individual per contact per unit of time.
            \\
     	    $k_1$ & Rate at which an individual leaves the latent class of TB
 			\\
			    & by becoming infectious. 
            \\
     	    $k_2$ & Rate at which an individual leaves the latent class of MDR-TB
 			\\
			    & by becoming infectious.
            \\
			$\mu$ & Per-capita natural death rate.
			\\
     	    $d_1$ & Per-capita disease induced death rate for TB.	 
     	    \\
     	    $d_2$ & Per-capita disease induced death rate for MDR-TB.
     	    \\
     	    $r_1$ & Treatment rate of individuals with latent TB.
     	    \\
     	    $r_2$ & Treatment rate of individuals with infectious TB.
     	    \\
     	    $p+q$ & Proportion of treated infectious individuals that did 
     	    \\
     	        & not complete their treatment.	
			\\
		\bottomrule
	\end{tabular}
	\caption{Description of parameters for the MDR-TB model}
	\label{tbl-ModRes}
\end{table}
    The disease-free equilibrium is 
    $$
        x_0 = \left(\frac{\lambda}{\mu}, 0 , 0 ,0, 0, 0\right)
    $$
    In the next part we calculate the basic reproduction number $\mathscr{R}_{0}$. 
    First we have to reorder the equations in such a way the first equations
    are the infectious ones.
    \begin{align*}
        \dot{L}_1 &= 
        	\beta_{1}S \frac{I_1}{N} - (\mu + k_1 + r_1)L_{1} + 
            pr_2 I_1 + \beta_{2} T \frac{I_1}{N} -
            \beta_{3} L_{1}\frac{I_2}{N}  
            \\
        \dot{I}_1 &=  
        	k_1 L_{1} - (\mu + d_1 + r_2)I_1  
            \\
        \dot{L}_2 &=  
        	q r_2 I_1 - (\mu + k_2)L_2 + \beta_{3}(S + L_1 + 
            T)\frac{I_2}{N}  
            \\
        \dot{I}_2 &=  
        	k_2 L_2 - (\mu + d_2)I_2  
            \\
        \dot{S}	&= 
        	\Lambda -\beta_{1} S \frac{I_1}{N} - \beta_{3}S\frac{I_2}{N} - 
            \mu S 
            \\
        \dot{T} &=  
        	r_1 L_{1} + (1-(p+q))r_2 I_1 - \beta_{2} T \frac{I_1}{N} - 
            \beta_{3}T\frac{I_2}{N}  
    \end{align*}
    
    \noindent For this model, the progression from $L_1$ to $I_1$, and $L_2$ 
    to $I_2$ are 
    not considered to be new infections, the only terms that represent new infections
    are $ \beta_{1} S \frac{I_1}{N}$, $\beta_{2} T \frac{I_1}{N}$,  $(p + q )r_2 I_1$, and
    $\beta_{3} (S + L_{1} + T)\frac{I_2}{N} $. Hence, the functions $\mathscr{F}$ and 
    $\mathscr{V}$ are
    \begin{equation*}
        \mathscr{F} = 
            \begin{pmatrix}
                \beta_{1} S \frac{I_1}{N} + \beta_{2} T \frac{I_1}{N} + p r_2 I_1 \\
                \beta_{3} (S + L_{1} + T)\frac{I_2}{N} + q r_2 I_1 \\
                0 \\
                0 \\
                0 \\
                0
            \end{pmatrix},
    \end{equation*}
    
    \begin{equation*}
        \mathscr{V} = 
            \begin{pmatrix}
                (\mu + k_1 + r_1)L_{1} +  \beta_{3} L_{1}\frac{I_2}{N} \\
                (\mu + k_2)L_2 \\
                - k_1 L_{1} + (\mu + d_1 + r_2)I_1 \\
                - k_2 L_2 + (\mu + d_2)I_2 \\
                - \Lambda + \beta_{1} S \frac{I_1}{N} + \beta_{3}S\frac{I_2}{N} + \mu S \\
                - r_1 L_{1} + (p+q - 1)r_2 I_1 + \beta_{2} T \frac{I_1}{N} + \beta_{3}T\frac{I_2}{N}  
            \end{pmatrix}.
    \end{equation*}
    
    \begin{equation*}
        F(x) = 
            \begin{pmatrix}
                0 & 0 & \beta_{1} S \frac{1}{N} + \beta_{2} T \frac{1}{N} + p r_2 & 0 \\
                \beta_{3}\frac{I_2}{N} & 0 &   q r_2  &   \beta_{3} (S + L_{1} + T)\frac{1}{N} \\
        		0 & 0 & 0 & 0 \\
    		    0 & 0 & 0 & 0
            \end{pmatrix},
    \end{equation*}
    \begin{equation*}
        V(x) =
            \begin{pmatrix}
                \mu + k_1 + r_1 +  \beta_{3} \frac{I_2}{N} & 0 & 0 & \beta_{3} \frac{L_1}{N} \\
                0 & \mu + k_2 & 0 & 0 \\
                -k_1 & 0 & \mu + d_1 + r_2 & 0 \\
                0 & 0 & -k_2 & \mu + d_2 
            \end{pmatrix}
    \end{equation*}
    
    Evaluate the disease-free equilibrium $x_0 = \left(0, 0, 0, 0, N, 0 \right)$ 
    \begin{equation*}
        F(x) = 
            \begin{pmatrix}
                0 & 0 & \beta_{1} + p r_2 & 0 \\
                0 & 0 &   q r_2  &   \beta_{3} \\
        		0 & 0 & 0 & 0 \\
    		    0 & 0 & 0 & 0
            \end{pmatrix},
    \end{equation*}
    \begin{equation*}
        V(x) =
            \begin{pmatrix}
                \mu + k_1 + r_1 & 0 & 0 & 0 \\
                0 & \mu + k_2 & 0 & 0 \\
                -k_1 & 0 & \mu + d_1 + r_2 & 0 \\
                0 & 0 & -k_2 & \mu + d_2
            \end{pmatrix}
    \end{equation*}
    Calculate the next generation matrix on $x_0$
    \begin{align*}	
    	FV^{-1}(x_0) 
    	&=  
        \begin{pmatrix}
            0 & 0 & \beta_{1} + p r_2 & 0 \\
            0 & 0 &   q r_1  &   \beta_{3} \\
    		0 & 0 & 0 & 0 \\
    	    0 & 0 & 0 & 0
        \end{pmatrix}
    	\begin{pmatrix}
    		\frac{1}{\mu + k_1 + r_1} & 0 & 0 & 0 \\
    		0 & \frac{1}{\mu + k_2} & 0 & 0 \\
    		\frac{k_1}{(\mu + k_1 + r_1)(\mu + d_1 +r_2)} & 0 & \frac{1}{\mu + d_1 +r_2} & 0 \\
    		0 & \frac{k_2}{(\mu + k_2)(\mu + d_2)} & 0 & \frac{1}{\mu + d_2} \\
    	\end{pmatrix}  \\
    	&= 
    	\begin{pmatrix}
    		\dfrac{k_1(\beta_1 + pr_2)}{(\mu + k_1 + r_1)(\mu + d_1 +r_2)} 
            & 0 & \dfrac{\beta_1 + pr_2}{\mu + d_1 +r_2} & 0 
        \\
    		\dfrac{k_1 q r_2}{(\mu + k_1 + r_1)(\mu + d_1 +r_2)} 
    			& \dfrac{k_2 \beta_3}{(\mu + k_2)(\mu + d_2)} 
          & \dfrac{qr_2}{\mu + d_1 +r_2} & \dfrac{\beta_3}{\mu + d_2} 
        \\
    		0 & 0 & 0 & 0 
    		\\
    		0 & 0 & 0 & 0 
    		\\
    	\end{pmatrix}.
    \end{align*}
    Define $K = FV^{-1}(x_0)$, then
    \begin{align*}	
    	\det(K-\lambda I_d)
    	&= 
    	\begin{vmatrix}
    		\dfrac{k_1(\beta_1 + pr_2)}{(\mu + k_1 + r_1)(\mu + d_1 +r_2)} 
    	    -\lambda 
    	    & 0 
    	    & \dfrac{\beta_1 + pr_2}{\mu + d_1 +r_2} 
    	    & 0 
    	  \\
    		\dfrac{k_1 qr_2}{(\mu + k_1 + r_1)(\mu + d_1 +r_2)} 
    		& \dfrac{k_2 \beta_3}{(\mu + k_2)(\mu + d_2)} -\lambda 
    		& \dfrac{qr_2}{\mu + d_1 +r_2} 
    		& \dfrac{\beta_3}{\mu + d_2} 
    		\\
    		0 & 0 & -\lambda & 0 
    		\\
    		0 & 0 & 0 & -\lambda 
    		\\
    	\end{vmatrix} \notag \\
    	&= -\lambda
    	\begin{vmatrix}
    		\dfrac{k_1(\beta_1 + pr_2)}{(\mu + k_1 + r_1)(\mu + d_1 +r_2)} 
    			-\lambda & 0 & 0 
    		\\
    		\dfrac{k_1 q r_2}{(\mu + k_1 + r_1)(\mu + d_1 +r_2)} 
    			& \dfrac{k_2 \beta_3}{(\mu + k_2)(\mu + d_2)}
            -\lambda &\dfrac{\beta_3}{\mu + d_2} 
        \\
    		0 & 0 & -\lambda 
    	\end{vmatrix}
    	\\
    	&= \lambda^2
    	\begin{vmatrix}
    		\dfrac{k_1(\beta_1 + pr_2)}{(\mu + k_1 + r_1)(\mu + d_1 +r_2)} -\lambda 
    			& 0 
    		\\
    		\dfrac{k_1 q r_2}{(\mu + k_1 + r_1)(\mu + d_1 +r_2)} 
    			& \dfrac{k_2 \beta_3}{(\mu + k_2)(\mu + d_2)} 
           -\lambda 
        \\
    	\end{vmatrix} 
    	\\
    		&= 
    		\lambda^2 
    	  \left( 
    		  \dfrac{k_1(\beta_1 + pr_2)}{(\mu + k_1 + r_1)(\mu + d_1 +r_2)} 
    		  -\lambda 
    		\right) 
        \left(
    	    \dfrac{k_2 \beta_3}{(\mu + k_2)(\mu + d_2)}-\lambda 
    	  \right).
    \end{align*} 
%---------%---------%---------%---------%---------%---------%---------%---------
%---------%---------%---------%---------%---------%---------%---------%---------
    By the above,
    $$
    	\mathscr{R}_{1} = 
    		\dfrac{k_1(\beta_{1} + pr_2)}{(\mu + k_1 + r_1)(\mu + d_1 +r_2)} ,
    		\qquad
    	\mathscr{R}_{2} = 
    		\dfrac{k_2 	\beta_{3}}{(\mu + k_2)(\mu + d_2)}.
    $$
    The basic reproduction number is 
    $
    	\mathscr{R}_{0} =  \max\lbrace \mathscr{R}_{1},\mathscr{R}_{2} 
    \rbrace
    $.
    As Castillo-Chavez established in \cite{Castillo-Chavez1997}, the disease free equilibrium is stable when $\mathscr{R}_{0} < 1$ and it is unstable if $mathscr{R}_{0} > 1$
%---------%---------%---------%---------%---------%---------%---------%---------
%---------                       MDR-TB with control                 ----------%
%---------%---------%---------%---------%---------%---------%---------%---------
    \noindent We consider the same MDR-TB model presented above but with two controls $u_1$ and $u_2$. This example is from {\citep{articleLenhart}}. So we have the same compartments and parameters. The control $u_1$ (case finding) represents the fraction of typical TB latent individuals that are identified and put under treatment. The term $1 - u_2$ (case holding), represents the measures to avoid the failure of treatment.  So the controlled dynamic is 
    \begin{align*}
    	\dot{S}	&= 
        	\psi -\beta_{1} S \frac{I_1}{N} - \beta_{3}S\frac{I_2}{N} - 
            \mu S 
            \\
        \dot{L}_1 &= 
        	\beta_{1}S \frac{I_1}{N} - (\mu + k_1 + u_1(t)r_1)L_{1} + 
            (1 - u_2(t))pr_2 I_1 + \beta_{2} T \frac{I_1}{N} -
            \beta_{3} L_{1}\frac{I_2}{N}  
            \\
        \dot{I}_1 &=  
        	k_1 L_{1} - (\mu + d_1 + r_2)I_1  
            \\
        \dot{L}_2 &=  
        	(1 - u_2(t))q r_2 I_1 - (\mu + k_2)L_2 + \beta_{3}(S + L_1 + 
            T)\frac{I_2}{N}  
            \\
        \dot{I}_2 &=  
        	k_2 L_2 - (\mu + d_2)I_2  
            \\
        \dot{T} &=  
        	u_1(t)r_1 L_{1} + (1-(1-u_2(t))(p+q))r_2 I_1 - \beta_{2} T \frac{I_1}{N} - 
            \beta_{3}T\frac{I_2}{N}  
    \end{align*}
    and the objective functional to be minimized is 
    $$
        J(u_1(t),u_2(t)) = \int_{0}^{t_f} %
            \left[% 
                L_2(t) + I_2(t) + \dfrac{B_1}{2} u_{1}^{2}(t) + \dfrac{B_2}{2} u_{2}^{2}(t)%
            \right] dt
    $$
    Our goal is to find an optimal control pair $u_{1}^{*}$ and $u_{2}^{*}$ such that
    $$
        J(u_{1}^{*},u_{2}^{*}) = min_{\Omega}J(u_1,u_2)
    $$
    where $\Omega = \{(u_1,u_2) \in L^{1}(0, t_{f}) | a_i \leq u_i \leq b_i, \}$ and
    $a_i , b_i$ are fixed positive constants. 
    
    From the definition (ref) the Hamiltonian is
    $$
        H = L_2 + I_2 + \dfrac{B_1}{2} u_{1}^{2} + \dfrac{B_2}{2} u_{2}^{2} +
            \sum_{i=1}^{6} \psi_i g_i
    $$
    where $g_i$ is the right hand side of the differential equation of the $i$th
    state variable.
    
    \begin{theorem}
        There exists an optimal control pair $u_{1}^{*}$ and $u_{2}^{*}$ and 
        corresponding solution, $S^{*}, L_{1}^{*}, I_{1}^{*}, L_{2}^{*}, I_{2}^{*}$
        and $T^{*}$, that minimizes $J(u_1,u_2)$ over $\Omega$. Moreover, there exists
        adjoint functions, $\psi_{1}(t), \ldots, \psi_{6}(t)$ such that
        \begin{align*}
    	    \dot{\psi}_1	&= 
    	        \psi_{1} \left(\beta_{1}\frac{I_1}{N} + \beta_{3}\frac{I_2}{N} 
    	            + \mu \right)
    	       -\psi_{2} \beta_{1}\frac{I_1}{N}
    	       -\psi_{4} \beta_{3}\frac{I_2}{N}
            \\
            \dot{\psi}_2	&= 
    	        \psi_{2} \left( \mu + k_1 + u_1 r_1 + \beta_{3}\frac{I_2}{N}         \right)
    	       -\psi_{3} k_1
    	       -\psi_{4} \beta_{3}\frac{I_2}{N}
    	       -\psi_{6} \left( u_1 r_1 \right)
            \\
            \dot{\psi}_3	&= 
    	        \psi_{1} \beta_{1}\frac{S}{N}
    	       -\psi_{2} \left( \beta_{1}\frac{S}{N} + (1 - u_2)pr_2 +
    	            \beta_{2}\frac{T}{N} \right)
    	       +\psi_{3} \left( \mu + d_1 + r_2 \right)
    	       -\psi_{4} \left(1 - u_2\right)qr_2  \\
    	       &\quad -\psi_{6} \left((1 - ( 1- u_2)(p+q))r_2 - \beta_{2}\frac{T}{N}
    	            \right)
    	   \\
    	   \dot{\psi}_4	&= -1
    	       +\psi_{4} \left( \mu + k_2 \right)
    	       -\psi_{5} k_2
            \\
            \dot{\psi}_5	&= -1
    	       +\psi_{1} \beta_{3}\frac{S}{N} 
    	       +\psi_{2} \beta_{3}\frac{L_1}{N}
    	       -\psi_{4} \beta_{3}\frac{S + L_1 + T}{N}
    	       +\psi_{5} \left( \mu + d_2 \right)
    	       +\psi_{6} \beta_{3}\frac{T}{N}
            \\
            \dot{\psi}_6	&= 
    	       -\psi_{2} \beta_{2}\frac{I_1}{N}
    	       -\psi_{4} \beta_{3}\frac{I_2}{N}
    	       -\psi_{6} \left( \beta_{2}\frac{I_1}{N} + \beta_{3}\frac{I_2}{N}
    	            + \mu \right).
        \end{align*}
    \end{theorem}
    
    %---------%---------%---------%---------%---------%---------%---------%---------
    \begin{table}
	\centering
	\begin{tabular}{ll}
		\toprule
			\textbf{Parameters} & \textbf{Values}
            \\
        \midrule
            $\beta_1$ & 13
            \\
        	$\beta_2$ & 13
			\\
            $\beta_3$ & 0.0131, 0.0217, 0.029, 0.0436
       		\\
     	    $\mu$ & 0.0143
			\\
     	    $d_1$ & 0
     	    \\
     	    $d_2$ & 0
     	    \\
     	    $k_1$ & 0.5
 			\\
     	    $k_2$ & 1
			\\
     	    $r_1$ & 2
     	    \\
     	    $r_2$ & 1
     	    \\
     	    $p$ & 0.4
     	    \\
     	    $q$ & 0.1
			\\
			$N$ & 6000, 12000, 30000
            \\
			$\Lambda$ & $\mu N$
            \\
            $t_f$ & $5$ years
            \\
     	    $B_1$ & $50$
			\\
     	    $B_2$ & $500$
     	    \\
     	    Lower bound for controls & $0.05$
			\\
            Upper bound for controls & $0.95$
       		\\
		\bottomrule
    \end{tabular}
	\caption{Values of the parameters}
\end{table}


    \begin{table}
	\centering
	\begin{tabular}{ll}
		\toprule
			\textbf{States} & \textbf{Values}
            \\
        \midrule
            $S(0)$ & $(76/120)N$
            \\
        	$L_1(0)$ & $(36/120)N$
			\\
            $I_1(0)$ & $(4/120)N$
       		\\
     	    $L_2(0)$ & $(2/120)N$
			\\
     	    $I_2(0)$ & $(1/120)N$
     	    \\
     	    $T(0)$ & $(1/120)N$
     	    \\
		\bottomrule
    \end{tabular}
	\caption{Initial conditions}
\end{table}
    %---------%---------%---------%---------%---------%---------%---------%---------
    In Figure \ref{Figure_TBM_1} we can observe that the infected population without these countermeasures is growing linearly, in contrast with the controlled model the infected population remains almost constant in the beginning.  After 3 years the controlled MDR-TB shows that the infected population grows faster but stays low. 
    
    \begin{figure}[htb] 
    	\begin{center}
        	\includegraphics[width=\textwidth,keepaspectratio]%
        	{Chapters/Chapter6_Applications2/Figures/figure_1_two_strain_tbm}
    		\caption{%
            	In the left side, the green line represents the uncotrolled state of
            	MDR-TB infected population (I/N) and the orange dashed line represents the 
            	controlled state. In the right side, the two controls are plotted
                The parameters that were used are:
        	}\label{Figure_TBM_1}
    	\end{center}
    \end{figure}
    %---------%---------%---------%---------%---------%---------%---------%---------
    \begin{figure}[htb] 
    	\begin{center}
        	\includegraphics[width=\textwidth,keepaspectratio]%
        	{Chapters/Chapter6_Applications2/Figures/figure_2_two_strain_tbm}
    		\caption{}\label{Figure_TBM_2}
    	\end{center}
    \end{figure}
    %---------%---------%---------%---------%---------%---------%---------%---------
    \begin{figure}[htb] 
    	\begin{center}
        	\includegraphics[width=\textwidth,keepaspectratio]%
        	{Chapters/Chapter6_Applications2/Figures/figure_3_two_strain_tbm}
    		\caption{}\label{Figure_TBM_3}
    	\end{center}
    \end{figure}
    %---------%---------%---------%---------%---------%---------%---------%---------
    %---------%---------%---------%---------%---------%---------%---------%---------
    \newpage