%---------%---------%---------%---------%---------%---------%---------%---------
\section{HIV}
%---------%---------%---------%---------%---------%---------%---------%---------
%---------%---------%---------%---------%---------%---------%---------%---------
    The Human Immunodeficiency Virus (HIV) is a condition that targets the 
    immune system and weakens people's defense system against other infections 
    and some types of cancer. The HIV has no cure and that's why it is one of 
    the biggest problems in public health. However, there exist treatments to 
    control this sickness, like drugs or chemotherapy. So, in this example
    the control will represent chemotherapy strategies, and the goal is to find 
    the  best chemotherapy strategy. 
    
    \noindent  We consider a model for HIV reported in \citep{butler1997optimal} which describes the interaction between the immune system and the HIV virus. Let $T(t)$ be the concentration of uninfected $CD4^{+}T$ cells and $T_i(t)$ the infected $CD4^{+}T$ cells. Let $V(t)$ be the concentration of free virus cells and $\dfrac{s}{1 + V(t)}$ the rate of generation of new $CD4^{+}T$ cells. We consider $r$ as the growth rate of T cells per day. This growth is assumed to be logistic, with a maximum level of $T_{max}$. Let $u(t)$ be the strength of the chemotherapy, where $u(t) = 0$ represents maximum therapy and $u(t) = 1$ is no therapy.
    The dynamics without control reads
    \begin{align*}
        \dfrac{dT}{dt} &= \dfrac{s}{1 + V} - \mu_{1}T + rT \left( 1 - %
                            \dfrac{T + T_{i}}{T_{max}} \right) - k_{1}VT , \\
        \dfrac{dT_{i}}{dt} &= k_{1}VT - \mu_{2}T_{i} , \\
        \dfrac{dV}{dt} &= N \mu_2 T_{i} - \mu_{3}V.
    \end{align*}
    And the objective is to maximize the functional
    $$
        \max_{u} J(u) = \max_{u} \int_{t_i}^{t_f} {\left[ AT(t) - 
                        (1 - u(t))^{2} \right]} dt, 
    $$
    subject to 
    \begin{equation}
    \label{eqn:ode_hiv}
        \begin{aligned}
            \dfrac{dT}{dt} &= \dfrac{s}{1 + V} - \mu_{1}T + rT \left( 1 - %
                \dfrac{T + T_{i}}{T_{max}} \right) - u(t)kVT \\
            \dfrac{dT_{i}}{dt} &= u(t)kVT - \mu_{2}T_{i} \\
            \dfrac{dV}{dt} &= N \mu_2 T_{i} - \mu_{3}V
        \end{aligned}
    \end{equation}
    where $ 0 \leq u(t) \leq 1 $ and $A \geq 0$ is the weight parameter. 
    Let $g =g(t,x,u)$, 
    %=  \left(
    %    \frac{dT}{dt}, \frac{dT_{i}}{dt}, \frac{dV}{dt} \right)$ and 
        $x = (T,T_i,V)$ the right hand side of ODE \eqref{eqn:ode_hiv}. 
    By the definition of the Hamiltonian (ref) we have that
    $$
        H(t,x,u,\psi) = AT - (1 - u)^{2} + \sum_{i \in J} \psi_{i}f_{i}, 
    $$
    with $i \in \{T, T_i, V \}$. Then the adjoint equations are
    \begin{align*}
        \dot{\psi}_{T}   
            &=  - A + \psi_{T} \left[\mu_1 
            - r\left(1 - \dfrac{T_i}{T_{max}}\right) \right] - \psi_{T_i} u k V      
            \\ 
        \dot{\psi}_{T_i} &=  \psi_{T}\dfrac{rT}{T_{max}} + 
            \psi_{T_i}\mu_2 - \psi_{V}N\mu_2                      \\
        \dot{\psi}_{V}   &=  \psi_{T} \left(\dfrac{s}{(1+V)^2} + 
            ukT\right) - \psi_{T_i}ukT + \psi_{V}\mu_{3}   
    \end{align*}

    \noindent  the Hamiltonian we obtain the optimality condition 
    $$
        \dfrac{\partial H}{\partial u} = 2(1 - u) + (\psi_{T_i} - \psi_{T})kVT
    $$
    
    \noindent Since the control is bounded, the optimality conditions are
    $$  
        \begin{cases}
            u^{*} = 0  \quad &\text{if} \quad  
               \frac{\partial H}{\partial u} < 0 \\
            0 \leq u^{*} \leq 1 \quad  &\text{if} \quad
                \frac{\partial H}{\partial u} = 0 \\
            u^{*} = 1 \quad  &\text{if} \quad  
                \frac{\partial H}{\partial u} > 0
        \end{cases}
    $$
    
    Considering the values of the following tables we obtain Figure \ref{Figure_HIV_1}
%---------%---------%---------%---------
\begin{table}
	\centering
	\begin{tabular}{ll}
		\toprule
			\textbf{Parameters} & \textbf{Values}
            \\
        \midrule
            $\mu_1$ & 0.02
            \\
        	$\mu_2$ & 0.2
			\\
            $\mu_3$ & 4.4
            \\
     	    $k$ & 0.000024
 			\\
     	    $r$ & 0.03
     	    \\
     	    $N$ & 300
            \\
     	    $T_{\text{max}}$ & 1500
     	    \\
     	    $s$ & 10
			\\
     	    $B$ & 10
     	    \\
		\bottomrule
    \end{tabular}
	\caption{Values for the parameters}
\end{table}
\begin{table}
	\centering
	\begin{tabular}{ll}
		\toprule
			\textbf{States} & \textbf{Values}
            \\
        \midrule
            $T(0)$ & 806.4
            \\
        	$T_{i}(0)$ & 0.04
			\\
            $V(0)$ & 1.5
            \\
		\bottomrule
    \end{tabular}
	\caption{Initial conditions for the states}
\end{table}
%---------%---------%---------%---------

    \noindent Can we discuss this example? I want to do a comparative of the same with no control.
    
\begin{figure}[htb] 
	\begin{center}
    	\includegraphics[width=\textwidth,keepaspectratio]%
        {Chapters/Chapter6_Applications2/Figures/figure_1_hiv_chemo}
		\caption{%
        	The horizontal axis represents time $t$ on days. The vertical axis
            represents, in each case, the states 
            Susceptibles $T_{\text{cells}}$, Infected $T_{\text{cells}}$, Virus and the control $u$, Chemotherapy. The parameters used are presented in (ref).
    	}\label{Figure_HIV_1}
	\end{center}
\end{figure}

%---------%---------%---------%---------%---------%---------%---------%---------
