\section{Ekeland's Variational Principle}
%---------%---------%---------%---------%---------%---------%---------%---------
%---------%---------%---------%---------%---------%---------%---------%---------    
    The aim of this section is to present and prove the Ekeland's 
    $\veps$-Variational Principle \citep{Ekeland_OTVP}. Also, we prove the Banach 
    fixed point theorem applying this principle. Before we prove the Ekeland
    principle we need to introduce a partial order to define what a 
    minimal point is. Let $(M,d)$ be a metric space, and let 
    $f : M \to \mathbf{R}$ be any function. Define a relation $\preceq$ on $M$ by
    the condition 
    $$
        y \preceq x \iff f(y) + d(x,y) \leq f(x).
    $$
    This is a partial ordering on M, that is, for all $x$, $y$, $z$ $\in M$ we have
    \begin{asparaenum}
        \item[i)] $x \preceq x$, (reflexivity)
        \item[ii)] $x \preceq y$ and $y \preceq x$ implies $x = y$, (antisymmetry)
        \item[iii)] $x \preceq$ and $y \preceq z$ implies $x \preceq z$. %
            (transitivity)
    \end{asparaenum}
    Now, we prove these properties
    \begin{asparaenum}
        \item[i)]
            Note that $f(x) + d(x,x) = f(x) \leq f(x)$, then $x \preceq x$
        \item[ii)]
            Suppose that $x \preceq y$ and $y \preceq x$ that is
            \begin{align}
                f(x) &+ d(y, x) \leq f(y), \label{po_ii_eq1}\\
                f(y) &+ d(x, y) \leq f(x). \label{po_ii_eq2}
            \end{align}
            If we add $\cref{po_ii_eq1}$ and $\cref{po_ii_eq2}$, then we get
            $
                f(x) + f(y) + 2 d(x,y) \leq f(x) + f(y).
            $
            Thus,
            $
                d(x, y) = 0,
            $
            which implies $x = y$.
        \item[iii)]
            Suppose that $ x \preceq y$ and $y \preceq z$, that is
            \begin{align}
                f(x) &+ d(y, x) \leq f(y), \label{po_iii_eq1}\\
                f(y) &+ d(y, z) \leq f(z). \label{po_iii_eq2}  
            \end{align}
            Adding $\cref{po_iii_eq1}$ and $\cref{po_iii_eq2}$, we get
            $
                f(x) + f(y) + d(y,x) + d(y, z) \leq f(y) + f(z).
            $
            Then $f(x) + d(x,y) + d(y, z) \leq f(z)$. 
            Using the triangle inequality $f(x) + d(x, z) \leq f(z)$, implying
            that $x \preceq z$.
    \end{asparaenum}
%---------%---------%---------%---------%---------%---------%---------%---------
%---------%---------%---------%---------%---------%---------%---------%---------
    We call a minimal point in the partial order $\preceq$ a \textit{d-point}.
    Thus, a point $x \in M$ is a $d$-point if $y \preceq x$ implies that $y = x$,
    or equivalently,
    $
        f(x) < f(y) + d(x,y),
    $
    for all $y \in M$, $y \neq x$. Now, define the set 
    $$
        S(x) := \{ y \in M : y \preceq x\} =%
                \{ y \in M : f(y) + d(x,y) \leq f(x) \}.
    $$
    Note that $x \in S(x)$, since $x \preceq x$, so $S(x) \neq \varnothing$.
    Since $\preceq$ is a partial order, we claim that $y \preceq x$ if and only
    if $S(y) \subseteq S(x)$. Suppose that $y \preceq x $ and let $z \in S(y)$, 
    then $z \preceq y$ and $y \preceq x$. This implies that $z \preceq x$, 
    so $z \in S(x)$. Now, suppose that $S(y) \subseteq S(x)$, then $y \in S(x)$.
    Which implies $y \preceq x$. Note that if $f$ is a $\lsc$
    function then $S(x)$ is a closed subset of M. Also a $d$-point $x$ is 
    characterized by the condition that $S(x)$ is  a singleton, that is,
    $S(x) = \{ x \}$. Now, we show an auxiliary result to prove the Ekeland
    principle.
    \begin{theorem}\label{Thm_3.1}
        Let $(M, d)$ be a metric space. The following condition are equivalent:
        \begin{asparaenum}
            \item[a)]
                $(M, d)$ is a complete metric space,
            \item[b)]
                For any proper $\lsc$ function  $f: M \to \mathbb{R}\cup%
                \{+ \infty\} $  bounded from below, and any point $x_0 \in M$,
                there exists a $d$-point $x_0$ satisfying $x \preceq x_0$
        \end{asparaenum}
    \end{theorem}
    
    \begin{proof} \hspace{1cm} \\
        $b) \implies a)$.
        
        Fix $x_0 \in M$. Let $\{x_n\}_{n=1}^{\infty}$ be a Cauchy sequence in M.
        Consider the proper $\lsc$ function
        $$
            f(x) := 2 \lim_{n \to \infty} d(x, x_n).
        $$
        Let us prove the numerical sequence $\{ d(x,x_n)\}$ is a Cauchy sequence
        in $\mathbb{R}$. Since $\{ x_n\}$ is a Cauchy sequence, given $\veps$
        there is $N(\varepsilon) \in \mathbb{N}$ such that for all $m, n \geq N$, 
        $
            d(x_m, x_n) < \varepsilon.
        $
        It follows from 
        \begin{align*}
            d(x, x_n) &\leq d(x,x_m) + d(x_m,x_n), \\
            d(x, x_m) &\leq d(x,x_n) + d(x_n,x_m),
        \end{align*}
        that
        $
            \abs{d(x, x_m) - d(x, x_n)} \leq d(x_m, x_n) \leq \varepsilon,
        $
        for all $m, n \geq N$. Then, $\{ d(x, x_n)\}$ is a Cauchy sequence and
        this implies that $f$ is well-defined. \\
        We claim that $f$ is continuous at $x^{*} \in M$. Let  $\varepsilon>0$,
        $x^{*} \in M$ and take $\delta = \varepsilon /2$ such that $d(x, x^{*})< \delta$,
        with $x \in M$. Then, for any $n \in \mathbb{N}$
        \begin{equation} \label{Thm3.1_eq1}
            d(x, x_n) \leq d(x, x^{*}) + d(x^{*}, x_n),
        \end{equation}
        and
        \begin{equation} \label{Thm3.1_eq2}
            d(x^{*}, x_n) \leq d(x^{*}, x) + d(x, x_n).
        \end{equation}
        Combining $\cref{Thm3.1_eq1}$ and $\cref{Thm3.1_eq2}$, we obtain
        $$
            \abs{d(x,x_n) - d(x^{*}, x_n)} \leq d(x, x^{*}) \leq \frac{\varepsilon}{2}.
        $$
        Letting $n \to \infty$, we have 
        $$
            \abs{2\lim_{n \to \infty} d(x, x_n) - 2\lim_{n \to \infty} d(x^{*}, x_n)} %
            < \varepsilon.
        $$
        Hence, $f$ is continuous at $x^{*}$. Now, note $f(x_n) \to 0$, since
        \begin{align*}
            \lim_{n \to \infty} f(x_n) 
            &= 
                \lim_{n \to \infty} \left[2 \lim_{m \to \infty} d(x_n, x_m) \right] \\
            &= 2 \lim_{n \to \infty} \lim_{m \to \infty} d(x_n, x_m) \\
            &= 2 \cdot 0 = 0.
        \end{align*}
        Let $x \in M$ be a $d$-point of $f$, by definition 
        $f(x) < f(y) + d(x, y)$, for all $y \in M$. Then 
        $f(x) < f(x_n) + d(x, x_n)$. Letting $n \to \infty $ we have
        $$
            0 \leq \lim_{n \to \infty} d(x, x_n) = f(x) < f(x_n) + d(x, x_n) = %
                \frac{f(x)}{2}.
        $$
        This implies that $f(x) = 0$. So $d(x, x_n) \to 0$ as $n \to \infty$. 
        Hence, $M$ is a complete metric space. \\
%---------%---------%---------%---------%---------%---------%---------%---------
        $a) \implies b)$.
        Suppose that M is a complete metric space and let 
        $f: M \to \mathbb{R} \cup \{ + \infty\}$ be a proper $\lsc$ bounded
        below. We assume that $f(x_0) <\infty$ for $x_0 \in M$. Generate a
        sequence $\{x_n\}_{n=1}^{\infty}$ recursively such that given 
        $x_n \in M$, the term $x_{n+1} \in S(x_n)$. We claim that $\{x_n\}$ is a
        Cauchy sequence. Note that $x_{n+1} \in S(x_n)$, implies 
        $f(x_{n+1}) + d(x_n, x_{n+1}) \leq f(x_n)$. So,
        $$
            0 \leq d(x_n, x_{n+1}) \leq f(x_n) - f(x_{n+1}).
        $$
        Hence, the sequence $\{f(x_n)\}$ is decreasing. Since $f$ is bounded
        below, the sequence $\{f(x_n)\}$ converges. In particular, $\{f(x_n)\}$
        is a Cauchy sequence. Now, for any $n,m \in \mathbb{N}$ such that
        $n > m$ we have $d(x_n, x_m) < f(x_m) - f(x_n)$. Letting $m,n \to \infty$
        we see that $d(x_n, x_m) \to 0$. 
        That is, $\{x_n\}$ is a Cauchy sequence in M. Since M is complete, 
        $\{x_n\}$ converges to $x_0 \in M$. Moreover, $x_k \in S(x_n)$ for all
        $k \geq n$. We also have that $S(x_n)$ and 
        $$
            S(x) \subseteq \bigcap_{n =1}^{\infty} S(x_n).
        $$
        We prove that $S(x) = \{x\}$. We choose the following sequence 
        $\{ \widehat{x}_{n} \}$ such that
        $
            \widehat{x}_{n+1} \in S(\widehat{x}_{n}),
        $
        and
        $$
            f(\widehat{x}_{n+1}) \leq \inf \{ f(y) : y \in S(\widehat{x}_{n}) \} %
                + \dfrac{1}{n}.
        $$
        Let $z \in S(x)$, then $z \preceq x \preceq \widehat{x}_{n-1} \preceq \widehat{x}_{n}$ and
        $$
            f(z) + d(z, \widehat{x}_{n}) \leq f(\widehat{x}_{n}) \leq %
            \inf \{f(y) : y \in \widehat{x}_{n-1}\} + \frac{1}{n}. 
        $$
        Thus, letting $n \to \infty$, $d(z, \widehat{x}_{n}) \to 0$. Hence
        $
            \widehat{x}_{n} \to z = x,
        $
        that is,
        $
            S(x) = \{x\}.
        $
    \end{proof}
%---------%---------%---------%---------%---------%---------%---------%---------
%---------%---------%---------%---------%---------%---------%---------%---------
    Now, we enunciate and prove the Ekeland's $\veps$-Variational 
    principle.
    \begin{theorem}[{\cite[Thm. 3.2]{guler2010foundations}}, Ekeland's $\veps$%
                    -Variational Principle]\label{thm_evp}
        Let $(M, d)$ be a complete metric space, and let $f : M \to \extRealp$ be 
        a proper $\lsc$ function that is bounded from below. Then, 
        for every $\varepsilon > 0$, $\lambda > 0$, and $x \in M$ such that
        $$ 
            f(x) \leq \inf_{M} f + \varepsilon,
        $$
        there exists an element $x_{\varepsilon}\in M$ satisfying the following
        properties:
        \begin{asparaenum}[i)]
            \item 
                $ f(x_{\varepsilon}) \leq f(x)$,
            \item 
                $d(x_{\varepsilon}, x) \leq \lambda$,
            \item
                $f(x_{\varepsilon}) < f(z) + %
                \frac{\varepsilon}{\lambda}d(z,x_{\varepsilon})$, for all 
                $ z\in M, z \neq x_{\varepsilon}$.
        \end{asparaenum}
    \end{theorem}
    
    \begin{proof}
        Note that is enough to prove the result for $\lambda = 1$ and $\varepsilon = 1$,
        since we can replace $d$ by $d/\lambda$ and $f$ by $f/ \varepsilon$. Since
        $(M,d)$ is a complete metric space and $f$ is a proper $\lsc$ function bounded
        below, the conditions in \eqref{Thm_3.1} are satisfied. That is, for $x \in M$
        there exists a $d$-point $\bar{x} \preceq x$. We claim that $\bar{x}$ satisfies
        the properties in \Cref{thm_evp}:
        \begin{asparaenum}
            \item[i)] $f(\bar{x}) \leq f(x)$. \\
                Since $\bar{x} \preceq x$, we have
                \begin{equation} \label{thm_evp_eq1}
                    f(\bar{x}) \leq f(\bar{x}) + d(x, \bar{x}) \leq f(x),
                \end{equation}
                then
                $
                    f(\bar{x}) \leq f(x).
                $
            \item[ii)] $d(\bar{x}, x) \leq 1$. \\
                By hypothesis 
                $
                    f(x) \leq \inf_{x \in M} f(x) + 1 \leq f(\bar{x}) + 1.
                $
                Using $\eqref{thm_evp_eq1}$, we get
                $$
                    f(\bar{x}) + d(x, \bar{x}) \leq f(x) \leq f(\bar{x}) + 1.
                $$
                Then 
                $
                    d(x, \bar{x}) \leq 1.
                $
            \item[iii)] $f(\bar{x}) < f(x) + d(x, \bar{x})$. \\
                Since $\bar{x}$ is  a $d$-point 
                $
                    f(\bar{x}) <  f(x) + d(x, \bar{x}),
                $
                for all $x \in M$, with $x \neq \bar{x}$.
        \end{asparaenum}
    \end{proof}
    
    \noindent If we suppose that $M = \mathbb{R}^{n}$ and $d(x,y) = \norm{x - y}$ 
    in \Cref{thm_evp}, then we have an elementary proof.
%---------%---------%---------%---------%---------%---------%---------%---------
%---------%---------%---------%---------%---------%---------%---------%---------
    \begin{proof}
        As above, suppose that $\lambda = 1$ and define
        $
            g(z) := f(z) + \varepsilon \norm{z - x}.
        $
        Note that $f$ is $\lsc$ and $\norm{\cdot}$ is continuous, so $g$ is
        $\lsc$. Also, note that
        $
            \norm{z - x} \to \infty,
        $
        as $\norm{z} \to \infty$, because $x$ is fixed. Also, $f$ is bounded below, which
        implies that $g$ is coercive, that is
        $$
            \lim_{\norm{z} \to \infty} g(z) = + \infty.
        $$
        Consider the set of minimizers of $g$
        $$
            K := \left\{ m \in \mathbb{R}^{n} : g(m) \leq g(x), \forall x \in  %
             \mathbb{R}^{n} \right\}.
        $$
        By Corollary (2.43, part (b), i need this reference), the set $K$ is 
        non-empty and compact. \\
        Let $x_{\varepsilon} \in K$ be a point that minimizes g on K, that is $ g(x_{\varepsilon}) \leq g(z)$ for all $z \in K$. By definition of $g$ we have
        $$
            f(x_{\varepsilon}) + \varepsilon \norm{x_{\varepsilon} - x} \leq %
            f(z) + \varepsilon \norm{z-x},
        $$
        for all $z\in \mathbb{R}^{n}$. If $z = x$, then 
        \begin{align*}
            f(x_\varepsilon) + \varepsilon \norm{x_{\epsilon} - x} 
                &\leq f(x) \\
                &\leq \inf_{y \in \mathbb{R}^{n}} f(y) + \varepsilon \\
                &\leq f(x_{\varepsilon}) + \varepsilon .
        \end{align*}
        From the first inequality we have
        $$
            f(x_\varepsilon) \leq f(x_\varepsilon)  + \varepsilon\norm{x_\varepsilon -x} %
            \leq f(x),
        $$
        so, $f(x_{\varepsilon}) \leq f(x)$. Further,
        $
            f(x_{\varepsilon}) + \varepsilon \norm{x_{\varepsilon} - x} \leq %
                f(x_{\varepsilon}) + \varepsilon,
        $
        imply $\norm{x_{\varepsilon} - x} \leq 1$. 
        Hence, the first and second properties holds. To prove the third 
        condition  note that if $z\in K$ and $z\neq x_{\varepsilon}$ then
        $$
            f(x_{\varepsilon})\leq f(z)
        $$
        for all $z\in K$. Then
        $$
            f(x_{\varepsilon})\leq f(z)< f(z)+ \varepsilon\norm{z-x_{\varepsilon}},
        $$
        and the third condition holds over $K$. Now, if $z\notin K$, then
        \begin{align*}
            f(x_{\varepsilon})+\varepsilon\norm{x_{\varepsilon}-x} &< f(z)+\varepsilon
            \norm{z-x} \\
            &\leq f(z)+\varepsilon(\norm{z-x_{\varepsilon}}+\norm{x_{\varepsilon}-x}).
        \end{align*}
        Thus
        $$
            f(x_{\varepsilon})+\varepsilon\norm{x_{\varepsilon}-x}< f(z)+\varepsilon
            \norm{z-x_{\varepsilon}} + \varepsilon\norm{x_{\varepsilon}-x},
        $$
        implies $f(x_{\varepsilon})<f(z)+\varepsilon\norm{z-x_{\varepsilon}}$.
    \end{proof}
    
    \noindent Taking a particular value of $\lambda$ we get the following
    corollary. 
    
    \begin{corollary}\label{EVP_Corollary}
        Let the function $f$ and the point $x$ satisfying the conditions in the
        \Cref{thm_evp}. Then there exists a point $x_{\veps}$ satisfying
        the following conditions:
         \begin{align}
            & f(x_{\varepsilon}) \leq f(x), \notag \\
            & d(x_{\varepsilon}, x) \leq \sqrt{\varepsilon},  \\
            &  f(z) >  f(x_{\varepsilon}) -\sqrt{\varepsilon}d(z,x_{\varepsilon}), \ %
             \text{ for all } z\in M, z \neq x_{\varepsilon}. \notag 
        \end{align}
    \end{corollary}
    \begin{proof}
        Let $\lambda = \sqrt{\epsilon}$ and apply \Cref{thm_evp}.
    \end{proof}

    \noindent An application of the Ekeland's  $\veps$-Variational Principle is the proof 
    of the Banach Fixed Point Theorem.:

    \begin{definition}
        Let $\varphi : M \to M$ be a mapping . A point $\bar{x} \in M$ is called
        a fixed point of $\varphi$ if 
        $$
            \varphi(\bar{x}) = (\bar{x}).
        $$
    \end{definition}
    
    \noindent The mapping $\varphi$ is called a contractive mapping if there
    exists $\alpha \in [0,1)$ such that $d(\varphi(x), \varphi(y)) \leq %
    \alpha d(x,y)$

    \begin{theorem}[Banach Fixed Point Theorem] \label{thm: BFP} %Banach Fixed Point Theorem
        A contractive mapping $\varphi : M \to M$ on a complete metric space $(M,d)$
        has a unique fixed point.
    \end{theorem}
    \begin{proof}
        Let $\varphi$ be a contractive mappping. Define a function 
        $$
            f(x):= d(x, \varphi(x))\leq 0,
        $$
        and choose $\varepsilon\in (0, 1-\alpha)$. Note that $f$ is continuous and
        bounded from below by definition. Thus, all the hypothesis on the Ekeland's
        principle are satisfied so, there is $\bar{x}\in M$ such that
        \begin{equation}\label{thm_bfp_eq1}
            f(\bar{x})\leq f(x)+\varepsilon d(x, \bar{x})
        \end{equation}
        for all $x\in M$ with $x=\bar{x}$ and each $\varepsilon\in (0, 1-\alpha)$.
        
        Proceeding by contraction according to $\eqref{thm_bfp_eq1}$, we suppose that
        $\varphi\neq\bar{x}$. This implies, that exists $x\in M$ such that
        $\varphi(\bar{x})=x$, $x\neq\bar{x}$. By $\eqref{thm_bfp_eq1}$, we have
        \begin{align*}
            d(\varphi(\bar{x}), \bar{x}) = f(\bar{x}) 
                &\leq
                    f(x)+ \varepsilon d(x, \bar{x}) \\
                &= 
                    d(\varphi(\bar{x}), \varphi(\varphi(\bar{x})))+\varepsilon 
                    d(\varphi(\bar{x}),\bar{x}) \\
                &\leq 
                    \alpha d(\bar{x}, \varphi(\bar{x})) + \varepsilon 
                    d(\varphi(\bar{x}), \bar{x}) \\
                &=
                    (\alpha +\varepsilon) d(\varphi(\bar{x}),\bar{x}).
        \end{align*}
        Then, 
        \begin{equation}\label{thm_bfp_eq2}
            d(\varphi(\bar{x}), \bar{x})\leq (\alpha+\varepsilon) %
            d(\varphi(\bar{x}),\bar{x})
        \end{equation}
        Note that $\alpha+\varepsilon<\alpha +1-\alpha=1$, and from
        $Cref{thm_bfp_eq2}$, $1\leq \alpha+\varepsilon$, which leads to a
        contradiction. Hence  $\varphi(\bar{x})=\bar{x}$. 
        
        Now, to prove the uniqueness of the fixed point suppose that there
        are two fixed points $x_{1}$ and $x_{2}$, then
        $$
            d(x_{1}, x_{2})=d(\varphi(x_{1}), \varphi(x_{2}))\leq \alpha %
            d(x_{1}, x_{2})< d(x_{1}, x_{2}),
        $$
        which is a contraction.
    \end{proof}
%---------%---------%---------%---------%---------%---------%---------%---------
%---------%---------%---------%---------%---------%---------%---------%---------