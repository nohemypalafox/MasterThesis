\section{The Forward Backward Sweep Method}
%[theorem name, {\citep[Thm. #, page]{reference}}]
%---------%---------%---------%---------%---------%---------%---------%---------
%---------%---------%---------%---------%---------%---------%---------%---------
    The Pontryagins principle allows us to transform the problem of finding a
    control which optimizes the cost functional to optimizing the Hamiltonian. 
    To this end, we need to solve forward in time the dynamics with a given 
    initial condition, and backward in time the adjoint equation with a 
    transversality condition, using the four-stage  RK method. This way of
    solving is called the forward backward sweep metho. So the idea is to 
    follow the next steps.
    
    \begin{description}
    	\item[Step 1.]
        	Make an initial guess for $\vec{u}$ over the interval.
        \item[Step 2.]
        	Using the initial condition $x_1 = x(t_0) = a$ and the values for 
            $\vec{u}$, solve $\vec{x}$ forward in time according to its differential
            equation in the optimality system.
    	\item[Step 3.]
        	Using the transversality condition $\lambda_{N+1} = \lambda(t_1) = 0$ 
            and the values for $\vec{u}$ and $\vec{x}$, solve $\vec{\lambda}$ 
            backward in time according to its differential equation in the optimality
            system.
        \item[Step 4.]
        	Update $\vec{u}$ by entering the new $\vec{x}$ and $\vec{\lambda}$ values 
            into the characterization of the optimal control. 
    	\item[Step 5.]
        	Check convergence. If the values of the variables in this iteration and 
            the last iteration and the last iteration are negligibly close, output the 
            current values as solutions. If values are not close, return to Step 2.
    \end{description}

    Using Python, we made an implementation of this method \citep{python_Thesisrepo},
    which follows the next  \Cref{FBSM_alg}. 

\begin{algorithm}
	\caption{Forward Backward Sweep }\label{FBSM_alg}
    INPUT: $t_0, t_f, n_{max}, x_0,h, a, r, m, \epsilon, \lambda_{f}$ \\
    OUTPUT: $x^*, u^*, \lambda$ \\
	\begin{algorithmic}[1]
		\Procedure{Forward backward sweep}{$g,\lambda_{\text{function}}, 
        u, x_0, 
        \lambda_f, h, n_{max}$} 
			\While{$ \text{test} > \epsilon $}
				\State $u_{\text{old}} \gets u$ 
                \State $x_{\text{old}} \gets x$ 
                \State $ x \gets$ \Call{runge\_kutta\_forward}%
                        {$g, u, x_0, h,n_{max}$}
                \State $\lambda_{\text{old}} \gets \lambda $
				\State $\lambda \gets$ \Call{runge\_kutta\_backward}%
				        {$\lambda_{\text{function}}, x, \lambda_f, h, n_{max}$}
                \State $\displaystyle u_1 \gets$ \Call{optimality\_condition} %
                        {$u, x, \lambda$}
                \State $\displaystyle u \gets \frac{u_1 + u_{old}}{2}$
                \State $test_1 \gets \displaystyle 
                \frac{||u - u_{\text{old}}||}{||u||}$
                \State $test_2 \gets \displaystyle 
                \frac{||x - x_{\text{old}}||}{||x||}$
                \State $test_3 \gets \displaystyle 
                \frac{||\lambda - \lambda_{\text{old}}||}{||\lambda||}$
                \State $\text{test} \gets \max{ \{ test_1, test_2, test_3 \}}$
			\EndWhile\label{}
			\State \textbf{return} $ x^*, u^*, \lambda$
            \Comment{Optimal pair}
		\EndProcedure
	\end{algorithmic}
\end{algorithm}
%---------%---------%---------%---------%---------%---------%---------%---------
%---------%---------%---------%---------%---------%---------%---------%---------

    Now, consider the following maximization problem.
    \begin{equation*}
        \max_{u} \int_{0}^{1} Ax(t) - B(u(t))^{2} dt
    \end{equation*}
    with $A \geq 0$, $B>0$, subject to the dynamics
    \begin{equation*}
        x'(t) = -\dfrac{1}{2}(x(t))^{2} + Cu(t), \, \, x(0) = x_0 >-2.
    \end{equation*}
    Note that the Hamiltonian 
    $$
        H = Ax - Bu^2 - \dfrac{1}{2}\lambda x^2 + C\lambda u.
    $$
    Using the optimility condition, 
    $$
        \dfrac{\partial H}{\partial u} = -2Bu + C\lambda
    $$
    we get $u^{*} = \dfrac{C\lambda}{2B}$. Calculating the adjoint equation we
    find
    \begin{align*}
        x'(t) &= -\dfrac{1}{2}x^2 + Cu, \, \, x(0) = x_0, \\
        \lambda'(t) = -A + x \lambda, \, \, \lambda(1) = 0.
    \end{align*}
    
    

    
    \todo{example 4.1}