%
\chapter{The Forward-Backward Sweep Method}
\lhead{\emph{Numerical Methods}}
%---------%---------%---------%---------%---------%---------%---------%---------
%---------%---------%---------%---------%---------%---------%---------%---------
%           MODIFICAR :
    The purpose of this chapter is to present and explain the four-stage Runge 
    Kutta (RK) method and the forward backward sweep method. Both of them are 
    used to find the optimal control. Also, we present examples of a one-stage 
    and two-stage explicit RK method, and an example of how to apply the
    forward-backward sweep method. For a more extensive explanation see
    \cite{griffiths2010numerical}.
%%%%%%%%%       \section*{The Runge Kutta Method}
%---------%---------%---------%---------%---------%---------%---------%---------

%---------%---------%---------%---------%---------%---------%---------%---------
    \noindent Consider the following initial value problem (IVP)
    \begin{equation}\label{eq:ivp}
        \begin{aligned}
            x'(t)  &= f(t, x(t)), \hspace{1cm} t \geq t_0, \\
            x(t_0) &= x_0,
        \end{aligned}
    \end{equation}
%       MODIFICAR: 
    The Runge-Kutta (RK) method is used to approximate the solution of the IVP 
    described by \cref{eq:ivp}. This method is iterative and consists of a number
    of stages. For example, Euler's method is a one-stage RK method. The RK 
    method computes the slopes of nearby points at a time $t_n$ and then 
    calculates the mean of these slopes to approximate the solution at the next
    time $t_{n+1}$. The general $s$-stage RK method is defined as
    \begin{equation}\label{eq:GenSSRK}
        x_{n+1} = x_n + h \sum_{i=1}^{s} b_i k_i,
    \end{equation}
    where the terms $k_i$ are computed from the following evaluation in the 
    right-hand side of the IVP  \eqref{eq:ivp}:
    \begin{equation}\label{eq:kterms}
        k_i = f \left( t_n + c_i h, x_n + h \sum_{j=1}^{s} a_{i,j}k_j\right), %
                \hspace{1cm} i = 1, \cdots s,
    \end{equation}
    with
    \begin{equation}
        c_i =  \sum_{j=1}^{s} a_{i,j}, \hspace{1cm} i = 1, \cdots, s.
    \end{equation}
%---------%---------%---------%---------%---------%---------%---------%---------
%---------%---------%---------%---------%---------%---------%---------%---------
    Thus, given a value of $s$, the method depends on $s^2 +s$ parameters
    $\{a_{i,j},b_{j}\}$. These parameters can be displayed in a tableau known as the
    Butcher array (see \cref{tab:GButcherArray}).
    \begin{table}[h!]\label{tab:GButcherArray}
        \begin{center}
            \begin{tabular}{l | l } 
                c  & A  \\
                \hline
                \hspace{0.1cm} & b
            \end{tabular}
            \caption{General form of a Butcher array.}
        \end{center}
    \end{table}
%---------%---------%---------%---------%---------%---------%---------%---------
%---------%---------%---------%---------%---------%---------%---------%---------
    \noindent Vector $c$ indicates the positions within the step of the 
    stage values. The matrix $A$ indicates the dependence of the stages on the
    derivatives found at other stages. And $b$ is a vector of quadrature weights,
    showing how the final result depends on the derivatives computed at various
    stages. In the explicit RK methods the upper-triangular components of the 
    matrix $A$ are zero. So, in Table \cref{tab:ExplicitBA} is shown an 
    extended Butcher array for an explicit RK method. 
    \begin{table}[h!]\label{tab:ExplicitBA}
        \begin{center}
            \begin{tabular}{l | l l l l l} 
                0     & 0 & 0 & $\cdots$ &0 &    0 \\
                $c_2$     & $a_{2,1}$ & 0 & $\cdots$ &0 & 0 \\
                $c_3$     & $a_{3,1}$ & $a_{3,2}$ & $\cdots$ & 0& $\vdots$ \\
                $\vdots$  & $\vdots$  & $\vdots$  &          &0 & 0 \\
                $c_s$     & $a_{s,1}$ & $a_{s,2}$ & $\cdots$ & $a_{s,s-1}$ & 0 \\
                \hline
                \hspace{0.1cm} & $b_1$ & $b_2$ & $\cdots$ & $_{s-1}$ & $b_s$ \\
            \end{tabular}
            \caption{The Butcher array for an explicit RK method}
        \end{center}
    \end{table}
%---------%---------%---------%---------%---------%---------%---------%---------
%---------%---------%---------%---------%---------%---------%---------%---------
    Let us introduce the definition of the local truncation error. 
    \begin{definition}[Local Truncation Error, {\citep[def. 9.3, p. 127]{griffiths2010numerical}}]
        The Local Truncation Error, $T_{n+1}$ of an RK method is defined to be
        the difference between the exact and the numerical solution of the IVP at
        time $t = t_{n+1}$:
        $$
            T_{n+1} = x(t_{n+1}) - x_{n+1},
        $$
        under the localizing assumption that $x_n = x(t_n)$. If $T_{n+1} = \mathcal{O}^{p+1}$,
        the method is said to be of order $p$.
    \end{definition}
    We obtain the parameters in the RK methods, using the definition below. To clarify this idea, we present an example of a one-stage RK method, that is,
    $s = 1$. Following the definitions \eqref{eq:GenSSRK} and \cref{eq:kterms},
    we get
    $
        x_{n+1} = x_n + hb_1 k_1,
    $
    and
    $
        k_1 = f(t_n + c_1 h, x_n + h a_{1,1} k_1).
    $
    By the definition of $c_i$ we have that, $c_1 = a_{1,1} = 0$. Also, 
    $k_1 = f(t_n,x_n) = f_n$, then $ x_{n+1} = x_n + h b_1 f_n$. To find the
    coefficient $b_1$, we have to compare this expression with the expression for
    $x(t_{n+1})$:
    $$
        x(t_{n+1}) = x(t_n) + hx'(t_n) + \dfrac{1}{2}h^{2}x''(t_n) + \mathcal{O}(h^{3}).
    $$
    To do this comparison, we need to differentiate the equation $x'(t) = f(t,x(t))$, 
    with respect to t. By the chain rule, we obtain
    $$ 
        x''(t) = \dfrac{\partial f}{\partial t}t' + %
                 \dfrac{\partial f}{\partial x}  x'(t) %
            = f_{t} + f_{x} f.
    $$
    Then, 
    \begin{align*}
        T_{n+1}
            &= x(t_{n+1}) - x_{n+1} \\
            &= x(t_n) + hx'(t_n) + \dfrac{1}{2}h^2 x''(t_n)  - x_n - hb_1 f_n +  \mathcal{O}(h^3)\\
            &= x_n + hf_n + \dfrac{1}{2}h^2(f_t + ff_x)\vert_{t = t_n} - x_n - hb_1f_n + \mathcal{O}(h^3)\\
            &= h(1-b_1)f_n + \dfrac{1}{2}h^2(f_t + ff_x)\vert_{t = t_n} + \mathcal{O}(h^3),
    \end{align*}
    Choosing $b_1 = 1$ the method is consistent of order $p=1$. This choosing 
    gives the only first-order one-stage explicit RK method, the Euler's method
    $$
        x_{n+1} = x_n + hf_n.
    $$
%---------%---------%---------%---------%---------%---------%---------%---------

%---------%---------%---------%---------%---------%---------%---------%---------
    To fix ideas, consider the next example for the two stage RK method, which 
    is a modified Euler method. Let the IVP
    \begin{equation}\label{eq:IVP2}
        \begin{aligned}
            x'(t) &= (1-2t)x(t), \hspace{1cm} t>0, \\
            x(0)  &= 1.
        \end{aligned}
    \end{equation}
%---------%---------%---------%---------%---------%---------%---------%---------
%---------%---------%---------%---------%---------%---------%---------%---------
    According to the two stage RK method, 
    \begin{align*}
        t_{n+1} &= t_n + h, \\
        x_{n+1} &= x_n + hk_2, 
    \end{align*}
    and
    \begin{equation}\label{eq:IVP2_kterms}
        \begin{aligned}
            k_1 &= f(t_n, x_n), \\
            k_2 &= f\left(t_n + \dfrac{1}{2}h, x_n + \dfrac{1}{2}hk_1 \right),
        \end{aligned}
    \end{equation}
    we calculate the approximate solution of \ref{eq:IVP2} choosing $h=0.2$.
    Thus, substituting \ref{eq:IVP2_kterms} in \ref{eq:IVP2} we obtain
    \begin{align*}
        k_1 &= (1- 2t_n)x_n,\\
        k_2 &= (1 - (2t_n + h))\left(x_n + \dfrac{1}{2}h k_1\right).
    \end{align*}
    So, for $n = 0$, we have
    \begin{align*}
        k_1 &= (1-2t_0)x_0 = (1 - 2(0))(1) = 1,    \\ 
        k_2 &= (1 - 2t_0 - h)(x_0 + 0.5hk_1) = (1-0.2)(1+0.5(0.2)(1)) = 0.88, \\
        t_1 &= t_0 + h = 0 + 0.2 = 0.2, \\
        x_1 &= x_0 + hk_2 = 1 + (0.2)(0.88) = 1.176.
    \end{align*}
    Now, for $n=1$, yields
    \begin{align*}
        k_1 &= (1-2t_1)x_1 = (1 - 2(0.2))(1.176) = 0.7056,    \\ 
        k_2 &= (1 - 2t_1 - h)(x_1 + 0.5hk_1) =
            (1-2(0.2)-0.2)(1.176+0.5(0.2)(0.7056)) = 0.4986, \\
        t_2 &= t_1 + h = 0.2 + 0.2 = 0.4, \\
        x_2 &= x_1 + hk_2 = 1.176 + (0.2)(0.4986) = 1.2757,
    \end{align*}
    and so on. This is an example of an explicit RK method.
    
%    -> here goes a tiny graph <3
%---------%---------%---------%---------%---------%---------%---------%---------
%---------%---------%---------%---------%---------%---------%---------%---------
    Now, letting $s=4$, we get a four-stage RK explicit method for the IVP 
    (\ref{eq:ivp})
    \begin{equation}\label{RK4_method}
        x_{n+1} = x_n + h(b_{1}k_{1} + b_{2}k_{2} + b_{3}k_{3} + b_{4}k_{4})
    \end{equation}
    where
    \begin{equation}\label{RK4_kterms}
        \begin{aligned}
            k_1 &=  f(t_n, x_n), \\
            k_2 &=  f(t_n + c_2 h, x_n + h a_{2,1}k_1), \\
            k_3 &=  f\left(t_n + c_3 h, x_n + h\sum_{j=1}^{2}a_{3,j}k_j\right), \\
            k_4 &=  f\left(t_n + c_4 h, x_n + h\sum_{j=1}^{3}a_{1,j}k_j\right),
        \end{aligned}
    \end{equation}
    To obtain the parameters we have to satisfy the following conditions
    \begin{align*}
        b_1 + b_2 + b_3 + b_4 &= 1, \\
        b_2 c_2 + b_3 c_3 + b_4 c_4 &= \dfrac{1}{2}, \\
        b_2 {c_2}^{2} + b_3{c_3}^{2} + b_4{c_4}^{2} &= \dfrac{1}{3}, \\
        b_3 a_{3,2}c_2 + b_4 a_{4,2}c_2 + b_4 a_{4,3}c_3 &= \dfrac{1}{6}, \\
        b_2 {c_2}^{3} + b_3{c_3}^{3} + b_4{c_4}^{3} &= \dfrac{1}{4}, \\
        b_3 c_3 a_{3,2}c_2 + b_4 c_4 a_{4,2}c_2 + b_4 c_4 a_{4,3}c_3 &= \dfrac{1}{8}, \\
        b_3 a_{3,2}{c_2}^{2} + b_4 a_{4,2}{c_2}^{2} + b_4 a_{4,3}{c_3}^{2} &= \dfrac{1}{12}, \\
        b_4 a_{4,3}a_{3,2} c_2 &= \dfrac{1}{24}.
    \end{align*}
%---------%---------%---------%---------%---------%---------%---------%---------
%---------%---------%---------%---------%---------%---------%---------%---------
    There is a lot of solutions and families of solutions. We show in Table \cref{tab:BA-RK4} some parameters that solves the equations
    below. This set of parameters conform the Runge Kutta-Felberg method
    of fourth order. This method is common used in package from R, Julia and
    MATLAB. 
    
    \begin{table}[h!]\label{tab:BA-RK4}
        \begin{center}
            \begin{tabular}{l | l l l l l} 
                0     & 0 &  &  &     \\
                $\frac{1}{2}$ & $\frac{1}{2}$ & 0 &  & \\
                $\frac{1}{2}$ & 0 & $\frac{1}{2}$ & 0 & \\
                1  & 0  &  0  & 1 & 0 \\
                \hline
                \hspace{0.1cm} & $\frac{1}{6}$ & $\frac{1}{3}$ & $\frac{1}{3}$ & $\frac{1}{6}$  \\
            \end{tabular}
            \caption{The Butcher array for an explicit four stage RK method}
        \end{center}
    \end{table}
    
    \noindent Substituting the parameters showed in Table \cref{tab:BA-RK4} 
    in (\ref{RK4_method}) and (\ref{RK4_kterms}), we obtain the Runge Kutta-Felberg method
    \begin{equation}\label{RK4_method}
        x_{n+1} = x_n + \dfrac{1}{6}h(k_{1} + 2k_{2} + 2k_{3} + k_{4})
    \end{equation}
    where
    \begin{equation}\label{RK4_kterms}
        \begin{aligned}
            k_1 &=  f(t_n, x_n), \\
            k_2 &=  f\left(t_n + \dfrac{1}{2}h, x_n + \dfrac{1}{2}hk_1\right), \\
            k_3 &=  f\left(t_n + \dfrac{1}{2}h, x_n + \dfrac{1}{2}hk_2\right), \\
            k_4 &=  f\left(t_n + h, x_n + hk_3\right). 
        \end{aligned}
    \end{equation}
\section*{The Forward Backward Sweep Method}
%[theorem name, {\citep[Thm. #, page]{reference}}]
%---------%---------%---------%---------%---------%---------%---------%---------
%---------%---------%---------%---------%---------%---------%---------%---------
    The Pontryagins principle allows us to transform the problem of finding a
    control which optimizes the cost functional to optimizing the Hamiltonian. 
    To this end, we need to solve forward in time the dynamics with a given 
    initial condition, and backward in time the adjoint equation with a 
    transversality condition, using the four-stage  RK method. This way of
    solving is called the forward backward sweep metho. So the idea is to 
    follow the next steps.
    
    \begin{description}
    	\item[Step 1.]
        	Make an initial guess for $\vec{u}$ over the interval.
        \item[Step 2.]
        	Using the initial condition $x_1 = x(t_0) = a$ and the values for 
            $\vec{u}$, solve $\vec{x}$ forward in time according to its differential
            equation in the optimality system.
    	\item[Step 3.]
        	Using the transversality condition $\lambda_{N+1} = \lambda(t_1) = 0$ 
            and the values for $\vec{u}$ and $\vec{x}$, solve $\vec{\lambda}$ 
            backward in time according to its differential equation in the optimality
            system.
        \item[Step 4.]
        	Update $\vec{u}$ by entering the new $\vec{x}$ and $\vec{\lambda}$ values 
            into the characterization of the optimal control. 
    	\item[Step 5.]
        	Check convergence. If the values of the variables in this iteration and 
            the last iteration and the last iteration are negligibly close, output the 
            current values as solutions. If values are not close, return to Step 2.
    \end{description}

    Using Python, we made an implementation of this method \citep{python_Thesisrepo},
    which follows the next  \Cref{FBSM_alg}. 

\begin{algorithm}
	\caption{Forward Backward Sweep }\label{FBSM_alg}
    INPUT: $t_0, t_f, n_{max}, x_0,h, a, r, m, \epsilon, \lambda_{f}$ \\
    OUTPUT: $x^*, u^*, \lambda$ \\
	\begin{algorithmic}[1]
		\Procedure{Forward backward sweep}{$g,\lambda_{\text{function}}, 
        u, x_0, 
        \lambda_f, h, n_{max}$} 
			\While{$ \text{test} > \epsilon $}
				\State $u_{\text{old}} \gets u$ 
                \State $x_{\text{old}} \gets x$ 
                \State $ x \gets$ \Call{runge\_kutta\_forward}%
                        {$g, u, x_0, h,n_{max}$}
                \State $\lambda_{\text{old}} \gets \lambda $
				\State $\lambda \gets$ \Call{runge\_kutta\_backward}%
				        {$\lambda_{\text{function}}, x, \lambda_f, h, n_{max}$}
                \State $\displaystyle u_1 \gets$ \Call{optimality\_condition} %
                        {$u, x, \lambda$}
                \State $\displaystyle u \gets \frac{u_1 + u_{old}}{2}$
                \State $test_1 \gets \displaystyle 
                \frac{||u - u_{\text{old}}||}{||u||}$
                \State $test_2 \gets \displaystyle 
                \frac{||x - x_{\text{old}}||}{||x||}$
                \State $test_3 \gets \displaystyle 
                \frac{||\lambda - \lambda_{\text{old}}||}{||\lambda||}$
                \State $\text{test} \gets \max{ \{ test_1, test_2, test_3 \}}$
			\EndWhile\label{}
			\State \textbf{return} $ x^*, u^*, \lambda$
            \Comment{Optimal pair}
		\EndProcedure
	\end{algorithmic}
\end{algorithm}
%---------%---------%---------%---------%---------%---------%---------%---------
%---------%---------%---------%---------%---------%---------%---------%---------

    Now, consider the following maximization problem.
    \begin{equation*}
        \max_{u} \int_{0}^{1} Ax(t) - B(u(t))^{2} dt
    \end{equation*}
    with $A \geq 0$, $B>0$, subject to the dynamics
    \begin{equation*}
        x'(t) = -\dfrac{1}{2}(x(t))^{2} + Cu(t), \, \, x(0) = x_0 >-2.
    \end{equation*}
    Note that the Hamiltonian 
    $$
        H = Ax - Bu^2 - \dfrac{1}{2}\lambda x^2 + C\lambda u.
    $$
    Using the optimility condition, 
    $$
        \dfrac{\partial H}{\partial u} = -2Bu + C\lambda
    $$
    we get $u^{*} = \dfrac{C\lambda}{2B}$. Calculating the adjoint equation we
    find
    \begin{align*}
        x'(t) &= -\dfrac{1}{2}x^2 + Cu, \, \, x(0) = x_0, \\
        \lambda'(t) = -A + x \lambda, \, \, \lambda(1) = 0.
    \end{align*}
    
    

    
    \todo{example 4.1}



%    \section{Lower and Upper Semicontinuous Functions} 
%---------%---------%---------%---------%---------%---------%---------%---------
%---------%---------%---------%---------%---------%---------%---------%---------

    Here we discuss about preliminary results from semicontinuous functions. The objective is to
    characterize the concept of semicontinuity and assure the existence of global minimum.
    Those concepts are defined over a metric space $(E,d)$. 
    
    Define the set of extended real numbers as
    $
        \extRealp := \{- \infty \} \cup \mathbb{R} \cup \{ + \infty \}
    $.
    This set of extended real numbers is ordered and we can define the following 
    operations additionally to the usual operations over $\mathbb{R}$
    \begin{align*}
        x \in \mathbb{R} \cup \{+\infty\} &\implies x + (+ \infty) = +\infty, \\
        x \in \mathbb{R} \cup \{-\infty\} &\implies x + (- \infty) = -\infty, \\
        x > 0 &\implies x(+ \infty) = + \infty, \\
        x < 0 &\implies x(+ \infty) = - \infty, \\
        (- \infty)(+ \infty) &= -\infty, \\
        (- \infty)(- \infty) &= (+ \infty)(+ \infty) = + \infty, \\
        0(+ \infty) &= 0(-\infty) = 0.
    \end{align*}
%---------%---------%---------%---------%---------%---------%---------%---------
    The semicontinuity of a function is defined by the limit inferior
    and superior of the function. So, we introduce the definition of
    limit inferior and superior for a given sequence. 
    \begin{definition}
        Let $\{ x_n \}$ be a sequence of extended real numbers, that is 
        $x_n \in \extRealp$. The limit inferior of $\{ x_n \}$ is
        $$
            \varliminf_{n \to \infty} x_n := \lim_{n \to \infty} \inf_{k \geq n}{x_{k}} %
            = \sup_{n} \inf_{k \geq n}{x_k},
        $$
        where the second equality follows since $\{ \inf_{k \geq n}{x_k} \}$ 
        is an increasing sequence in n. Similarly, the limit superior of 
        $\{ x_n \}$ is
        $$
            \varlimsup_{n \to \infty} x_n := \lim_{n \to \infty} \sup_{k \geq n}{x_{k}} %
            = \inf_{n} \sup_{k \geq n}{x_k}.
        $$
    \end{definition}
%---------%---------%---------%---------%---------%---------%---------%---------
    Now, the limit inferior and superior of a function is defined follows.
    \begin{definition}
        Let $f : E \to \extRealp $ be an extended real-valued function. The limit
        inferior of f as $x \in E$ converges to $x_0 \in E$  is defined by 
        $$
            \varliminf_{x \to x_0} f(x) := \lim_{\delta \to 0} %
            \inf_{d(x, x_0) < \delta} f(x) = \sup_{\delta\to 0 }\inf_{d(x, x_0) %
            < \delta} f(x),
        $$
        and its limit superior by 
        $$
            \varlimsup_{x \to x_0} f(x) := \lim_{\delta \to 0} %
            \sup_{d(x, x_0) < \delta} f(x) = \inf_{\delta} \to 0 \sup_{d(x, x_0) %
            < \delta} f(x).
        $$
    \end{definition}
%---------%---------%---------%---------%---------%---------%---------%---------
    Combining the above definition we establish the following lemma. 
%---------%---------%---------%---------%---------%---------%---------%---------
    \begin{lemma}\label{c1s5L1}
        Let $f : E \to \extRealp $. We have
        $$
            \varliminf_{x \to x_0} f(x) = \inf_{\{x_n\}}{\varliminf_{n \to \infty}f(x_n)},
        $$
        where the infimum on the right-hand side is taken over all sequences 
        $x_n \to x_0$. Similarly, 
        $$
            \varlimsup_{x \to x_0} f(x) = \sup_{\{x_n\}}{\varlimsup_{n \to \infty}f(x_n)}.
        $$
    \end{lemma}
%---------%---------%---------%---------%---------%---------%---------%---------
    \begin{proof}
        Define 
        \begin{align*}
            M &:= \varliminf_{x \to x_0} f(x),  \\
            L &:= \inf_{x_n}{\varliminf_{n \to \infty} f(x_n)}, \\
            N_{\delta} &:= \{ x \in E : d(x, x_0) < \delta \}.
        \end{align*}
        We have to consider the cases $M = - \infty$, $M = \infty$ and 
        $ -\infty < M < \infty $. \\
        \underline{Case $M = -\infty$:}
        Note that it is enough to prove the 
        existence of a sequence $x_n \to x_0$ such that $f(x_n) \to - \infty$. 
        Because, we can write
        $$
            \varliminf_{n \to \infty}{f(x_n)} = \lim_{n \to \infty} %
            \inf \{ f(x_n), f(x_{n+1}), \ldots \} = - \infty,
        $$
        and, any other sequence $y_n$ that converges to $x_0$ satisfies
        $\varliminf_{n \to \infty}f(y_n) \geq - \infty$, we get
        $$
            \inf_{x_n} \varliminf_{n \to \infty} f(x_n) = -\infty.
        $$
        Since, $M = -\infty$ we have that
        $$
            M = \varliminf_{x \to x_0} f(x) = \lim_{\delta \to 0} %
            \inf_{x \in N_{\delta}} f(x) = - \infty .
        $$
        Let $\delta = \frac{1}{n}$, for all $n \in \mathbb{N}$, then
        $$
            \inf_{x \in N_{1/n}} f(x) = -\infty, \forall n \in \mathbb{N}.
        $$
        Thus, there is $y_n \in N_{1/n}$ such that $f(y_n) < -n$. We take the 
        sequence defined by 
        $$
            x_n := y_n, \quad y_n \in  N_{1/n} \text{ for each } n \in \mathbb{N}.
        $$
        By the above, there is a sequence $x_n \to x_0$ such that 
        $f(x_n) \to - \infty$. Hence, $M = L$. \\
        \underline{Case $M = \infty$:} Since,
        $$
            \lim_{\delta} \inf_{x \in N_{\delta}} f(x) = \infty, 
        $$
        for a given $\epsilon > 0 $ there exists $\delta > 0 $ such that 
        $\inf_{x \in N_{\delta}} f(x) > \epsilon$.
        By the convergence of $x_n$, there is $N \in \mathbb{N}$ such that 
        $x_n \in N_{\delta}$ for all $n \geq N$. Then, 
        $$
            f(x_n) \geq \inf_{x_n \in N_{\delta}} f(x_n) > %
            \epsilon \quad \forall n \geq N
        $$ 
        That is, $f(x_n) \to \infty$ for any sequence $x_n \to x_0$. By the above
        we conclude $L = M$.
        \underline{Case $ -\infty < M < \infty$ :} \\
        First, we prove that $M \leq L$. By definition, given $\epsilon>0$ there
        is $\delta >0$ such that $\inf_{x \in N_{\delta}} f(x)$, this implies that
        $f(x) > M - \epsilon$ for all $x \in N_{\delta}$. Let $\{x_n\}$ be a sequence
        that converges to $x_0$. Thus, there is $N \in \mathbb{N}$ such that 
        $x_n \in N_{\delta}$ for all $n \in \mathbb{N}$ and
        $f(x_n) > M - \epsilon$ for all $x_n \in N_{\delta}$, $n \geq N$.
        Then
        $$
            \varliminf_{n \to \infty} f(x_n) = \lim_{n \to \infty} \inf_{x_n \in N_{\delta}} %
            \geq M - \epsilon.
        $$
        The above holds for any sequence $x_n \to x_0$, then
        $
            \inf_{\{x_n\}} \varliminf_{n \to \infty} f(x_n) \geq M - \epsilon,
        $
        for any $\epsilon > 0$. Hence, $M \leq L$.
        For the reverse inequality, we have that 
        $
            \inf_{x \in N_{\delta}} f(x) \to M,
        $
        as $\delta \to 0$. Let $\delta = \dfrac{1}{n}$. By definition of infimum
        there is $k_{\delta} \in \mathbb{N}$ such that $x_{k_{\delta}} \in N_{\delta}$ and
        \begin{equation*}
            f(x_{k_\delta}) \leq \inf_{ x_{k_\delta} \in N_{\delta}} %
                f(x_{k_\delta}) + \dfrac{1}{n}.
        \end{equation*}
        Thus, we choose $x_n \in N_{\frac{1}{n}}$ satisfying the above for each
        $n \in  \mathbb{N}$. Then
        \begin{align*}
            L   &=      \inf_{x_n} f(x_n )  \\
                &\leq   \varliminf_{n \to \infty} f(x_0) \\
                &\leq   \varliminf_{n \to \infty} %
                            \inf_{x_n \in N_{\frac{1}{n}}} f(x_n) + \dfrac{1}{n}\\
                &= M + 0.
        \end{align*}
        Hence $L \leq M$. Finally, notice that
        $
            \varlimsup_{n \to \infty} y_n = - \varliminf_{n \to \infty}(- y_n),
        $
        and
        $
            \sup_{x \in A} f(x) = - \inf_{x \in A}(- f(x)),
        $
        implies the second assertion.
    \end{proof}
%---------%---------%---------%---------%---------%---------%---------%---------
    To fix ideas, let $f(x):= \sin(x)$, then $\liminf f(x) = -1$ and 
    $\limsup f(x)= 1$ when $x \to 0$ because the minimum of all the accumulation
    points is $-1$  and the maximum is $1$. 
    \begin{figure}[htb] 
        \begin{center}
            \includegraphics[width=0.5\textwidth,keepaspectratio]%
                {Chapters/Chapter1_Ekeland/Figures/FunctionSin.png}
            \caption{%
                The blue dots are the maximum value of the function $f$ and the
                red diamonds represents the minimum values of $f$.}%
            \label{Example_FSin}
        \end{center}
    \end{figure}
    The following definitions describe the lower and upper semicontinuity of a 
    function.
    \begin{definition}
        Let $f: E \to \extRealp$. The function $f$ is lower semicontinuous ($\lsc$)  at a point 
        $x_0 \in E$ if 
        \begin{equation*}
            f(x_0) \leq \varliminf_{x \to x_0} f(x).
        \end{equation*}
        Equivalently, by Lemma $\eqref{c1s5L1}$, f is $\lsc$ at $x_0$ if 
        $f(x_0) \leq \varliminf_{n \to \infty} f(x_n)$, for every
        sequence $x_n \to x_0$.
    \end{definition}
%---------%---------%---------%---------%---------%---------%---------%---------
    \begin{definition}
        The function $f$ is upper semicontinuous ($\usc$) at a point $x_0$ if
        \begin{equation*}
            f(x_0) \leq \varlimsup_{x \to x_0} f(x).
        \end{equation*}
        Equivalently, f is $\usc$ at $x_0$ if 
        $f(x_0) \leq \varlimsup_{n \to \infty} f(x_n)$, for every sequence 
        $x_n \to x_0$.
    \end{definition}
    
    \Cref{fig:Example_LSCF} and \Cref{fig:Example_USCF}, presents examples of
    functions that are $\lsc$ and upper semicontinuous, respectively. 

    \begin{figure}[htb]
        \centering
        \begin{subfigure}[b]{0.4\textwidth}
            \centering
            \includegraphics[width=\textwidth,keepaspectratio]%
                {Chapters/Chapter1_Ekeland/Figures/Example_LSCF.png}
            \caption{Lower}
            \label{fig:Example_LSCF}
        \end{subfigure}
        \hfill
        \begin{subfigure}[b]{0.4\textwidth}
            \centering
            \includegraphics[width=\textwidth,keepaspectratio]%
                {Chapters/Chapter1_Ekeland/Figures/Example_USCF.png}
            \caption{Upper}
            \label{fig:Example_USCF}
         \end{subfigure}
            \caption{Semicontinuous Functions}
            \label{fig:Example_LUSCF}
    \end{figure}
    
    
    We say that $f$ is $\lsc$  or closed on $E$ if $f$ is $\lsc$
    at every point $x \in E$. Similarly, $f$ is $\usc$ on $E$ 
    if $f$ is $\usc$ at every point in $E$. Also, note that 
    \begin{equation*}
        \varliminf_{x \to x_0} f(x) \leq f(x_0), \, \, %
        \varlimsup_{x \to x_0} f(x) \geq f(x_0),
    \end{equation*}
    since $x_0$ lies in every neighborhood $N_{\delta}$. This implies, 
    $f$ is lsc at $x_0$ if and only if
    $$
        f(x_0) = \varliminf_{x \to x_0} f(x),
    $$
    and, $f$ is usc at $x_0$ if and only if
    $$
        f(x_0) = \varlimsup_{x \to x_0} f(x).
    $$
    Also, note that any function is $\lsc$ at a point $x$ if $f(x) = - \infty$
    and similarly any function is $\usc$ at a point $x$ if $f(x) = \infty$. 
    
%---------%---------%---------%---------%---------%---------%---------%---------
    \noindent Since the $\lsc$ and $\usc$ functions are, respectively, related
    with its epigraph and hypograph, we enunciate its definitions.
    \begin{definition}
        Let $f : E \to \extRealp$ be a function. We define the epigraph of $f$
        as the set
        \begin{equation*}
            \mathtt{epi}(f) := \{ (x, t) \in E \times \mathbb{R} : f(x) \leq t \}.
        \end{equation*}
        Similarly, define the hypograph of $f$ as
        \begin{equation*}
            \mathtt{hypo}(f) := \{ (x,t) \in E \times \mathbb{R} : f(x) \geq t \}.
        \end{equation*}
    \end{definition}
    
    \noindent The following result relates the semicontinuity of a functions, 
    with its $\epi$ or $\hypo$.
    
    \begin{theorem}
        Let $f: E \to \extRealp$. The following statements are equivalent:
        \begin{asparaenum}[a)]
            \item
                The function $f$ is $\lsc$ ($\usc$) on E,
            \item
                The set $\text{epi}(f) \ (\text{hypo}(f))$ is a closed subset of 
                $ E \times \mathbb{R}$,
            \item
                The sublevel set 
                \begin{align*}
                    & l_{\alpha}(f) := \{ x \in E : f(x) \leq \alpha \}, \\
                    & (l^{\alpha}(f) := \{ x \in E : f(x) \geq \alpha \})
                \end{align*}
                is closed for all $\alpha \in \mathbb{R}$.
        \end{asparaenum}
    \end{theorem}
%---------%---------%---------%---------%---------%---------%---------%---------
%---------%---------%---------%---------%---------%---------%---------%---------
    \begin{proof}
    We have to proof the following implications 
    \textbf{$a) \Rightarrow b)$} \\
        Suppose that $f$ is a $\lsc$ function on $E$. Let 
        $(x_n, t_n)$ be a sequence on $\mathtt{epi}(f)$ converging to a point 
        $(x,t)$. Since $f$ is l.s.c. at $x$,
        \begin{equation}\label{thm2.39.a1}
            f(x) \leq \varliminf f(x_n).
        \end{equation}
        On the other hand , since $(x_n, t_n) \in \mathtt{epi}(f)$ 
        \begin{equation}\label{thm2.39.a2}
            f(x_n) \leq t_n.
        \end{equation}
        for all $n \in \mathbb{N}$. Combining $\eqref{thm2.39.a1}$ and
        $\eqref{thm2.39.a2}$ we get 
        $$
            f(x) \leq \varliminf f(x_n) \leq \lim f(x_n) \leq \lim t_n = t.
        $$
        Thus, $(x,t) \in \mathtt{epi}(f)$. Hence $\mathtt{epi}(f)$ is closed.
        
    \textbf{$b) \Rightarrow c)$}
        Suppose that $\mathtt{epi}(f)$ is closed and let $(x_n)$ be a sequence in
        $l_{\alpha}(f)$ converging to a point $x \in E$. Note that 
        $(x_n, \alpha) \in \mathtt{epi}(f)$ and its limit $(x, \alpha) \in \mathtt{epi}(f)$. 
        Hence $x \in l_{\alpha}(f)$.
        
    \textbf{$c) \Rightarrow a)$}
        First, we consider $-\infty < f(x) < \infty$. For this case we proceed by
        contradiction, assuming that the sublevel set $l_{\alpha}(f)$  
        is closed for all $\alpha \in \mathbb{R}$ and $f$ is not $\lsc$.
        Then, there is $\epsilon > 0$ such that
        $$
            \varliminf_{x \to x_0} f(x) = \sup_{\delta >0} \inf_{N_{\delta}} f(x)
            = f(x) - 2\epsilon.
        $$
        Thus, for any $\delta > 0$, we have 
        $
            \inf_{N_\delta} f(x) \leq f(x) - 2\epsilon < f(x_0) - \epsilon.
        $
        Let $\alpha = f(x_0) - \epsilon$. Define the sequence $\{x_n\}$ as 
        follows $x_n \in N_{\delta}(x_0)$, $x_n \to x_0$,
        such that $f(x_n) \leq f(x_0) - \epsilon = \alpha $. That is, 
        $x_n \in l_{\alpha}(f)$. Since $l_{\alpha}(f)$ is closed, 
        $x_0 \in l_{\alpha}(f)$ and 
        $
            f(x_0) < f(x_0) - \epsilon,
        $
        which is a contradiction. Now, if $f(x) = -\infty$ for a point $x \in E$, then
        $f$ is lsc. Now, consider the last case, $f(x) = \infty$ for some $x \in E$. 
        Proceeding by contradiction, suppose that $f$ is not $\lsc$ at 
        $x_0 \in E$ and $f(x_0) = \infty$, so there is $\alpha \in \mathbb{R}$ 
        such that $\sup_{\delta > 0} \inf_{x \in N_{\delta}} f(x) = \alpha$. 
        For any $\delta >0$, $\inf_{x \in N_{\delta}} \leq \alpha$. Let 
        $\beta \in \mathbb{R}$ such that $\alpha < \beta$. Thus, there exist a 
        sequence $\{x_n\} \subset N_{\delta}$ that converges to $x_0$ and $f(x_n) %
        \leq \alpha < \beta$. Since $l_{\alpha}(f)$ is closed, 
        $x_0 \in l_{\alpha}(f)$ and $\infty = f(x_0) \leq \beta$,
        which is a contradiction. 
    \end{proof}
%---------%---------%---------%---------%---------%---------%---------%---------
%---------%---------%---------%---------%---------%---------%---------%---------
    \noindent Now, we prove that the family of $\lsc$ functions is closed 
    under the sum 
    \begin{corollary}
        If the functions $f, \ g \ : E \to \mathbb{R} \cup \{+ \infty\}$ are 
        $\lsc$, then so is $f + g$.
    \end{corollary}
    \begin{proof}
        We have to prove that $\{x \in E : f(x) + g(x) \leq t \}$ is closed. 
        We claim that
        \begin{equation}\label{Cor2.40}
            \{x \in E : f(x) + g(x) > t \} = \bigcup_{\alpha \in \mathbb{R}} %
                \{x \in E : f(x) > t - \alpha \} \cap \{x \in E : g(x) >  \alpha\},
        \end{equation}
        We prove the following contentions:
        \begin{asparaenum}
            \item["$\subseteq$".]
                Let $x \in \{x \in E : f(x) + g(x) > t \} $ then, there is
                $\epsilon > 0$ and some $\alpha \in \mathbb{R}$, such that
                $
                    f(x) + g(x) = t + 2\epsilon > t,
                $
                and 
                $
                    g(x) = \alpha + \epsilon > \alpha.
                $
                Thus 
                $$
                    f(x) = t + 2\epsilon - \alpha - \epsilon = t - \alpha %
                    + \epsilon > t - \alpha.
                $$
            \item["$\supseteq$".] Let
            $$
                x \in \bigcup_{\alpha \in \mathbb{R}} %
                \{x \in E : f(x) > t - \alpha \} \cap \{x \in E : g(x) >  \alpha\}.
            $$
            Then 
            $
                f(x) + g(x) > t -\alpha + \alpha = t.
            $
        \end{asparaenum}
        So, the equality is proved. In the right-hand side of the equality \eqref{Cor2.40} we have the
        arbitrary union of open sets, this implies that
        $
            \{x \in E : f(x) + g(x) > t \},
        $
        is open. Thus, the complement
        $
            \{x \in E : f(x) + g(x) \leq t \},
        $
        is a closed set. By Theorem \eqref{thm2.41}, $f + g $ is lsc.
    \end{proof}
%---------%---------%---------%---------%---------%---------%---------%---------
%---------%---------%---------%---------%---------%---------%---------%---------
    The following results establish conditions to assure existence of a global
    minimum for $\lsc$ functions.
    
    \begin{theorem} \label{thm2.41}
        Let $f : E \to \mathbb{R} \cup \{+ \infty\}$ are $\lsc$, 
        defined on a metric space $E$. If $f$ has a nonempty compact sublevel
        set $l_{\alpha}(f)$, then $f$ achieves its global minimum on $E$. 
    \end{theorem}
    \begin{proof}
        Let $\{x_n\}_{n = 1}^{\infty}$ be a minimizing sequence for $f$, that is
        $$
            f(x_n) \searrow \inf\{f(x) : x \in E \}.
        $$
        Define 
        $$
            \inf_{E} f := \{ f(x) : x \in E \} .
        $$
        Since $f(x_n) \searrow \inf_{E} f $ there is $N \in \mathbb{N}$ such that
        $x_n \ in l_{\alpha}(f)$ for all $n \geq N$, that is
        $$
            f(x_n) \leq \alpha ,
        $$
        for all $n \geq N$. By hypothesis $l_{\alpha}(f)$ is a compact set, this implies
        that $\{x_n\}_{n=N}^{\infty}$ has a convergent subsequence
        $$
            x_{n_k} \to x^{*} \in l_{\alpha}(f).
        $$
        Since f is $\lsc$ at $x^{*}$, we have 
        $$
            \inf_{E} f \leq f(x^*) \leq \varliminf_{x_{n_k} \to x^{*}} f(x_{n_k}) %
            = \varliminf_{n \to \infty} f(x_n) = \lim_{n \to \infty} f(x_n) = \inf_{E} f.
        $$
        This means that $f(x^{*}) = \inf_{E} f$, that is, $f$ achieves its minimum
        on $E$ at the point $x^{*}$. If 
        $$
            \alpha = \inf_{x \in E} f(x),
        $$
        then
        $$
            l_{\alpha}(f) = \left\{ x \in E : f(x) \leq \inf_{x \in E} f(x) \right\}.
        $$
        By the definition of infimum we have that 
        $$
            f(x) = \inf_{x \in E} f(x).
        $$
        Thus, any element of $l_{\alpha}(f)$ is a global minimum. 
    \end{proof}

    \begin{definition}
        A function $f : D \to \mathbb{R}$ on a subset $D$ of a normed vector space $E$
        is called coercive if 
        $$
            f(x) \to \infty \text{ as } \norm{x} \to + \infty.
        $$
    \end{definition}
    
    \begin{corollary}
        Let $f : K \to \mathbb{R}$ be a $\lsc$ function defined on
        a metric space $K$.
        \begin{asparaenum}
            \item[a)]
                If $K$ is compact, or
            \item[b)] $K$ is a subset of a finite-dimensional normed vector space
                        $E$ and $f$ is coercive,
        \end{asparaenum}
        then $f$ achieves a global minimum on $K$.
    \end{corollary}
    \begin{proof} \hspace{5cm} \\
        \begin{asparaenum}
            \item[a)]
                Note that all sublevel sets are closed because $f$ is $\lsc$.
                Also, we know that all closed subsets of a compact set are compact. Therefore,
                the Theorem \eqref{thm2.41} implies that $f$ achieves its global minimum on $K$.
            \item[b)]
                Consider the sublevel set
                $l_{\alpha}(f) = \{ x \in D : f(x) \leq \alpha \}$. Since $f$ is
                coercive there is $\delta_{\alpha} > 0$ such that
                $\norm{x} > \delta_{\alpha}$, which implies $f(x) > \alpha $.
                Then, for any $y \in l_{\alpha}(f)$, $\norm{y} \leq %
                \delta_{\alpha}$. Thus, the sublevel set $l_{\alpha}(f)$
                is bounded and also close because $f$ is $\lsc$. Hence
                $l_{\alpha}(f)$ is compact and the conclusion follows from
                \Cref{thm2.41}.
        \end{asparaenum}
    \end{proof}
    
    For the next section we consider a proper $\lsc$ function with values in 
    $\extRealp$, since the Ekeland principle we present is established for this 
    kind of function. 
%---------%---------%---------%---------%---------%---------%---------%---------
%---------%---------%---------%---------%---------%---------%---------%---------

%    \section{The Forward Backward Sweep Method}
%[theorem name, {\citep[Thm. #, page]{reference}}]
%---------%---------%---------%---------%---------%---------%---------%---------
%---------%---------%---------%---------%---------%---------%---------%---------
    The Pontryagins principle allows us to transform the problem of finding a
    control which optimizes the cost functional to optimizing the Hamiltonian. 
    To this end, we need to solve forward in time the dynamics with a given 
    initial condition, and backward in time the adjoint equation with a 
    transversality condition, using the four-stage  RK method. This way of
    solving is called the forward backward sweep metho. So the idea is to 
    follow the next steps.
    
    \begin{description}
    	\item[Step 1.]
        	Make an initial guess for $\vec{u}$ over the interval.
        \item[Step 2.]
        	Using the initial condition $x_1 = x(t_0) = a$ and the values for 
            $\vec{u}$, solve $\vec{x}$ forward in time according to its differential
            equation in the optimality system.
    	\item[Step 3.]
        	Using the transversality condition $\lambda_{N+1} = \lambda(t_1) = 0$ 
            and the values for $\vec{u}$ and $\vec{x}$, solve $\vec{\lambda}$ 
            backward in time according to its differential equation in the optimality
            system.
        \item[Step 4.]
        	Update $\vec{u}$ by entering the new $\vec{x}$ and $\vec{\lambda}$ values 
            into the characterization of the optimal control. 
    	\item[Step 5.]
        	Check convergence. If the values of the variables in this iteration and 
            the last iteration and the last iteration are negligibly close, output the 
            current values as solutions. If values are not close, return to Step 2.
    \end{description}

    Using Python, we made an implementation of this method \citep{python_Thesisrepo},
    which follows the next  \Cref{FBSM_alg}. 

\begin{algorithm}
	\caption{Forward Backward Sweep }\label{FBSM_alg}
    INPUT: $t_0, t_f, n_{max}, x_0,h, a, r, m, \epsilon, \lambda_{f}$ \\
    OUTPUT: $x^*, u^*, \lambda$ \\
	\begin{algorithmic}[1]
		\Procedure{Forward backward sweep}{$g,\lambda_{\text{function}}, 
        u, x_0, 
        \lambda_f, h, n_{max}$} 
			\While{$ \text{test} > \epsilon $}
				\State $u_{\text{old}} \gets u$ 
                \State $x_{\text{old}} \gets x$ 
                \State $ x \gets$ \Call{runge\_kutta\_forward}%
                        {$g, u, x_0, h,n_{max}$}
                \State $\lambda_{\text{old}} \gets \lambda $
				\State $\lambda \gets$ \Call{runge\_kutta\_backward}%
				        {$\lambda_{\text{function}}, x, \lambda_f, h, n_{max}$}
                \State $\displaystyle u_1 \gets$ \Call{optimality\_condition} %
                        {$u, x, \lambda$}
                \State $\displaystyle u \gets \frac{u_1 + u_{old}}{2}$
                \State $test_1 \gets \displaystyle 
                \frac{||u - u_{\text{old}}||}{||u||}$
                \State $test_2 \gets \displaystyle 
                \frac{||x - x_{\text{old}}||}{||x||}$
                \State $test_3 \gets \displaystyle 
                \frac{||\lambda - \lambda_{\text{old}}||}{||\lambda||}$
                \State $\text{test} \gets \max{ \{ test_1, test_2, test_3 \}}$
			\EndWhile\label{}
			\State \textbf{return} $ x^*, u^*, \lambda$
            \Comment{Optimal pair}
		\EndProcedure
	\end{algorithmic}
\end{algorithm}
%---------%---------%---------%---------%---------%---------%---------%---------
%---------%---------%---------%---------%---------%---------%---------%---------

    Now, consider the following maximization problem.
    \begin{equation*}
        \max_{u} \int_{0}^{1} Ax(t) - B(u(t))^{2} dt
    \end{equation*}
    with $A \geq 0$, $B>0$, subject to the dynamics
    \begin{equation*}
        x'(t) = -\dfrac{1}{2}(x(t))^{2} + Cu(t), \, \, x(0) = x_0 >-2.
    \end{equation*}
    Note that the Hamiltonian 
    $$
        H = Ax - Bu^2 - \dfrac{1}{2}\lambda x^2 + C\lambda u.
    $$
    Using the optimility condition, 
    $$
        \dfrac{\partial H}{\partial u} = -2Bu + C\lambda
    $$
    we get $u^{*} = \dfrac{C\lambda}{2B}$. Calculating the adjoint equation we
    find
    \begin{align*}
        x'(t) &= -\dfrac{1}{2}x^2 + Cu, \, \, x(0) = x_0, \\
        \lambda'(t) = -A + x \lambda, \, \, \lambda(1) = 0.
    \end{align*}
    
    

    
    \todo{example 4.1}

\newpage