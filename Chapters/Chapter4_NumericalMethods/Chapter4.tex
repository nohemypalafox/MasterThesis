%
\chapter{The Forward-Backward Sweep Method}
\lhead{\emph{Numerical Methods}}
%---------%---------%---------%---------%---------%---------%---------%---------
%---------%---------%---------%---------%---------%---------%---------%---------
%           MODIFICAR :
    The purpose of this chapter is to present and explain the four-stage Runge 
    Kutta (RK) method and the forward backward sweep method. Both of them are 
    used to find the optimal control. Also, we present examples of a one-stage 
    and two-stage explicit RK method, and an example of how to apply the
    forward-backward sweep method. For a more extensive explanation see
    \cite{griffiths2010numerical}.
%%%%%%%%%       \section*{The Runge Kutta Method}
%---------%---------%---------%---------%---------%---------%---------%---------

%---------%---------%---------%---------%---------%---------%---------%---------
    \noindent Consider the following initial value problem (IVP)
    \begin{equation}\label{eq:ivp}
        \begin{aligned}
            x'(t)  &= f(t, x(t)), \hspace{1cm} t \geq t_0, \\
            x(t_0) &= x_0,
        \end{aligned}
    \end{equation}
%       MODIFICAR: 
    The Runge-Kutta (RK) method is used to approximate the solution of the IVP 
    described by \cref{eq:ivp}. This method is iterative and consists of a number
    of stages. For example, Euler's method is a one-stage RK method. The RK 
    method computes the slopes of nearby points at a time $t_n$ and then 
    calculates the mean of these slopes to approximate the solution at the next
    time $t_{n+1}$. The general $s$-stage RK method is defined as
    \begin{equation}\label{eq:GenSSRK}
        x_{n+1} = x_n + h \sum_{i=1}^{s} b_i k_i,
    \end{equation}
    where the terms $k_i$ are computed from the following evaluation in the 
    right-hand side of the IVP  \eqref{eq:ivp}:
    \begin{equation}\label{eq:kterms}
        k_i = f \left( t_n + c_i h, x_n + h \sum_{j=1}^{s} a_{i,j}k_j\right), %
                \hspace{1cm} i = 1, \cdots s,
    \end{equation}
    with
    \begin{equation}
        c_i =  \sum_{j=1}^{s} a_{i,j}, \hspace{1cm} i = 1, \cdots, s.
    \end{equation}
%---------%---------%---------%---------%---------%---------%---------%---------
%---------%---------%---------%---------%---------%---------%---------%---------
    Thus, given a value of $s$, the method depends on $s^2 +s$ parameters
    $\{a_{i,j},b_{j}\}$. These parameters can be displayed in a tableau known as the
    Butcher array (see \cref{tab:GButcherArray}).
    \begin{table}[h!]\label{tab:GButcherArray}
        \begin{center}
            \begin{tabular}{l | l } 
                c  & A  \\
                \hline
                \hspace{0.1cm} & b
            \end{tabular}
            \caption{General form of a Butcher array.}
        \end{center}
    \end{table}
%---------%---------%---------%---------%---------%---------%---------%---------
%---------%---------%---------%---------%---------%---------%---------%---------
    \noindent Vector $c$ indicates the positions within the step of the 
    stage values. The matrix $A$ indicates the dependence of the stages on the
    derivatives found at other stages. And $b$ is a vector of quadrature weights,
    showing how the final result depends on the derivatives computed at various
    stages. In the explicit RK methods the upper-triangular components of the 
    matrix $A$ are zero. So, in Table \cref{tab:ExplicitBA} is shown an 
    extended Butcher array for an explicit RK method. 
    \begin{table}[h!]\label{tab:ExplicitBA}
        \begin{center}
            \begin{tabular}{l | l l l l l} 
                0     & 0 & 0 & $\cdots$ &0 &    0 \\
                $c_2$     & $a_{2,1}$ & 0 & $\cdots$ &0 & 0 \\
                $c_3$     & $a_{3,1}$ & $a_{3,2}$ & $\cdots$ & 0& $\vdots$ \\
                $\vdots$  & $\vdots$  & $\vdots$  &          &0 & 0 \\
                $c_s$     & $a_{s,1}$ & $a_{s,2}$ & $\cdots$ & $a_{s,s-1}$ & 0 \\
                \hline
                \hspace{0.1cm} & $b_1$ & $b_2$ & $\cdots$ & $_{s-1}$ & $b_s$ \\
            \end{tabular}
            \caption{The Butcher array for an explicit RK method}
        \end{center}
    \end{table}
%---------%---------%---------%---------%---------%---------%---------%---------
%---------%---------%---------%---------%---------%---------%---------%---------
    Let us introduce the definition of the local truncation error. 
    \begin{definition}[Local Truncation Error, {\citep[def. 9.3, p. 127]{griffiths2010numerical}}]
        The Local Truncation Error, $T_{n+1}$ of an RK method is defined to be
        the difference between the exact and the numerical solution of the IVP at
        time $t = t_{n+1}$:
        $$
            T_{n+1} = x(t_{n+1}) - x_{n+1},
        $$
        under the localizing assumption that $x_n = x(t_n)$. If $T_{n+1} = \mathcal{O}^{p+1}$,
        the method is said to be of order $p$.
    \end{definition}
    We obtain the parameters in the RK methods, using the definition below. To clarify this idea, we present an example of a one-stage RK method, that is,
    $s = 1$. Following the definitions \eqref{eq:GenSSRK} and \cref{eq:kterms},
    we get
    $
        x_{n+1} = x_n + hb_1 k_1,
    $
    and
    $
        k_1 = f(t_n + c_1 h, x_n + h a_{1,1} k_1).
    $
    By the definition of $c_i$ we have that, $c_1 = a_{1,1} = 0$. Also, 
    $k_1 = f(t_n,x_n) = f_n$, then $ x_{n+1} = x_n + h b_1 f_n$. To find the
    coefficient $b_1$, we have to compare this expression with the expression for
    $x(t_{n+1})$:
    $$
        x(t_{n+1}) = x(t_n) + hx'(t_n) + \dfrac{1}{2}h^{2}x''(t_n) + \mathcal{O}(h^{3}).
    $$
    To do this comparison, we need to differentiate the equation $x'(t) = f(t,x(t))$, 
    with respect to t. By the chain rule, we obtain
    $$ 
        x''(t) = \dfrac{\partial f}{\partial t}t' + %
                 \dfrac{\partial f}{\partial x}  x'(t) %
            = f_{t} + f_{x} f.
    $$
    Then, 
    \begin{align*}
        T_{n+1}
            &= x(t_{n+1}) - x_{n+1} \\
            &= x(t_n) + hx'(t_n) + \dfrac{1}{2}h^2 x''(t_n)  - x_n - hb_1 f_n +  \mathcal{O}(h^3)\\
            &= x_n + hf_n + \dfrac{1}{2}h^2(f_t + ff_x)\vert_{t = t_n} - x_n - hb_1f_n + \mathcal{O}(h^3)\\
            &= h(1-b_1)f_n + \dfrac{1}{2}h^2(f_t + ff_x)\vert_{t = t_n} + \mathcal{O}(h^3),
    \end{align*}
    Choosing $b_1 = 1$ the method is consistent of order $p=1$. This choosing 
    gives the only first-order one-stage explicit RK method, the Euler's method
    $$
        x_{n+1} = x_n + hf_n.
    $$
%---------%---------%---------%---------%---------%---------%---------%---------

%---------%---------%---------%---------%---------%---------%---------%---------
    To fix ideas, consider the next example for the two stage RK method, which 
    is a modified Euler method. Let the IVP
    \begin{equation}\label{eq:IVP2}
        \begin{aligned}
            x'(t) &= (1-2t)x(t), \hspace{1cm} t>0, \\
            x(0)  &= 1.
        \end{aligned}
    \end{equation}
%---------%---------%---------%---------%---------%---------%---------%---------
%---------%---------%---------%---------%---------%---------%---------%---------
    According to the two stage RK method, 
    \begin{align*}
        t_{n+1} &= t_n + h, \\
        x_{n+1} &= x_n + hk_2, 
    \end{align*}
    and
    \begin{equation}\label{eq:IVP2_kterms}
        \begin{aligned}
            k_1 &= f(t_n, x_n), \\
            k_2 &= f\left(t_n + \dfrac{1}{2}h, x_n + \dfrac{1}{2}hk_1 \right),
        \end{aligned}
    \end{equation}
    we calculate the approximate solution of \ref{eq:IVP2} choosing $h=0.2$.
    Thus, substituting \ref{eq:IVP2_kterms} in \ref{eq:IVP2} we obtain
    \begin{align*}
        k_1 &= (1- 2t_n)x_n,\\
        k_2 &= (1 - (2t_n + h))\left(x_n + \dfrac{1}{2}h k_1\right).
    \end{align*}
    So, for $n = 0$, we have
    \begin{align*}
        k_1 &= (1-2t_0)x_0 = (1 - 2(0))(1) = 1,    \\ 
        k_2 &= (1 - 2t_0 - h)(x_0 + 0.5hk_1) = (1-0.2)(1+0.5(0.2)(1)) = 0.88, \\
        t_1 &= t_0 + h = 0 + 0.2 = 0.2, \\
        x_1 &= x_0 + hk_2 = 1 + (0.2)(0.88) = 1.176.
    \end{align*}
    Now, for $n=1$, yields
    \begin{align*}
        k_1 &= (1-2t_1)x_1 = (1 - 2(0.2))(1.176) = 0.7056,    \\ 
        k_2 &= (1 - 2t_1 - h)(x_1 + 0.5hk_1) =
            (1-2(0.2)-0.2)(1.176+0.5(0.2)(0.7056)) = 0.4986, \\
        t_2 &= t_1 + h = 0.2 + 0.2 = 0.4, \\
        x_2 &= x_1 + hk_2 = 1.176 + (0.2)(0.4986) = 1.2757,
    \end{align*}
    and so on. This is an example of an explicit RK method.
    
%    -> here goes a tiny graph <3
%---------%---------%---------%---------%---------%---------%---------%---------
%---------%---------%---------%---------%---------%---------%---------%---------
    Now, letting $s=4$, we get a four-stage RK explicit method for the IVP 
    (\ref{eq:ivp})
    \begin{equation}\label{RK4_method}
        x_{n+1} = x_n + h(b_{1}k_{1} + b_{2}k_{2} + b_{3}k_{3} + b_{4}k_{4})
    \end{equation}
    where
    \begin{equation}\label{RK4_kterms}
        \begin{aligned}
            k_1 &=  f(t_n, x_n), \\
            k_2 &=  f(t_n + c_2 h, x_n + h a_{2,1}k_1), \\
            k_3 &=  f\left(t_n + c_3 h, x_n + h\sum_{j=1}^{2}a_{3,j}k_j\right), \\
            k_4 &=  f\left(t_n + c_4 h, x_n + h\sum_{j=1}^{3}a_{1,j}k_j\right),
        \end{aligned}
    \end{equation}
    To obtain the parameters we have to satisfy the following conditions
    \begin{align*}
        b_1 + b_2 + b_3 + b_4 &= 1, \\
        b_2 c_2 + b_3 c_3 + b_4 c_4 &= \dfrac{1}{2}, \\
        b_2 {c_2}^{2} + b_3{c_3}^{2} + b_4{c_4}^{2} &= \dfrac{1}{3}, \\
        b_3 a_{3,2}c_2 + b_4 a_{4,2}c_2 + b_4 a_{4,3}c_3 &= \dfrac{1}{6}, \\
        b_2 {c_2}^{3} + b_3{c_3}^{3} + b_4{c_4}^{3} &= \dfrac{1}{4}, \\
        b_3 c_3 a_{3,2}c_2 + b_4 c_4 a_{4,2}c_2 + b_4 c_4 a_{4,3}c_3 &= \dfrac{1}{8}, \\
        b_3 a_{3,2}{c_2}^{2} + b_4 a_{4,2}{c_2}^{2} + b_4 a_{4,3}{c_3}^{2} &= \dfrac{1}{12}, \\
        b_4 a_{4,3}a_{3,2} c_2 &= \dfrac{1}{24}.
    \end{align*}
%---------%---------%---------%---------%---------%---------%---------%---------
%---------%---------%---------%---------%---------%---------%---------%---------
    There is a lot of solutions and families of solutions. We show in Table \cref{tab:BA-RK4} some parameters that solves the equations
    below. This set of parameters conform the Runge Kutta-Felberg method
    of fourth order. This method is common used in package from R, Julia and
    MATLAB. 
    
    \begin{table}[h!]\label{tab:BA-RK4}
        \begin{center}
            \begin{tabular}{l | l l l l l} 
                0     & 0 &  &  &     \\
                $\frac{1}{2}$ & $\frac{1}{2}$ & 0 &  & \\
                $\frac{1}{2}$ & 0 & $\frac{1}{2}$ & 0 & \\
                1  & 0  &  0  & 1 & 0 \\
                \hline
                \hspace{0.1cm} & $\frac{1}{6}$ & $\frac{1}{3}$ & $\frac{1}{3}$ & $\frac{1}{6}$  \\
            \end{tabular}
            \caption{The Butcher array for an explicit four stage RK method}
        \end{center}
    \end{table}
    
    \noindent Substituting the parameters showed in Table \cref{tab:BA-RK4} 
    in (\ref{RK4_method}) and (\ref{RK4_kterms}), we obtain the Runge Kutta-Felberg method
    \begin{equation}\label{RK4_method}
        x_{n+1} = x_n + \dfrac{1}{6}h(k_{1} + 2k_{2} + 2k_{3} + k_{4})
    \end{equation}
    where
    \begin{equation}\label{RK4_kterms}
        \begin{aligned}
            k_1 &=  f(t_n, x_n), \\
            k_2 &=  f\left(t_n + \dfrac{1}{2}h, x_n + \dfrac{1}{2}hk_1\right), \\
            k_3 &=  f\left(t_n + \dfrac{1}{2}h, x_n + \dfrac{1}{2}hk_2\right), \\
            k_4 &=  f\left(t_n + h, x_n + hk_3\right). 
        \end{aligned}
    \end{equation}
\section*{The Forward Backward Sweep Method}
%[theorem name, {\citep[Thm. #, page]{reference}}]
%---------%---------%---------%---------%---------%---------%---------%---------
%---------%---------%---------%---------%---------%---------%---------%---------
    The Pontryagins principle allows us to transform the problem of finding a
    control which optimizes the cost functional to optimizing the Hamiltonian. 
    To this end, we need to solve forward in time the dynamics with a given 
    initial condition, and backward in time the adjoint equation with a 
    transversality condition, using the four-stage  RK method. This way of
    solving is called the forward backward sweep metho. So the idea is to 
    follow the next steps.
    
    \begin{description}
    	\item[Step 1.]
        	Make an initial guess for $\vec{u}$ over the interval.
        \item[Step 2.]
        	Using the initial condition $x_1 = x(t_0) = a$ and the values for 
            $\vec{u}$, solve $\vec{x}$ forward in time according to its differential
            equation in the optimality system.
    	\item[Step 3.]
        	Using the transversality condition $\lambda_{N+1} = \lambda(t_1) = 0$ 
            and the values for $\vec{u}$ and $\vec{x}$, solve $\vec{\lambda}$ 
            backward in time according to its differential equation in the optimality
            system.
        \item[Step 4.]
        	Update $\vec{u}$ by entering the new $\vec{x}$ and $\vec{\lambda}$ values 
            into the characterization of the optimal control. 
    	\item[Step 5.]
        	Check convergence. If the values of the variables in this iteration and 
            the last iteration and the last iteration are negligibly close, output the 
            current values as solutions. If values are not close, return to Step 2.
    \end{description}

    Using Python, we made an implementation of this method \citep{python_Thesisrepo},
    which follows the next  \Cref{FBSM_alg}. 

\begin{algorithm}
	\caption{Forward Backward Sweep }\label{FBSM_alg}
    INPUT: $t_0, t_f, n_{max}, x_0,h, a, r, m, \epsilon, \lambda_{f}$ \\
    OUTPUT: $x^*, u^*, \lambda$ \\
	\begin{algorithmic}[1]
		\Procedure{Forward backward sweep}{$g,\lambda_{\text{function}}, 
        u, x_0, 
        \lambda_f, h, n_{max}$} 
			\While{$ \text{test} > \epsilon $}
				\State $u_{\text{old}} \gets u$ 
                \State $x_{\text{old}} \gets x$ 
                \State $ x \gets$ \Call{runge\_kutta\_forward}%
                        {$g, u, x_0, h,n_{max}$}
                \State $\lambda_{\text{old}} \gets \lambda $
				\State $\lambda \gets$ \Call{runge\_kutta\_backward}%
				        {$\lambda_{\text{function}}, x, \lambda_f, h, n_{max}$}
                \State $\displaystyle u_1 \gets$ \Call{optimality\_condition} %
                        {$u, x, \lambda$}
                \State $\displaystyle u \gets \frac{u_1 + u_{old}}{2}$
                \State $test_1 \gets \displaystyle 
                \frac{||u - u_{\text{old}}||}{||u||}$
                \State $test_2 \gets \displaystyle 
                \frac{||x - x_{\text{old}}||}{||x||}$
                \State $test_3 \gets \displaystyle 
                \frac{||\lambda - \lambda_{\text{old}}||}{||\lambda||}$
                \State $\text{test} \gets \max{ \{ test_1, test_2, test_3 \}}$
			\EndWhile\label{}
			\State \textbf{return} $ x^*, u^*, \lambda$
            \Comment{Optimal pair}
		\EndProcedure
	\end{algorithmic}
\end{algorithm}
%---------%---------%---------%---------%---------%---------%---------%---------
%---------%---------%---------%---------%---------%---------%---------%---------

    Now, consider the following maximization problem.
    \begin{equation*}
        \max_{u} \int_{0}^{1} Ax(t) - B(u(t))^{2} dt
    \end{equation*}
    with $A \geq 0$, $B>0$, subject to the dynamics
    \begin{equation*}
        x'(t) = -\dfrac{1}{2}(x(t))^{2} + Cu(t), \, \, x(0) = x_0 >-2.
    \end{equation*}
    Note that the Hamiltonian 
    $$
        H = Ax - Bu^2 - \dfrac{1}{2}\lambda x^2 + C\lambda u.
    $$
    Using the optimility condition, 
    $$
        \dfrac{\partial H}{\partial u} = -2Bu + C\lambda
    $$
    we get $u^{*} = \dfrac{C\lambda}{2B}$. Calculating the adjoint equation we
    find
    \begin{align*}
        x'(t) &= -\dfrac{1}{2}x^2 + Cu, \, \, x(0) = x_0, \\
        \lambda'(t) = -A + x \lambda, \, \, \lambda(1) = 0.
    \end{align*}
    
    

    
    \todo{example 4.1}



%    \section{Preliminaries} %\label{Chap1-Sect1} 



%---------%---------%---------%---------%---------%---------%---------%---------
%---------%---------%---------%---------%---------%---------%---------%---------





\begin{theorem}[Interior Maximum, {\citep[Thm. 19.4, p.~209]{bartle1982elements} Theorem} ]
    Let $c$ be an interior point of the domain of f, at which $f$ has a relative 
    maximum. If the derivative of $f$ at $c$ exists, then it must be equal to zero.
    
\end{theorem}
%---------%---------%---------%---------%---------%---------%---------%---------
%---------%---------%---------%---------%---------%---------%---------%---------
\begin{theorem}{\cite[Thm. 19.5, p.~209]{bartle1982elements}}
    Suppose that $f$ is continuous on a closed interval $J = [a,b]$, that the 
    derivative $f'$ exists in the open interval $(a,b)$, and that 
    $f(a) = f(b) = 0$. Then there exists a point $c \in (a,b)$ such that 
    $f'(c) = 0$.
\end{theorem}
\begin{proof}

\end{proof}
%---------%---------%---------%---------%---------%---------%---------%---------
%---------%---------%---------%---------%---------%---------%---------%---------
\begin{theorem}\cite[Thm. 19.6, p.~210]{bartle1982elements}[Mean Value Theorem] \\
	Suppose that $f$ is continuous on a closed interval $J = [a,b]$ and 
	differentiable on the open interval $(a,b)$. Then there exists $c \in (a,b)$ 
	such that 
	$$
	    f(b) - f(a) = f'(c)(b - a).
	$$
\end{theorem}
\begin{proof}

\end{proof}


%---------%---------%---------%---------%---------%---------%---------%---------
\begin{lemma}\citep[Thm. 4.8, p.~33]{bartle2014elements}[Fatou's Lemma]
    If $(f_n)$ belongs to $M^{+}(X,X)$, then
    $$
        \int \liminf f_n d\mu \leq \liminf \int f_n d\mu
    $$
\end{lemma}
%---------%---------%---------%---------%---------%---------%---------%---------
\begin{corollary}\citep[Thm. 5.4, p.~43]{bartle2014elements}
    If f is measurable, g is integrable and $|f| \leq |g|$, then f is integrable
    and
    $$
        \int |f| d\mu \leq  \int |g| d\mu
    $$
\end{corollary}
%---------%---------%---------%---------%---------%---------%---------%---------
\begin{theorem}\citep[Thm. 5.5, p.~43]{bartle2014elements}
    A constant multiply $\alpha f$ and a sum $f + g$ of functions in $L$ belongs
    to $L$ and
    \begin{align*}
         \int \alpha fd\mu &= \alpha \int f d\mu \\
         \int (f + g)d\mu  &= \int fd\mu + \int gd\mu
    \end{align*}
\end{theorem}
%---------%---------%---------%---------%---------%---------%---------%---------
\begin{theorem}\citep[Thm. 5.6, p.~44]{bartle2014elements}[Lebesgue Dominated Convergence Theorem]
    Let $(f_n)$ be a sequence of integrable functions which converges almost
    everywhere to a real-valued measurable function $f$. If the exists an 
    integrable function $g$ such that $|f_n| < g$ for all n, then f is integrable
    and
    $$
    \int f d\mu = \lim \int f_n d\mu
    $$
\end{theorem}
\begin{proof}

\end{proof}
%---------%---------%---------%---------%---------%---------%---------%---------
\begin{corollary}\citep[Thm. 5.9, p.~46]{bartle2014elements} \\
    Suppose that for some $t_0 \in [a,b]$,  the function $x \rightarrow f(x, t_0)$
    is integrable on $X$, that $\partial f / \partial t$ exists on 
    $X \times [a,b]$, and that there exists an integrable function g on $X$ such 
    that
    $$
        \left|\dfrac{\partial f}{\partial t} (x,t) \right| \leq g(x).
    $$
    Then the function $F(t) = \displaystyle \int f(x,t)d\mu(x)$ is differentiable 
    on $[a,b]$ and
    $$
         \dfrac{dF}{dt} = \dfrac{d }{dt} 
         \int f(x,t)d\mu(x)  = \int \dfrac{\partial f}{\partial t} f(x,t)d\mu(x)
    $$
\end{corollary}
%---------%---------%---------%---------%---------%---------%---------%---------
%---------%---------%---------%---------%---------%---------%---------%---------
\begin{theorem}\citep[Thm.4, p.~391]{}[Taylor's Theorem]
    Suppose that $f', \ldots f^{(n+1)}$, are defined on $[a,x]$ and 
    that $R_{n,a}(x)$ is defined by
    $$
        f(x) = f(a) + f'(a)(x-a) + \dots +
        \frac{f^{(n)}(a)}{n!}(x-a)^{n} + R_{n,a}(x). 
    $$
    Then
    \begin{enumerate}
        \item[i)]
            $R_{n,a}(x) = \dfrac{f^{(n+1)}(t)}{(n)!}(x-t)^{n}(x-a)$ for some 
            $t \in (a,x)$.
        \item[ii)]
            $R_{n,a}(x) = \dfrac{f^{(n+1)}(t)}{(n+1)!}(x-a)^{n+1}$ for some 
            $t \in (a,x)$.
        \item[iii)]
            Moreover, if $f^{(n+1)}$ is integrable on $[a,x]$, then
            $$
                R_{n,a}(x) = \int_{0}^{x}\dfrac{f^{(n+1)}(t)}{(n)!}(x-t)^{n} dt.
            $$
    \end{enumerate}
\end{theorem}
%---------%---------%---------%---------%---------%---------%---------%---------
\begin{theorem}[Lagrange]
    
\end{theorem}
%---------%---------%---------%---------%---------%---------%---------%---------
%---------%---------%---------%---------%---------%---------%---------%---------
\begin{corollary}
    Suppose that for some $t_0 \in [a,b]$, $f(x,t_0) = \lim_{t \to t_0} f(x,t)$
    for each $x \in X$, and that there exists an integrable function g on X 
    such that $|f(t,x)| \leq g(x)$ for all $t\in[a,b]$. Then 
    $$
        \int{f(x,t_0)}d\mu = \lim_{t \to t_0} \int{f(x,t)}d\mu. 
    $$
\end{corollary}

\begin{corollary}
    If the function $t \to f(t,x)$ is continuous on $[a,b]$ for each fixed 
    $x \in X$ and exists $g \in \mathscr{L}$ such that 
    $$
        |f(x,t)| \leq g(x).
    $$
    Then the function
    $$
        F(t) = \displaystyle \int f(x,t)d\mu(x)
    $$
    is continuous on $[a,b]$.
\end{corollary}


%---------%---------%---------%---------%---------%---------%---------%---------
%---------%---------%---------%---------%---------%---------%---------%---------
%    \section{The Forward Backward Sweep Method}
%[theorem name, {\citep[Thm. #, page]{reference}}]
%---------%---------%---------%---------%---------%---------%---------%---------
%---------%---------%---------%---------%---------%---------%---------%---------
    The Pontryagins principle allows us to transform the problem of finding a
    control which optimizes the cost functional to optimizing the Hamiltonian. 
    To this end, we need to solve forward in time the dynamics with a given 
    initial condition, and backward in time the adjoint equation with a 
    transversality condition, using the four-stage  RK method. This way of
    solving is called the forward backward sweep metho. So the idea is to 
    follow the next steps.
    
    \begin{description}
    	\item[Step 1.]
        	Make an initial guess for $\vec{u}$ over the interval.
        \item[Step 2.]
        	Using the initial condition $x_1 = x(t_0) = a$ and the values for 
            $\vec{u}$, solve $\vec{x}$ forward in time according to its differential
            equation in the optimality system.
    	\item[Step 3.]
        	Using the transversality condition $\lambda_{N+1} = \lambda(t_1) = 0$ 
            and the values for $\vec{u}$ and $\vec{x}$, solve $\vec{\lambda}$ 
            backward in time according to its differential equation in the optimality
            system.
        \item[Step 4.]
        	Update $\vec{u}$ by entering the new $\vec{x}$ and $\vec{\lambda}$ values 
            into the characterization of the optimal control. 
    	\item[Step 5.]
        	Check convergence. If the values of the variables in this iteration and 
            the last iteration and the last iteration are negligibly close, output the 
            current values as solutions. If values are not close, return to Step 2.
    \end{description}

    Using Python, we made an implementation of this method \citep{python_Thesisrepo},
    which follows the next  \Cref{FBSM_alg}. 

\begin{algorithm}
	\caption{Forward Backward Sweep }\label{FBSM_alg}
    INPUT: $t_0, t_f, n_{max}, x_0,h, a, r, m, \epsilon, \lambda_{f}$ \\
    OUTPUT: $x^*, u^*, \lambda$ \\
	\begin{algorithmic}[1]
		\Procedure{Forward backward sweep}{$g,\lambda_{\text{function}}, 
        u, x_0, 
        \lambda_f, h, n_{max}$} 
			\While{$ \text{test} > \epsilon $}
				\State $u_{\text{old}} \gets u$ 
                \State $x_{\text{old}} \gets x$ 
                \State $ x \gets$ \Call{runge\_kutta\_forward}%
                        {$g, u, x_0, h,n_{max}$}
                \State $\lambda_{\text{old}} \gets \lambda $
				\State $\lambda \gets$ \Call{runge\_kutta\_backward}%
				        {$\lambda_{\text{function}}, x, \lambda_f, h, n_{max}$}
                \State $\displaystyle u_1 \gets$ \Call{optimality\_condition} %
                        {$u, x, \lambda$}
                \State $\displaystyle u \gets \frac{u_1 + u_{old}}{2}$
                \State $test_1 \gets \displaystyle 
                \frac{||u - u_{\text{old}}||}{||u||}$
                \State $test_2 \gets \displaystyle 
                \frac{||x - x_{\text{old}}||}{||x||}$
                \State $test_3 \gets \displaystyle 
                \frac{||\lambda - \lambda_{\text{old}}||}{||\lambda||}$
                \State $\text{test} \gets \max{ \{ test_1, test_2, test_3 \}}$
			\EndWhile\label{}
			\State \textbf{return} $ x^*, u^*, \lambda$
            \Comment{Optimal pair}
		\EndProcedure
	\end{algorithmic}
\end{algorithm}
%---------%---------%---------%---------%---------%---------%---------%---------
%---------%---------%---------%---------%---------%---------%---------%---------

    Now, consider the following maximization problem.
    \begin{equation*}
        \max_{u} \int_{0}^{1} Ax(t) - B(u(t))^{2} dt
    \end{equation*}
    with $A \geq 0$, $B>0$, subject to the dynamics
    \begin{equation*}
        x'(t) = -\dfrac{1}{2}(x(t))^{2} + Cu(t), \, \, x(0) = x_0 >-2.
    \end{equation*}
    Note that the Hamiltonian 
    $$
        H = Ax - Bu^2 - \dfrac{1}{2}\lambda x^2 + C\lambda u.
    $$
    Using the optimility condition, 
    $$
        \dfrac{\partial H}{\partial u} = -2Bu + C\lambda
    $$
    we get $u^{*} = \dfrac{C\lambda}{2B}$. Calculating the adjoint equation we
    find
    \begin{align*}
        x'(t) &= -\dfrac{1}{2}x^2 + Cu, \, \, x(0) = x_0, \\
        \lambda'(t) = -A + x \lambda, \, \, \lambda(1) = 0.
    \end{align*}
    
    

    
    \todo{example 4.1}

\newpage